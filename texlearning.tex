\documentclass[a4paper,UTF8]{book}
\usepackage{ctex}       % necessary for chinese
\usepackage[margin=1.25in]{geometry}
\usepackage{color}
\usepackage{hyperref}
\usepackage{fancyhdr}
\usepackage{enumitem}
%\usepackage{paralist}
%\usepackage{enumerate}
% \setlength{\leftmargin}{1.2em} %左边界
% \setlength{\parsep}{0ex} %段落间距
% \setlength{\topsep}{0ex} %列表到上下文的垂直距离
% \setlength{\itemsep}{0pt} %条目间距
% \setlength{\labelsep}{0pt} %标号和列表项之间的距离,默认0.5em
% \setlength{\itemindent}{0pt} %标签缩进量
% \setlength{\listparindent}{0pt} %段落缩进量

\usepackage{layout}

\setlength{\evensidemargin}{.25in}
\setlength{\textwidth}{6in}
\setlength{\topmargin}{-0.5in}
\setlength{\topmargin}{-0.5in}

\begin{document}
\title{南京大学工程管理学院\\2018飞跃手册\\预览版本}
\author{Shiqi Lian\\Henry Zhang\\other 2018 Fliers\\201x Fliers }
\maketitle % necessary for title
\tableofcontents
\chapter{前言}
非常荣幸能够邀请到已经在国外经历了一段留学时光的学长学姐们和我们分享留学经验,也感谢一路相互支撑的18飞友们。
有前辈担忧这些过于个人化的经历分享对于学弟学妹们没有很多参考的意义,对此我们的理解是
\chapter{学长学姐留学经历分享}  %document class book required

\newpage
\section{付国峪(CE,Ph.D.@Texas A\&M University)}
\paragraph{留学教育背景:Texas A\&M University (TAMU), Computer Engineering, Ph.D., 2013-2018\\联系方式:fgy108@gmail.com}

\subsection*{在XXX University读XXX专业是怎样的体验?}
    \begin{enumerate}[itemindent=0pt,itemsep=0pt,parsep=0pt,topsep=-40pt]
        \item TAMU在一个小镇上,与最近的大城市(休斯顿)有两小时的车程。生活有一种与世无争的安静和简单。
        \item 研究生课程难度不小,对编程和数学的要求都不低。
        \item TAMU的计算机系较强的方向是机器人和计算视觉。TAMU的电力电子系较强的方向是模电、集成电路、强电、计算机体系结构。
        \item 个人认为TAMU的科研风气崇尚硬实力,鼓励做出有难度的东西,而不鼓励为发论文而发论文。这样对培养学生的能力有好处,毕业出来的学生在工业界很受欢迎。也因为这样,很多学生论文发得并不多,毕业找学术岗位的时候就没有优势了。
    \end{enumerate}
\subsection*{您在读研期间经历过哪些实习/科研,它们的体验是怎样的?}

\subsection*{您现在回顾当初选择留学、选择专业的初衷,在经历了留学生活后有什么新的感受?}

\subsection*{[Ph.D.]您选择了在(香港/新加坡/美国/欧洲/…)攻读博士学位,它和其它选项相比优劣有哪些?就以博士为最终学位的学术深造而言您在出国深造国家的选择上对学弟学妹有什么建议?}

\subsection*{[Master]您选择了在(香港/新加坡/美国/欧洲/…)读master,请问您的选择相比于国内和其他国家地区而言有什么优劣呢?}

\subsection*{[Ph.D.]对于以博士学位为最终学位的深造而言,先在国内读master作为跳板,国外读master作为跳板和本科直接申请Ph.D.三者上有何优劣?您就此对学弟学妹的建议是什么?}

\subsection*{目前南大自动化类的同学每年的出国比例都要低于南大本科生出国比例平均值不少,您认为造成这一现象的原因是什么?您对南大自动化(类)在读的学弟学妹们在出国读研方向的选择上有什么建议吗(劝进/劝退)?}

\subsection*{从现在看留学时光,您会给即将开始留学生涯的大四学弟学妹们什么建议?}

\subsection*{[Ph.D.]就您的了解而言,您目前所读专业的Ph.D.未来发展前景如何,有哪些方向(学术界和工业界),您个人更期待哪一个方向?对于这些方向,在Ph.D.在读期间该做哪些准备?}

\subsection*{[Master]您目前打算毕业后直接就业还是继续读博深造?对于您master之后的去向选择而言,在国内读研和在国外读研有何优劣?就您目前打算的方向而言(读博/工作),在master期间需要做一些怎样的准备(越详细越好)?}

\newpage
\section{XXX(, Ph.D., MS@XXX)}
\paragraph{留学教育背景:XXX\\联系方式:XXX}

\subsection*{在XXX University读XXX专业是怎样的体验?}

\subsection*{您在读研期间经历过哪些实习/科研,它们的体验是怎样的?}

\subsection*{您现在回顾当初选择留学、选择专业的初衷,在经历了留学生活后有什么新的感受?}

\subsection*{[Ph.D.]您选择了在(香港/新加坡/美国/欧洲/…)攻读博士学位,它和其它选项相比优劣有哪些?就以博士为最终学位的学术深造而言您在出国深造国家的选择上对学弟学妹有什么建议?}

\subsection*{[Master]您选择了在(香港/新加坡/美国/欧洲/…)读master,请问您的选择相比于国内和其他国家地区而言有什么优劣呢?}

\subsection*{[Ph.D.]对于以博士学位为最终学位的深造而言,先在国内读master作为跳板,国外读master作为跳板和本科直接申请Ph.D.三者上有何优劣?您就此对学弟学妹的建议是什么?}

\subsection*{目前南大自动化类的同学每年的出国比例都要低于南大本科生出国比例平均值不少,您认为造成这一现象的原因是什么?您对南大自动化(类)在读的学弟学妹们在出国读研方向的选择上有什么建议吗(劝进/劝退)?}

\subsection*{从现在看留学时光,您会给即将开始留学生涯的大四学弟学妹们什么建议?}

\subsection*{[Ph.D.]就您的了解而言,您目前所读专业的Ph.D.未来发展前景如何,有哪些方向(学术界和工业界),您个人更期待哪一个方向?对于这些方向,在Ph.D.在读期间该做哪些准备?}

\subsection*{[Master]您目前打算毕业后直接就业还是继续读博深造?对于您master之后的去向选择而言,在国内读研和在国外读研有何优劣?就您目前打算的方向而言(读博/工作),在master期间需要做一些怎样的准备(越详细越好)?}



\chapter{2018届申请总结}

\newpage
\section{XXX(, Ph.D., MS@XXX)}



\chapter{致谢}
设计参考了计算机系2015年的飞跃手册,在此表达感谢。
\end{document}