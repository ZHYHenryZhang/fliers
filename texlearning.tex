\documentclass[a4paper,UTF8]{book}
\usepackage{ctex}       % necessary for chinese
\usepackage[margin=1.25in]{geometry}
\usepackage{color}
\usepackage{hyperref}
\usepackage{fancyhdr}
\usepackage{enumitem}
%\usepackage{paralist}
%\usepackage{enumerate}
% \setlength{\leftmargin}{1.2em} %左边界
% \setlength{\parsep}{0ex} %段落间距
% \setlength{\topsep}{0ex} %列表到上下文的垂直距离
% \setlength{\itemsep}{0pt} %条目间距
% \setlength{\labelsep}{0pt} %标号和列表项之间的距离,默认0.5em
% \setlength{\itemindent}{0pt} %标签缩进量
% \setlength{\listparindent}{0pt} %段落缩进量

%\usepackage{layout}
\usepackage{indentfirst}
\setlength{\parindent}{0em}
%\setlength{\evensidemargin}{.25in}
%\setlength{\textwidth}{6in}
%\setlength{\topmargin}{-0.5in}
%\setlength{\topmargin}{-0.5in}

%\setlength\parskip{.1\baselineskip}

\usepackage{titlesec}
\titleformat{\subsubsection}{\normalsize\bfseries}{\thesubsubsection}{0em}{}

\begin{document}

\title{南京大学工程管理学院\\2018飞跃手册\\预览版本\\记得去掉Ph.D.和Master的标签}
\author{Shiqi Lian\\Henry Zhang\\other 2018 Fliers\\201x Fliers }
\maketitle % necessary for title

\tableofcontents %目录

\chapter{前言}
非常荣幸能够邀请到已经在国外经历了一段留学时光的学长学姐们和我们分享留学经验,也感谢一路相互支撑的18飞友们。
有前辈担忧这些过于个人化的经历分享对于学弟学妹们没有很多参考的意义,对此我们的理解是
\clearpage

\chapter{学长学姐留学经历分享}  %document class book required

\clearpage
\section{付国峪(CE, Ph.D.@Texas A\&M University)}
    \textbf{留学教育背景:}Texas A\&M University (TAMU), Computer Engineering, Ph.D., 2013-2018\\
    \textbf{联系方式:}fgy108@gmail.com

    \subsubsection*{在Texas A\&M University读Computer Engineering是怎样的体验?}
        \begin{enumerate}[itemindent=0pt,itemsep=0pt,parsep=0pt]
            \item TAMU在一个小镇上,与最近的大城市(休斯顿)有两小时的车程。生活有一种与世无争的安静和简单。
            \item 研究生课程难度不小,对编程和数学的要求都不低。
            \item TAMU的计算机系较强的方向是机器人和计算视觉。TAMU的电力电子系较强的方向是模电、集成电路、强电、计算机体系结构。
            \item 个人认为TAMU的科研风气崇尚硬实力,鼓励做出有难度的东西,而不鼓励为发论文而发论文。这样对培养学生的能力有好处,毕业出来的学生在工业界很受欢迎。也因为这样,很多学生论文发得并不多,毕业找学术岗位的时候就没有优势了。
        \end{enumerate}
    \subsubsection*{您在读研期间经历过哪些实习/科研,它们的体验是怎样的?}
        \begin{enumerate}[itemindent=0pt,itemsep=0pt,parsep=0pt]
            \item 博士期间我是在计算机系做GPS定位方向的。这样的设定对我的软硬件能力要求非常高。软件方面,我得学习计算机的体系、算法,机器学习和优化,分布式系统,实时系统,甚至很多3D几何建模和graphics的知识。硬件方面,我得学习更高级的信号处理,控制论,电路设计,甚至天体物理的知识。
            \item 我的例子没有普遍性。不少的同学都是只做硬件或者软件,只需要专攻一个方向。但总的来说,美国的博士都是不好读的,要做好打硬仗的准备。
            \item 我在谷歌的GPS团队实习。项目与提高安卓手机的GPS精度有关。实习可以提高编程能力,更可以让你接触到最前沿的业界动态。但读博期间不建议过多实习,除非实习项目有利于博士的毕业。博士生应该多花时间在科研上,多学多想,把自己打造成这个领域的先锋。
            \item 当然如果你的志向不在学术上,或者毕业后也不打算继续做这个方向,那就应该早早毕业,或者多实习,锻炼工作能力。
        \end{enumerate}
    \subsubsection*{您现在回顾当初选择留学、选择专业的初衷,在经历了留学生活后有什么新的感受?}
        \begin{enumerate}[itemindent=0pt,itemsep=0pt,parsep=0pt]
            \item 当年留学的初衷是为了提高自己的能力,想学一些别人无法轻易追赶上的本领。在美国读博之后,我确实获得了很大的提升。科研上很多时候都是在做前无古人的事情,完全没有参考资料。这就锻炼了思维、胆量和能力。大家知道,很多研究都是however项目 – 改良一下前人的做法即可。据我所知,美国一流大学的研究很多都是“from scratch”- 推翻前人假设,重新构建一个新的理论或系统。这样操练几年下来,每个优秀的博士毕业生都有了独当一面的功夫。
            \item 当然我说的都是用功、爱学习的同学。水水毕业的博士也不在少数。。。
        \end{enumerate}
    \subsubsection*{[Ph.D.]您选择了在美国攻读博士学位,它和其它选项相比优劣有哪些?就以博士为最终学位的学术深造而言您在出国深造国家的选择上对学弟学妹有什么建议?}
        \begin{enumerate}[itemindent=0pt,itemsep=0pt,parsep=0pt]
            \item 读书都是为了找工作。如果你想以后当教授,建议读个好学校的博士,学校排名越高越好。先读个硕士,然后再申请更好的学校读博也是值得的,毕竟本科毕业直接拿一流大学的offer难度不小。不用介意到了30岁才毕业,很多教授都是30多岁才博士毕业的。如果你在排名50的学校博士毕业,那么你很有可能就在排名50及以后的学校当教授。你未来的学术成就可能就不如在排名30的学校来得高。
            \item 如果你以后想工作,建议越早毕业越好。但是读个硕士是很有帮助的。特别是要想在美国找工作的话,读个好学校的硕士是必须的。读博士就因人而异,你也可以利用博士期间规划人生。出来找工作时,计算机博士会比计算机硕士更占优势,工资也更高。
        \end{enumerate}
    \subsubsection*{[Ph.D.]对于以博士学位为最终学位的深造而言,先在国内读master作为跳板,国外读master作为跳板和本科直接申请Ph.D.三者上有何优劣?您就此对学弟学妹的建议是什么?}
        \begin{enumerate}[itemindent=0pt,itemsep=0pt,parsep=0pt]
            \item 个人认为如果在国内读了master更好是直接找工作。如果还要出来读博,那么读完就快30岁了,你会错过很多机会。等你的同学还是公司中层的时候,或者开始创业的时候,你才刚出来从头做起。
            \item 当然如果你要做教授,为了去个好学校读博,这是完全值得的。
            \item 本科直接申请PhD难度很大,因为硕士毕业生越来越多了,教授们更愿意挑训练有素、会写论文的硕士毕业生,不是吗?
        \end{enumerate}
    \subsubsection*{目前南大自动化类的同学每年的出国比例都要低于南大本科生出国比例平均值不少,您认为造成这一现象的原因是什么?您对南大自动化(类)在读的学弟学妹们在出国读研方向的选择上有什么建议吗(劝进/劝退)?}
        \begin{enumerate}[itemindent=0pt,itemsep=0pt,parsep=0pt]
            \item 数据上不必在意。南大出国比例是被理科拉高的。理科生出国容易、就业难,他们可选的路不多。
            \item 建议不要随大流,认真想想自己想要什么样的生活。留在国内,不继续深造,国内读研,或者国外读研,哪一条路都有秀出班行的例子。南大人都不用担心生存问题。南大人要考虑的,是怎样定义自己的成功,然后选一条路去实现它。
        \end{enumerate}
    \subsubsection*{从现在看留学时光,您会给即将开始留学生涯的大四学弟学妹们什么建议?}
        \begin{enumerate}[itemindent=0pt,itemsep=0pt,parsep=0pt]
            \item 多尝试新鲜事物,多出去走走看看。融入当地的环境。
            \item 相比起来,国内风气比较浮躁。在国外,会慢慢回归自我,思考清楚人生的意义。
        \end{enumerate}
    \subsubsection*{[Ph.D.]就您的了解而言,您目前所读专业的Ph.D.未来发展前景如何,有哪些方向(学术界和工业界),您个人更期待哪一个方向?对于这些方向,在Ph.D.在读期间该做哪些准备?}
        \begin{enumerate}[itemindent=0pt,itemsep=0pt,parsep=0pt]
            \item 现在AI和机器学习非常流行,工业界对这些岗位的需求也非常大。如果你们现在就从这些专业毕业,那当然会是很吃香。但是5年之后这些行业是不是还热门,AI泡沫会不会破灭,这都很难说。我自己的GPS方向需求不大,而且难度太高,不建议入坑。
            \item 只要是实用的,对数学、编程有提高的,就是好方向。过于理论的,专为发论文而发论文的,就是差方向。很多博士毕业去工业界,都不是做原来的方向。
        \end{enumerate}

\clearpage
\section{倪慧佳(Marketing, MS@Columbia University)}
    \textbf{留学教育背景:}Columbia University, MS in Marketing, 2015\\
    \textbf{联系方式:}WeChat:giraffe1122334

    \subsubsection*{在Columbia University读Marketing是怎样的体验?}
    哥大的市场营销硕士是哥大商学院唯二的硕士项目之一,而且program比较小,全球仅招十几人。好处是除了本身专业的必修课之外,商院内MBA和PHD的课可以随便选择,所以读一个硕士,可以同时感受三种生活。毕业不是光修完课程就可以,还需要写一篇高质量的毕业论文,整体压力很大。
    \subsubsection*{您在读研期间经历过哪些实习/科研,它们的体验是怎样的?}
    在做毕设的同时也做了暑期实习,在一家纽约的广告公司工作,agency比较小,没有独立的楼层,租的是WeWork的工位,会碰到带狗来上班的同事,氛围很好。

    毕设也很有趣,研究的是百老汇剧预售票的影响因素,以及各因素的影响能力。导师会帮助联系各个百老汇剧的制作公司,逐一签订保密协议后拿到大量一手销售数据进行研究。
    \subsubsection*{您现在回顾当初选择留学、选择专业的初衷,在经历了留学生活后有什么新的感受?}
    觉得自己的选择还是正确的,商科master的学习,区位优势很重要。CBS的slogan就是in the very center of business.  在宇宙中心纽约,各种机会更多。 
    \subsubsection*{[Master]您选择了在美国读master,请问您的选择相比于国内和其他国家地区而言有什么优劣呢?}
    优势:纽约作为国际大都市,机会更多,能看到的读到的体验到的东西多
    劣势:纽约的生活成本会比较高
    \subsubsection*{目前南大自动化类的同学每年的出国比例都要低于南大本科生出国比例平均值不少,您认为造成这一现象的原因是什么?您对南大自动化(类)在读的学弟学妹们在出国读研方向的选择上有什么建议吗?}
    自动化类在国内的就业机会就很多,所以很多同学不一定会选择出国读研。有机会有条件的话当然出去看看更好,master也普遍比国内节约一年时间。
    \subsubsection*{从现在看留学时光,您会给即将开始留学生涯的大四学弟学妹们什么建议?}
    充分利用好课余时间,多做一些项目或者实习。不要局限在中国人的小圈子。
    \subsubsection*{[Master]您目前打算毕业后直接就业还是继续读博深造?对于您master之后的去向选择而言,在国内读研和在国外读研有何优劣?就您目前打算的方向而言(读博/工作),在master期间需要做一些怎样的准备?}
    Master毕业之后回国工作了。如果在外企工作的话,国外读研的经历帮助更大,在如何与外籍老板同事打交道方面更得心应手。如果毕业就打算工作的话,一定要从入学第一天就开始注重就业问题。国外院校一般都有专门的老师负责Career 方面的咨询和训练,要多和他们沟通,不断完善简历和cover letter,面试前找老师进行模拟面试等。

\clearpage
\section{岳翔宇(EE, MS@Stanford)}
    \textbf{留学教育背景:}Stanford EE MS, 2014,UC Berkeley EECS PhD 2016\\
    \textbf{联系方式:}微信:yuexiangyu618,邮箱: xyyue@eecs.berkeley.edu

    \subsubsection*{在UCB读EECS是怎样的体验?}
        \begin{enumerate}[itemindent=0pt,itemsep=0pt,parsep=0pt]
            \item 课程:
            硕士:每个quarter三门课,课业量蛮大的,跟本科感觉不是一个级别的;有的时候还会通宵赶作业什么的;GPA对于找工作或者申请PhD还是有一定帮助的,尤其是找工作。
            博士:每学期课可以选的少一些,而且压力也没有那么大了;博士期间的课程要求都挺容易达到的,而且成绩也没有那么重要了。
            \item 科研:
                硕士:硕士科研的话还是要主动找教授,而且可能教授不太会一开始就给funding(当然这个不同学校,不同教授都不太一样);如果不给的话那就可能要先免费做一个学期证明一下自己,后面再拿funding。
                博士:博士科研的话;感觉还是有一定压力的,一方面来自老师以及合作者对于项目的要求,另一方面来自自己吧, 给自己找一个合适的,比较promising的,能做出成果的方向。
            \item 生活:
                硕士:个人感觉如果毕业想找工作的话生活会轻松一些,尽管找工作也会有压力;但我觉得如果打算硕士毕业申请博士的话可能压力稍微更大一点;毕竟换工作容易,换PhD学校,换PhD导师,或者换PhD的方向那就难多了。
                博士:生活的丰富程度还是可以的,虽然经常会因为project周末到学校加班。感觉时间支配还算自由。
        \end{enumerate}
    \subsubsection*{您在读研期间经历过哪些实习/科研,它们的体验是怎样的?}
        \begin{enumerate}[itemindent=0pt,itemsep=0pt,parsep=0pt]
            \item 实习:
            硕士:SAP实习,公司氛围环境还算不错吧。当时打算申请phd,没太多实习;这个实习也是当时上课的一个lecturer给推荐的。如果找工作的话实习还是挺关键的,要花不少时间去刷题,面试等等。
            博士:腾讯 Seattle。做的项目跟自己在学校做的非常match。公司中国人偏多,加班人感觉比SAP要更多一些,别的方面的话差别倒是没那么明显。
            \item 科研:
            硕士:第一年想找老师做computer architecture 方向的research,当时由于本科时期做的项目偏少,好多老师都不太给科研的机会,不拿funding做了两个学期,也没太多成果。第二年找到一个并行计算方面的老师,先免费做了一个学期,之后一个quarter给了RA;后来发现不是很感兴趣,再之后一个quarter又找了另外一个CS的教授做RA。吐槽一下,当时找教授做科研很曲折不顺利的时候还是有点痛苦的。不过不同学校不同专业可能都不一样吧,我也只是把经历给大家坐下参考。别的学长学姐有的可能会顺利不少。
            博士:博士科研还是蛮看老师的。老师的方向,老师的期望,老师的network。当然还是主要看自己,尽快找到一个感兴趣的,有动力做下去的方向。
        \end{enumerate}
    \subsubsection*{您现在回顾当初选择留学、选择专业的初衷,在经历了留学生活后有什么新的感受?}
    没有后悔出来留学吧应该,毕竟感受到了不同的生活方式,校园文化。不过国内还是好玩多了,不管是娱乐还是饮食。LOL
    \subsubsection*{[Ph.D.]您选择了在(香港/新加坡/美国/欧洲/…)攻读博士学位,它和其它选项相比优劣有哪些?就以博士为最终学位的学术深造而言您在出国深造国家的选择上对学弟学妹有什么建议?}
    不太好回答,美国的优势的话,其实我个人感觉主要是可能优秀的学校,科研人员比其他国家整体上更多一些吧。科研氛围应当也算是领先的,当然我也没去过其他国家的学校,不太好评论。
    我觉得还是学校,导师,方向最重要吧;国家没那么重要。当然在申请到的学校,导师,方向差不多的情况下,我还是推荐来美国的。如果申请时候就打算申请一个国家的话,如果托福,GRE都还可以的话,我建议美国吧。
    \subsubsection*{[Ph.D.]对于以博士学位为最终学位的深造而言,a先在国内读master作为跳板,b国外读master作为跳板和c本科直接申请PHD三者上有何优劣?您就此对学弟学妹的建议是什么?}
        我在上面标了a,b,c。
        \begin{enumerate}[itemindent=0pt,itemsep=0pt,parsep=0pt]
        \item a:\\
        优势:能够有机会硕士期间发表更多文章,有利于博士申请结果更好;跟国内的老师的关系搞得不错的话对于将来回国也是很好的。\\
        劣势:可能在国内多读了2,3年,因为国内读完硕士出国读博士通常还是需要5年左右的。
        \item b:\\
        优势:能够提前了解国外的学习生活习惯,提前跟国外的老师做research,如果做得好的话,老师可能就直接接受你转PhD了。也有时间explore不同的方向。\\
        劣势:国外硕士一般要自费,当然出去之后再拿TA,RA机会也是不小的,当然不同学校,院系不一样。
        \item c:\\
        优势:省时间,省钱。\\
        劣势:成果少一些,申请结果相对差一些,而且explore,选择research方向的自由度小。
        \end{enumerate}
    \subsubsection*{目前南大自动化类的同学每年的出国比例都要低于南大本科生出国比例平均值不少,您认为造成这一现象的原因是什么?您对南大自动化(类)在读的学弟学妹们在出国读研方向的选择上有什么建议吗?}
    我觉得原因是出国的氛围还是不是很强烈吧,学生也比他们要少一些。我觉得不要管别的院吧,自己做好了还是一切皆有可能的(我们那年除了我,丁家琛学长也申请到了斯坦福)。我还是觉得出国是一个不错的选择的,劝进啊。做好决定之后就努力去拼。中间遇到困难不要灰心,我当时托福就考了6次。。
    \subsubsection*{从现在看留学时光,您会给即将开始留学生涯的大四学弟学妹们什么建议?}
    淡定,放松。大四很多跟同学的时光都是我至今很怀念的。课余时间可以提前补充 一些觉得对自己将来国外生活有帮助的知识。
    \subsubsection*{[Ph.D.]就您的了解而言,您目前所读专业的Ph.D.未来发展前景如何,有哪些方向(学术界和工业界),您个人更期待哪一个方向?}
    每个PhD其实做的方向真的很窄;我现在做simulation在自动驾驶中的应用相关的项目,学术界和工业界感觉都还可以。


\clearpage
\section{应宙锋(EE, Ph.D.@UT Austin)}
    \textbf{留学教育背景:}UT Austin, optical computing, 2016\\
    \textbf{联系方式:}yjcyzf@gmail.com

    \subsubsection*{在UT Austin读EE是怎样的体验?}
    其实博士生的生活大部分取决于自己的研究小组,每个小组的科研方向,管理模式和科研压力都不同,甚至可以说天差地别。有些组的学生每天在世界各地环游,有些组的学生每晚都在实验室度过。我本人还比较喜欢我目前的科研状态,每周和老板讨论一次,大部分时间自己做研究自己安排生活,偶尔出去开会作报告。这里有比较大的超净间还有很多仪器设备,所以想法很容易得到实验验证。有很多公司也在这里加工芯片,白天人会比较多,所以我喜欢晚上在这里开展实验。
    
    奥斯汀是个比较宜居的城市,也是个音乐城市,经常有大大小小的音乐节,比较有名的是西南偏南,是个每年一次的盛会。冬天这里不冷,很少下雪,但夏天稍微有点热,幸好室内都有空调。奥斯汀对篮球迷来说也还不错,附近有圣安东尼奥的马刺,有休斯顿的火箭,有达拉斯的小牛。喜欢钓鱼的朋友也可以开车三小时去海边垂钓,有朋友还钓到过小鲨鱼。我也曾和朋友开车十二个小时北上去科罗拉多的丹佛滑雪。    
    
    \subsubsection*{您现在回顾当初选择留学、选择专业的初衷,在经历了留学生活后有什么新的感受?}
    出国的初衷也是想出来开阔眼界,学习点高精尖的技术。现在也努力在这条路上走。来了以后还是挺喜欢这里的生活的,因为比较自由,各方面的自由。
    
    \subsubsection*{[Ph.D.]您选择了在美国攻读博士学位,它和其它选项相比优劣有哪些?就以博士为最终学位的学术深造而言您在出国深造国家的选择上对学弟学妹有什么建议?}
    美国毕竟还是科技最领先的国家。但就我这个方向(光集成芯片)来说,欧洲也比较强,他们起步更早。要我给建议的话,我觉得不能一概而论,每个方向不一样,特别是读博士,一定要慎重。方向和老板都很重要,做好前期调研工作,多问问同组或者学校的师兄师姐。至于三种出国的方式我觉得都可以,国外的制度挺灵活,你进来phd可以转成master,也可以从master转成phd,也可以换组换导师转专业。关键是自己想清楚要不要读博,读什么方向。选个自己感兴趣的,能坚持的,而且四五年后等你毕业能够给社会带来价值的方向。

    \subsubsection*{[Ph.D.]对于以博士学位为最终学位的深造而言,先在国内读master作为跳板,国外读master作为跳板和本科直接申请Ph.D.三者上有何优劣?您就此对学弟学妹的建议是什么?}
    至于三种出国的方式我觉得都可以,国外的制度挺灵活,你进来phd可以转成master,也可以从master转成phd,也可以换组换导师转专业。关键是自己想清楚要不要读博,读什么方向。选个自己感兴趣的,能坚持的,而且四五年后等你毕业能够给社会带来价值的方向。

    \subsubsection*{目前南大自动化类的同学每年的出国比例都要低于南大本科生出国比例平均值不少,您认为造成这一现象的原因是什么?您对南大自动化(类)在读的学弟学妹们在出国读研方向的选择上有什么建议吗?}
    可能跟风气有关吧,而且国内现在发展也挺好的,出国的越来越少也正常。个人觉得出国看看挺好的,能长长见识,丰富下生活经历。
    
    \subsubsection*{从现在看留学时光,您会给即将开始留学生涯的大四学弟学妹们什么建议?}
    享受生活吧,不要被社会给你的条条框框束缚住,别在乎别人的闲言闲语,趁年轻,该干什么干什么。每个人都有自己的pace。有些人的人生是标准模式,什么时间该干什么,但你的人生应该由你自己做主,因为你是高级玩家,你走自定义模式。
    
    \subsubsection*{[Ph.D.]就您的了解而言,您目前所读专业的Ph.D.未来发展前景如何,有哪些方向(学术界和工业界),您个人更期待哪一个方向?对于这些方向,在Ph.D.在读期间该做哪些准备?}
    我这个方向应该是比较重要的方向,欧洲在十几年前起步,美国正在加足马力追赶。有个业内人士说,现在的光芯片就是八十年代的电芯片一样正处于奇点。所以比较期待工业界的发展。

\clearpage
\section{张缙颔(ECE, MS@UCSD)}
    \textbf{留学教育背景:}UCSD, 专业方向Intelligent Systems, Robotics, and Control,2016年入学\\
    \textbf{联系方式:}Jinhan.zhang94@gmail.com

    \subsubsection*{在UCSD读ECE是怎样的体验?}
    SD的天气和环境真的是好的没话讲啊,La Jolla也是全美治安排名前几的社区,人也大都非常nice,可能是四季如春的气候让大家都很happy吧。课程的话,SD的CSE的课程质量都还算比较高,也有蛮多大牛教授开的课,课程比较实用,对于想学CS的筒子们最好多选一些CSE的课啦。至于ECE的课,大多数并不推荐,很多prof的课非常理论,所以如果不是想走research的话,推荐就只选毕业所需的课。
    
    \subsubsection*{您在读研期间经历过哪些实习/科研,它们的体验是怎样的?}
    因为我个人的打算就是找工作,所以没有什么科研的相关经历。我的实习是在SD当地的一家自动驾驶公司,因为是华人偏多的创业公司,所以并没有美国公司著称的Work-life balance。它们东西都推得挺快的,所以学东西也会相对偏快一些。
    
    \subsubsection*{您现在回顾当初选择留学、选择专业的初衷,在经历了留学生活后有什么新的感受?}
    我个人的想法就是人生在世重在体验嘛 = =,所以留学也算是体验不同生活的一部分啦~美国的教育和中国的还是有非常大的差异的, 整体节奏会比国内快蛮多。读研的时候大多数人都非常辛苦,强度整体还是比国内本科大蛮多,现在工作之后会慢慢回归正常的生活节奏。

    \subsubsection*{目前南大自动化类的同学每年的出国比例都要低于南大本科生出国比例平均值不少,您认为造成这一现象的原因是什么?您对南大自动化(类)在读的学弟学妹们在出国读研方向的选择上有什么建议吗?}
    个人感觉劝退劝进脱离个体情况不太好说,总之看大家对自己未来的期许是怎样的吧。国外留学生活也是有好有坏,每个人性格和习惯的不同也会导致体验上的巨大差异。

    \subsubsection*{从现在看留学时光,您会给即将开始留学生涯的大四学弟学妹们什么建议?}
    提前准备,找工作的话提前刷题做project丰富简历, 科研的话体验联系老板,过来刷GPA,争取早日进实验室。

    \subsubsection*{[Master]您目前打算毕业后直接就业还是继续读博深造?对于您master之后的去向选择而言,在国内读研和在国外读研有何优劣?就您目前打算的方向而言(读博/工作),在master期间需要做一些怎样的准备?}
    上班党已经入职啦。。美国读研最直观的优势就是可以在美国找工作吧,至于美国工作的优势大概就是大多数公司的work-life balance比较好,会有更多自己的时间。要说技术水平的话,其实有多领域国内很多公司是比美国的公司厉害的,但是缺点可能就是工作压力会偏大(当然这也不一定是确定)。对于找工作来讲,准备的话就是提前刷题,同时注意丰富自己的简历吧,其他真的没有太多捷径可走。




\clearpage
\section{彭凤超(CSE, Ph.D.@HKUST)}
    \textbf{留学教育背景:}Hong Kong University of Science and Technology,CSE,2014入学

    \subsubsection*{在HKUST读CSE是怎样的体验?}
    港科大CSE的课程设置涵盖了所有CS领域热门的研究领域,如AI,Database,Network,Software Engineering,算法,密码学等,并且会随着CS的发展潮流,适当的调整各门课程的开设频率,甚至是增开新课程,比如近两年新开的并行计算方面的课程。本科生的课程则是全面覆盖。授课形式以lecture为主,考核方式基本采用作业/project/考试的方式,一门课几种方式可能兼而有之。师资力量当属国际一流水平,所有的教授都会开课,也都会参与本科生的毕业论文工作,每门课都有本系优秀的研究生、博士生【夸自己一下咩哈哈】担任助教。可以说课程质量绝对杠杠的。
    \subsubsection*{您在读研期间经历过哪些实习/科研,它们的体验是怎样的?}
    我的科研经历基本就是做了几篇文章,文章质量么,很惭愧,除了一篇顶会文章以外,难说优质。感触就是做论文主要还是靠自己,第二作者能贡献10\%的力量就很好了,不只是我,我的很多同学都是这样。多读文章、多讨论、多尝试【不管想法看起来多扯淡】,再加上一定的运气,以及导师关键的一点点指导,文章才能成,读博还是挺艰难的。我有两段实习经历,一个在阿里,一个在地平线。阿里的体验就是,毕竟大厂,所有事情有章可循,分工明确,办事情效率很高,员工日常985本硕,海归phd一搂一大把。地平线规模不大,但技术实力不遑多让,AI芯片的研发已经走在国内前列。两次实习都是参与到了一个项目中去,任务很具体,数据清洗,设计模型,实现模型,调参,反复尝试bla。对于练手、寻找研究idea、积累简历、近距离接触业内大佬,都有帮助。
    \subsubsection*{您现在回顾当初选择留学、选择专业的初衷,在经历了留学生活后有什么新的感受?}
    还是应该申请之前就仔细了解一下,自己喜欢做哪方面的研究,不要为了读博而读博,我进门以后,万幸遇到一个我有兴趣的方向,回想一下是有些后怕的。
    \subsubsection*{[Ph.D.]您选择了在(香港/新加坡/美国/欧洲/…)攻读博士学位,它和其它选项相比优劣有哪些?就以博士为最终学位的学术深造而言您在出国深造国家的选择上对学弟学妹有什么建议?对于以博士学位为最终学位的深造而言,先在国内读master作为跳板,国外读master作为跳板和本科直接申请Ph.D.三者上有何优劣?您就此对学弟学妹的建议是什么?}
    我没有去欧美学校的经历,所以难说香港的phd和国外的有什么优劣之分。至于先读master还是直接phd,我觉得如果你很确定就研究某一个方向,且确定要读博,那就直接申博士,没有必要先读个master浪费时间。当然了,如果你要读名校phd,本科直接申请可能申不上,拿个master做跳板,蹭几封大佬推荐信,是个很好的选择。

    \subsubsection*{目前南大自动化类的同学每年的出国比例都要低于南大本科生出国比例平均值不少,您认为造成这一现象的原因是什么?您对南大自动化(类)在读的学弟学妹们在出国读研方向的选择上有什么建议吗?}
    我觉得原因是自动化这个专业,在海外基本没有了,我们出去要么申EE,要么CS,可能这会让一部分同学觉得出去也学不到啥,专业也不对口。但事实,完全不是这样,海外招生对于专业的要求十分不严格,EE、CS方向,你只要能和编程、算法、电路沾上边就行,有的cs老板还特爱招物理系、数学系毕业的学生。所以大胆的申啊!康忙北鼻动特比晒
    \subsubsection*{从现在看留学时光,您会给即将开始留学生涯的大四学弟学妹们什么建议?}
    找个女朋友
    \subsubsection*{[Ph.D.]就您的了解而言,您目前所读专业的Ph.D.未来发展前景如何,有哪些方向(学术界和工业界),您个人更期待哪一个方向?对于这些方向,在Ph.D.在读期间该做哪些准备?}
    cs,具体到ai领域,出路主要是继续深造走学术道路,或者进企业,进企业也分去业务部门和去研究部门,前者是应用导向,后者和学术道路没差。像我这种文章少的只能选择去企业了。
    \subsubsection{想说的话:}
    找个女朋友
        
\clearpage
\section{李珽光(EE, Ph.D.@CUHK)}
    \textbf{留学教育背景:}香港中文大学-电子工程系EE-2016入学Ph.D.\\
    \textbf{联系方式:} tgli0809@gmail.com
    \textbf{个人主页:} www.ee.cuhk.edu.hk/~tgli/

    \subsubsection*{在香港中文大学读EE是怎样的体验?}
    先说位置,香港中文大学(CUHK)在新界,距离口岸坐地铁只有半个小时,是香港所有大学里面去深圳最方便的一个。另外因为离市区较远,因此CUHK占地面积据说是香港其他六所大学之和,相对香港其他学校,CUHK的phd除了第一年可能申请不到宿舍之外这几年都可以住在学校,这是其他学校所没有的。
    总体上,香港的课程难度,工作量,有用程度应该是远胜于内地的,我们这届的phd要求至少修五门(我的下一届改成七门了),所以第一年和第二年上半年基本都在挣扎于课程。但作为phd十分矛盾的一点是,科研成果有时候和课程关系不大,而香港这边学制又比较短,所以需要权衡两者之间的关系。
    科研上比较看方向看导师,不同导师风格不同毕业要求也不同。我们导师散养风格,一切靠自己。
    \subsubsection*{您在读研期间经历过哪些实习/科研,它们的体验是怎样的?}
    我们实验室研究方向是机器人,我个人是强化学习用于地面移动机器人(详见我的个人主页)。总体而言科研是一个挺痛苦的过程,它和本科阶段上课考试完全不一样,因为你做的东西往往没有正确答案,需要不停的摸索,同时紧跟时代的方向,就是努力了也不一定会出成绩。但是当文章发表,去参加会议的时候又特别幸福,总体上是喜乐参半吧。
    \subsubsection*{您现在回顾当初选择留学、选择专业的初衷,在经历了留学生活后有什么新的感受?}
    我来香港的动机很简单,这边的Ph.D.时间短,性价比高,而且申请流程相对简单,不需要考GRE。现在来看这个选择也没什么问题,适合懒人。
    \subsubsection*{[Ph.D.]您选择了在香港攻读博士学位,它和其它选项相比优劣有哪些?就以博士为最终学位的学术深造而言您在出国深造国家的选择上对学弟学妹有什么建议?对于以博士学位为最终学位的深造而言,先在国内读master作为跳板,国外读master作为跳板和本科直接申请Ph.D.三者上有何优劣?您就此对学弟学妹的建议是什么?}
    这个我着重讲讲,欧洲情况我不太清楚,澳洲留学据说比较水,我着重比较比较美国香港和新加坡。美国的phd(人工智能方向)最近很难申请,用我朋友的话说就是神仙打架,很多申请人本科就坐拥好几篇顶会paper,但是美国phd质量非常高,前景也非常好,可以先申请美国master作为跳板,至于国内读个master作跳板我个人感觉没必要。美国适合很早就开始准备出国(包括语言成绩,GPA,比赛,交换,尽早进实验室进行科研)的同学。剩下新加坡和香港则各有利弊。新加坡的两所学校排名更靠前,新加坡更国际化一些,但香港优势在于紧邻大陆,暑期回去实习非常方便,同时对国内政策,产业情况都比较清楚,可以提早打下基础。
    我认为香港非常适合那些GPA很靠前,但是出国准备的晚了的或者之前只准备保研的同学,因为每年这边都会有summer workshop招收这样的学生,我觉得算是一条捷径。

    \subsubsection*{目前南大自动化类的同学每年的出国比例都要低于南大本科生出国比例平均值不少,您认为造成这一现象的原因是什么?您对南大自动化(类)在读的学弟学妹们在出国读研方向的选择上有什么建议吗?}
    还是没有形成出国的风气,大家对于出国了解的比较少。我觉得在家庭经济条件允许的情况下相比国内读研,我还是倾向国外读,美国phd>香港新加坡phd = 清华phd(清华势头很好,可以认真考虑)>美国master>国内master。个人看法仅供参考。
    
    \subsubsection*{从现在看留学时光,您会给即将开始留学生涯的大四学弟学妹们什么建议?}
    好好的玩吧,不要在这个时间假装学习了。
    
    \subsubsection*{[Ph.D.]就您的了解而言,您目前所读专业的Ph.D.未来发展前景如何,有哪些方向(学术界和工业界),您个人更期待哪一个方向?对于这些方向,在Ph.D.在读期间该做哪些准备?}
    目前人工智能产业正处于风口浪尖,工业界的待遇很可观。PHD期间适当多关注新闻,了解产业情况,条件允许的情况下可以进入公司实习。

    \subsubsection{想说的话:}
    相比保研,考研,其实出国的路更难走,需要尽早进行准备。祝学弟学妹们好运!

        
        
\clearpage
\section{XXX(, Ph.D., MS@XXX)}
    \textbf{留学教育背景:}\\
    \textbf{联系方式:}XXX

    \subsubsection*{在XXX University读XXX专业是怎样的体验?}
        \begin{enumerate}[itemindent=0pt,itemsep=0pt,parsep=0pt]
            \item 难道是
            \item 因为
        \end{enumerate}
    \subsubsection*{您在读研期间经历过哪些实习/科研,它们的体验是怎样的?}

    \subsubsection*{您现在回顾当初选择留学、选择专业的初衷,在经历了留学生活后有什么新的感受?}

    \subsubsection*{[Ph.D.]您选择了在(香港/新加坡/美国/欧洲/…)攻读博士学位,它和其它选项相比优劣有哪些?就以博士为最终学位的学术深造而言您在出国深造国家的选择上对学弟学妹有什么建议?}

    \subsubsection*{[Master]您选择了在(香港/新加坡/美国/欧洲/…)读master,请问您的选择相比于国内和其他国家地区而言有什么优劣呢?}

    \subsubsection*{[Ph.D.]对于以博士学位为最终学位的深造而言,先在国内读master作为跳板,国外读master作为跳板和本科直接申请Ph.D.三者上有何优劣?您就此对学弟学妹的建议是什么?}

    \subsubsection*{目前南大自动化类的同学每年的出国比例都要低于南大本科生出国比例平均值不少,您认为造成这一现象的原因是什么?您对南大自动化(类)在读的学弟学妹们在出国读研方向的选择上有什么建议吗?}

    \subsubsection*{从现在看留学时光,您会给即将开始留学生涯的大四学弟学妹们什么建议?}

    \subsubsection*{[Ph.D.]就您的了解而言,您目前所读专业的Ph.D.未来发展前景如何,有哪些方向(学术界和工业界),您个人更期待哪一个方向?对于这些方向,在Ph.D.在读期间该做哪些准备?}

    \subsubsection*{[Master]您目前打算毕业后直接就业还是继续读博深造?对于您master之后的去向选择而言,在国内读研和在国外读研有何优劣?就您目前打算的方向而言(读博/工作),在master期间需要做一些怎样的准备?}
        
        
\clearpage
\section{XXX(, Ph.D., MS@XXX)}
    \textbf{留学教育背景:}\\
    \textbf{联系方式:}XXX

    \subsubsection*{在XXX University读XXX专业是怎样的体验?}
        \begin{enumerate}[itemindent=0pt,itemsep=0pt,parsep=0pt]
            \item 难道是
            \item 因为
        \end{enumerate}
    \subsubsection*{您在读研期间经历过哪些实习/科研,它们的体验是怎样的?}

    \subsubsection*{您现在回顾当初选择留学、选择专业的初衷,在经历了留学生活后有什么新的感受?}

    \subsubsection*{[Ph.D.]您选择了在(香港/新加坡/美国/欧洲/…)攻读博士学位,它和其它选项相比优劣有哪些?就以博士为最终学位的学术深造而言您在出国深造国家的选择上对学弟学妹有什么建议?}

    \subsubsection*{[Master]您选择了在(香港/新加坡/美国/欧洲/…)读master,请问您的选择相比于国内和其他国家地区而言有什么优劣呢?}

    \subsubsection*{[Ph.D.]对于以博士学位为最终学位的深造而言,先在国内读master作为跳板,国外读master作为跳板和本科直接申请Ph.D.三者上有何优劣?您就此对学弟学妹的建议是什么?}

    \subsubsection*{目前南大自动化类的同学每年的出国比例都要低于南大本科生出国比例平均值不少,您认为造成这一现象的原因是什么?您对南大自动化(类)在读的学弟学妹们在出国读研方向的选择上有什么建议吗?}

    \subsubsection*{从现在看留学时光,您会给即将开始留学生涯的大四学弟学妹们什么建议?}

    \subsubsection*{[Ph.D.]就您的了解而言,您目前所读专业的Ph.D.未来发展前景如何,有哪些方向(学术界和工业界),您个人更期待哪一个方向?对于这些方向,在Ph.D.在读期间该做哪些准备?}

    \subsubsection*{[Master]您目前打算毕业后直接就业还是继续读博深造?对于您master之后的去向选择而言,在国内读研和在国外读研有何优劣?就您目前打算的方向而言(读博/工作),在master期间需要做一些怎样的准备?}
        
            
        
\clearpage
\section{XXX(, Ph.D., MS@XXX)}
    \textbf{留学教育背景:}\\
    \textbf{联系方式:}XXX

    \subsubsection*{在XXX University读XXX专业是怎样的体验?}
        \begin{enumerate}[itemindent=0pt,itemsep=0pt,parsep=0pt]
            \item 难道是
            \item 因为
        \end{enumerate}
    \subsubsection*{您在读研期间经历过哪些实习/科研,它们的体验是怎样的?}

    \subsubsection*{您现在回顾当初选择留学、选择专业的初衷,在经历了留学生活后有什么新的感受?}

    \subsubsection*{[Ph.D.]您选择了在(香港/新加坡/美国/欧洲/…)攻读博士学位,它和其它选项相比优劣有哪些?就以博士为最终学位的学术深造而言您在出国深造国家的选择上对学弟学妹有什么建议?}

    \subsubsection*{[Master]您选择了在(香港/新加坡/美国/欧洲/…)读master,请问您的选择相比于国内和其他国家地区而言有什么优劣呢?}

    \subsubsection*{[Ph.D.]对于以博士学位为最终学位的深造而言,先在国内读master作为跳板,国外读master作为跳板和本科直接申请Ph.D.三者上有何优劣?您就此对学弟学妹的建议是什么?}

    \subsubsection*{目前南大自动化类的同学每年的出国比例都要低于南大本科生出国比例平均值不少,您认为造成这一现象的原因是什么?您对南大自动化(类)在读的学弟学妹们在出国读研方向的选择上有什么建议吗?}

    \subsubsection*{从现在看留学时光,您会给即将开始留学生涯的大四学弟学妹们什么建议?}

    \subsubsection*{[Ph.D.]就您的了解而言,您目前所读专业的Ph.D.未来发展前景如何,有哪些方向(学术界和工业界),您个人更期待哪一个方向?对于这些方向,在Ph.D.在读期间该做哪些准备?}

    \subsubsection*{[Master]您目前打算毕业后直接就业还是继续读博深造?对于您master之后的去向选择而言,在国内读研和在国外读研有何优劣?就您目前打算的方向而言(读博/工作),在master期间需要做一些怎样的准备?}
                    




\chapter{2018届申请总结}

\clearpage
\section{XXX(, Ph.D., MS@XXX)}
    \subsection*{个人背景}
        \textbf{GPA:}overall xx/xx, major xx/xx\\
        \textbf{Ranking:}xx/xx\\
        \textbf{TOEFL(R/L/S/W):}xxx (xx/xx/xx/xx)\\
        \textbf{GRE(V+Q+AW):}xx+xx+\\
        \textbf{推荐信:}\\
        \textbf{科研/实习/交流:}\\ 
        \textbf{联系方式:}

    \subsection*{申请结果}
        \textbf{AD:}\\
        \textbf{Rej:}\\
        \textbf{Accept:}

    \subsection*{申请的前期准备}
        \subsubsection*{选择}
        \subsubsection*{GPA}
        \subsubsection*{科研/实习/比赛/交流}
        \subsubsection*{GRE/TOEFL}

    \subsection*{申请过程}
        \subsubsection*{选校}
        \subsubsection*{文书}
        \subsubsection*{推荐信}
        \subsubsection*{网申}
        \subsubsection*{等待结果}
        \subsubsection*{项目介绍}

    \subsection*{总结}

    \section{XXX(, Ph.D., MS@XXX)}
    \subsection*{个人背景}
        \textbf{GPA:}overall xx/xx, major xx/xx\\
        \textbf{Ranking:}xx/xx\\
        \textbf{TOEFL(R/L/S/W):}xxx (xx/xx/xx/xx)\\
        \textbf{GRE(V+Q+AW):}xx+xx+\\
        \textbf{推荐信:}\\
        \textbf{科研/实习/交流:}\\ 
        \textbf{联系方式:}

    \subsection*{申请结果}
        \textbf{AD:}\\
        \textbf{Rej:}\\
        \textbf{Accept:}

    \subsection*{申请的前期准备}
        \subsubsection*{选择}
        \subsubsection*{GPA}
        \subsubsection*{科研/实习/比赛/交流}
        \subsubsection*{GRE/TOEFL}

    \subsection*{申请过程}
        \subsubsection*{选校}
        \subsubsection*{文书}
        \subsubsection*{推荐信}
        \subsubsection*{网申}
        \subsubsection*{等待结果}
        \subsubsection*{项目介绍}

    \subsection*{总结}



    \section{XXX(, Ph.D., MS@XXX)}
    \subsection*{个人背景}
        \textbf{GPA:}overall xx/xx, major xx/xx\\
        \textbf{Ranking:}xx/xx\\
        \textbf{TOEFL(R/L/S/W):}xxx (xx/xx/xx/xx)\\
        \textbf{GRE(V+Q+AW):}xx+xx+\\
        \textbf{推荐信:}\\
        \textbf{科研/实习/交流:}\\ 
        \textbf{联系方式:}

    \subsection*{申请结果}
        \textbf{AD:}\\
        \textbf{Rej:}\\
        \textbf{Accept:}

    \subsection*{申请的前期准备}
        \subsubsection*{选择}
        \subsubsection*{GPA}
        \subsubsection*{科研/实习/比赛/交流}
        \subsubsection*{GRE/TOEFL}

    \subsection*{申请过程}
        \subsubsection*{选校}
        \subsubsection*{文书}
        \subsubsection*{推荐信}
        \subsubsection*{网申}
        \subsubsection*{等待结果}
        \subsubsection*{项目介绍}

    \subsection*{总结}

    \section{XXX(, Ph.D., MS@XXX)}
    \subsection*{个人背景}
        \textbf{GPA:}overall xx/xx, major xx/xx\\
        \textbf{Ranking:}xx/xx\\
        \textbf{TOEFL(R/L/S/W):}xxx (xx/xx/xx/xx)\\
        \textbf{GRE(V+Q+AW):}xx+xx+\\
        \textbf{推荐信:}\\
        \textbf{科研/实习/交流:}\\ 
        \textbf{联系方式:}

    \subsection*{申请结果}
        \textbf{AD:}\\
        \textbf{Rej:}\\
        \textbf{Accept:}

    \subsection*{申请的前期准备}
        \subsubsection*{选择}
        \subsubsection*{GPA}
        \subsubsection*{科研/实习/比赛/交流}
        \subsubsection*{GRE/TOEFL}

    \subsection*{申请过程}
        \subsubsection*{选校}
        \subsubsection*{文书}
        \subsubsection*{推荐信}
        \subsubsection*{网申}
        \subsubsection*{等待结果}
        \subsubsection*{项目介绍}

    \subsection*{总结}

\chapter{致谢}
设计参考了计算机系2015年的飞跃手册,在此表达感谢。
\end{document}