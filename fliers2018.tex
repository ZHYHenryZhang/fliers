\documentclass[a4paper,UTF8]{book}
\usepackage{ctex}       % necessary for chinese
\usepackage[margin=1.25in]{geometry}
\usepackage{color}
\usepackage{hyperref}
\usepackage{titlesec}
\titleformat{\subsection}{\Large\bfseries}{\thesubsection}{1em}{}
\titleformat{\subsubsection}{\normalsize\bfseries}{\thesubsubsection}{0em}{}
\usepackage{fancyhdr}

\pagestyle{fancy}
% \lhead{\nouppercase{\rightmark} (\nouppercase{\leftmark})}
% \chead{}
% \rhead{}
% \lfoot{\today}
% \cfoot{}
% \rfoot{\thepage}
% \renewcommand{\headrulewidth}{0.4pt}
% \renewcommand{\footrulewidth}{0.4pt}


\fancyhf{}
\fancyhead[EL,OR]{\subsectionmark}
\fancyfoot[EC,OC]{\thepage}
\fancyhead[ER,OL]{\nouppercase{\leftmark}-\nouppercase{\rightmark}}
\renewcommand{\chaptermark}[1]{\markboth{#1}{}}
\renewcommand{\sectionmark}[1]{\markright{#1}{}}
%\renewcommand{\subsectionmark}[1]{\markright{#1}{}}

%\renewcommand{\chaptermark}[1]{%
%\markboth{#1}{}}
\usepackage{enumitem}
%\usepackage{paralist}
%\usepackage{enumerate}
% \setlength{\leftmargin}{1.2em} %左边界
% \setlength{\parsep}{0ex} %段落间距
% \setlength{\topsep}{0ex} %列表到上下文的垂直距离
% \setlength{\itemsep}{0pt} %条目间距
% \setlength{\labelsep}{0pt} %标号和列表项之间的距离,默认0.5em
% \setlength{\itemindent}{0pt} %标签缩进量
% \setlength{\listparindent}{0pt} %段落缩进量

%\usepackage{layout}
\usepackage{indentfirst}
\setlength{\parindent}{0em}
%\setlength{\evensidemargin}{.25in}
%\setlength{\textwidth}{6in}
%\setlength{\topmargin}{-0.5in}
%\setlength{\topmargin}{-0.5in}

%\setlength\parskip{.3\baselineskip}

\usepackage{graphicx}
\usepackage{setspace}
\usepackage{amssymb}
\usepackage{wasysym}
\begin{document}

\title{南京大学工程管理学院\\自动化类\\2018飞跃手册\\预览版本}
\author{18 Fliers}
\maketitle % necessary for title

\tableofcontents %目录

\chapter{前言}
\begin{spacing}{1.5}
\paragraph[indent=0pt]{}18飞跃手册分为两个部分,留学经历分享和18届申请总结。
\paragraph{}第一部分中非常荣幸能够邀请到已经在国外经历了一段留学时光的学长学姐们和我们分享留学经验。学长学姐们介绍了自己的学校专业的就读体验,科研实习经历和未来的打算。并对留学各地的对比,自动化类的留学状况和他们就读专业的前景发表了自己的看法。看看和我们同样从南大工管出发去往世界各地,带着不同想法,经历过各式各样生活的学长学姐们写下的肺腑之言,也许对于还在犹豫是否要出国,下定决心了还没有方向,或者是在准备中仍有迷茫的学弟学妹们有一些帮助。
\paragraph{}第二部分是18飞友们对刚刚过去的申请季的总结。格式参考了一亩三分地的定位板块并给出了申请结果,为学弟学妹们提供定位参考。申请准备部分是大一到大三期间为了出国需要做的准备,包括课程学期、科研实习交换和英语准备。申请过程部分对选校,文书,推荐信,网申这些申请中重要的部分均有所涉及。
\paragraph{}我们得到了来自小百合传承的飞跃手册的帮助,得到了一亩三分地的分享,得到了学长学姐的指引,也想用这样的方式来传递这种善意。
\end{spacing}

\clearpage

\chapter{留学经历分享}  %document class book required
\clearpage
\section{美国篇}
\subsection{付国峪(CE, Ph.D.@TAMU)}
    \textbf{留学教育背景:}Texas A\&M University (TAMU), Computer Engineering, Ph.D., 2013-2018\\
    \textbf{联系方式:}fgy108@gmail.com

    \subsubsection*{在Texas A\&M University读Computer Engineering是怎样的体验?}
        \begin{enumerate}[itemindent=0pt,itemsep=0pt,parsep=0pt]
            \item TAMU在一个小镇上,与最近的大城市(休斯顿)有两小时的车程。生活有一种与世无争的安静和简单。
            \item 研究生课程难度不小,对编程和数学的要求都不低。
            \item TAMU的计算机系较强的方向是机器人和计算视觉。TAMU的电力电子系较强的方向是模电、集成电路、强电、计算机体系结构。
            \item 个人认为TAMU的科研风气崇尚硬实力,鼓励做出有难度的东西,而不鼓励为发论文而发论文。这样对培养学生的能力有好处,毕业出来的学生在工业界很受欢迎。也因为这样,很多学生论文发得并不多,毕业找学术岗位的时候就没有优势了。
        \end{enumerate}
    \subsubsection*{您在读研期间经历过哪些实习/科研,它们的体验是怎样的?}
        \begin{enumerate}[itemindent=0pt,itemsep=0pt,parsep=0pt]
            \item 博士期间我是在计算机系做GPS定位方向的。这样的设定对我的软硬件能力要求非常高。软件方面,我得学习计算机的体系、算法,机器学习和优化,分布式系统,实时系统,甚至很多3D几何建模和graphics的知识。硬件方面,我得学习更高级的信号处理,控制论,电路设计,甚至天体物理的知识。
            \item 我的例子没有普遍性。不少的同学都是只做硬件或者软件,只需要专攻一个方向。但总的来说,美国的博士都是不好读的,要做好打硬仗的准备。
            \item 我在谷歌的GPS团队实习。项目与提高安卓手机的GPS精度有关。实习可以提高编程能力,更可以让你接触到最前沿的业界动态。但读博期间不建议过多实习,除非实习项目有利于博士的毕业。博士生应该多花时间在科研上,多学多想,把自己打造成这个领域的先锋。
            \item 当然如果你的志向不在学术上,或者毕业后也不打算继续做这个方向,那就应该早早毕业,或者多实习,锻炼工作能力。
        \end{enumerate}
    \subsubsection*{您现在回顾当初选择留学、选择专业的初衷,在经历了留学生活后有什么新的感受?}
        \begin{enumerate}[itemindent=0pt,itemsep=0pt,parsep=0pt]
            \item 当年留学的初衷是为了提高自己的能力,想学一些别人无法轻易追赶上的本领。在美国读博之后,我确实获得了很大的提升。科研上很多时候都是在做前无古人的事情,完全没有参考资料。这就锻炼了思维、胆量和能力。大家知道,很多研究都是however项目 – 改良一下前人的做法即可。据我所知,美国一流大学的研究很多都是“from scratch”- 推翻前人假设,重新构建一个新的理论或系统。这样操练几年下来,每个优秀的博士毕业生都有了独当一面的功夫。
            \item 当然我说的都是用功、爱学习的同学。水水毕业的博士也不在少数。。。
        \end{enumerate}
    \subsubsection*{您选择了在美国攻读博士学位,它和其它选项相比优劣有哪些?就以博士为最终学位的学术深造而言您在出国深造国家的选择上对学弟学妹有什么建议?}
        \begin{enumerate}[itemindent=0pt,itemsep=0pt,parsep=0pt]
            \item 读书都是为了找工作。如果你想以后当教授,建议读个好学校的博士,学校排名越高越好。先读个硕士,然后再申请更好的学校读博也是值得的,毕竟本科毕业直接拿一流大学的offer难度不小。不用介意到了30岁才毕业,很多教授都是30多岁才博士毕业的。如果你在排名50的学校博士毕业,那么你很有可能就在排名50及以后的学校当教授。你未来的学术成就可能就不如在排名30的学校来得高。
            \item 如果你以后想工作,建议越早毕业越好。但是读个硕士是很有帮助的。特别是要想在美国找工作的话,读个好学校的硕士是必须的。读博士就因人而异,你也可以利用博士期间规划人生。出来找工作时,计算机博士会比计算机硕士更占优势,工资也更高。
        \end{enumerate}
    \subsubsection*{对于以博士学位为最终学位的深造而言,先在国内读master作为跳板,国外读master作为跳板和本科直接申请Ph.D.三者上有何优劣?您就此对学弟学妹的建议是什么?}
        \begin{enumerate}[itemindent=0pt,itemsep=0pt,parsep=0pt]
            \item 个人认为如果在国内读了master更好是直接找工作。如果还要出来读博,那么读完就快30岁了,你会错过很多机会。等你的同学还是公司中层的时候,或者开始创业的时候,你才刚出来从头做起。
            \item 当然如果你要做教授,为了去个好学校读博,这是完全值得的。
            \item 本科直接申请Ph.D.难度很大,因为硕士毕业生越来越多了,教授们更愿意挑训练有素、会写论文的硕士毕业生,不是吗?
        \end{enumerate}
    \subsubsection*{目前南大自动化类的同学每年的出国比例都要低于南大本科生出国比例平均值不少,您认为造成这一现象的原因是什么?您对南大自动化(类)在读的学弟学妹们在出国读研方向的选择上有什么建议吗(劝进/劝退)?}
        \begin{enumerate}[itemindent=0pt,itemsep=0pt,parsep=0pt]
            \item 数据上不必在意。南大出国比例是被理科拉高的。理科生出国容易、就业难,他们可选的路不多。
            \item 建议不要随大流,认真想想自己想要什么样的生活。留在国内,不继续深造,国内读研,或者国外读研,哪一条路都有秀出班行的例子。南大人都不用担心生存问题。南大人要考虑的,是怎样定义自己的成功,然后选一条路去实现它。
        \end{enumerate}
    \subsubsection*{从现在看留学时光,您会给即将开始留学生涯的大四学弟学妹们什么建议?}
        \begin{enumerate}[itemindent=0pt,itemsep=0pt,parsep=0pt]
            \item 多尝试新鲜事物,多出去走走看看。融入当地的环境。
            \item 相比起来,国内风气比较浮躁。在国外,会慢慢回归自我,思考清楚人生的意义。
        \end{enumerate}
    \subsubsection*{就您的了解而言,您目前所读专业的Ph.D.未来发展前景如何,有哪些方向(学术界和工业界),您个人更期待哪一个方向?对于这些方向,在Ph.D.在读期间该做哪些准备?}
        \begin{enumerate}[itemindent=0pt,itemsep=0pt,parsep=0pt]
            \item 现在AI和机器学习非常流行,工业界对这些岗位的需求也非常大。如果你们现在就从这些专业毕业,那当然会是很吃香。但是5年之后这些行业是不是还热门,AI泡沫会不会破灭,这都很难说。我自己的GPS方向需求不大,而且难度太高,不建议入坑。
            \item 只要是实用的,对数学、编程有提高的,就是好方向。过于理论的,专为发论文而发论文的,就是差方向。很多博士毕业去工业界,都不是做原来的方向。
        \end{enumerate}



\clearpage
\subsection{应宙锋(EE, Ph.D.@UT Austin)}
    \textbf{留学教育背景:}UT Austin, optical computing, 2016\\
    \textbf{联系方式:}yjcyzf@gmail.com

    \subsubsection*{在UT Austin读EE是怎样的体验?}
    其实博士生的生活大部分取决于自己的研究小组,每个小组的科研方向,管理模式和科研压力都不同,甚至可以说天差地别。有些组的学生每天在世界各地环游,有些组的学生每晚都在实验室度过。我本人还比较喜欢我目前的科研状态,每周和老板讨论一次,大部分时间自己做研究自己安排生活,偶尔出去开会作报告。这里有比较大的超净间还有很多仪器设备,所以想法很容易得到实验验证。有很多公司也在这里加工芯片,白天人会比较多,所以我喜欢晚上在这里开展实验。
    
    奥斯汀是个比较宜居的城市,也是个音乐城市,经常有大大小小的音乐节,比较有名的是西南偏南,是个每年一次的盛会。冬天这里不冷,很少下雪,但夏天稍微有点热,幸好室内都有空调。奥斯汀对篮球迷来说也还不错,附近有圣安东尼奥的马刺,有休斯顿的火箭,有达拉斯的小牛。喜欢钓鱼的朋友也可以开车三小时去海边垂钓,有朋友还钓到过小鲨鱼。我也曾和朋友开车十二个小时北上去科罗拉多的丹佛滑雪。    
    
    \subsubsection*{您现在回顾当初选择留学、选择专业的初衷,在经历了留学生活后有什么新的感受?}
    出国的初衷也是想出来开阔眼界,学习点高精尖的技术。现在也努力在这条路上走。来了以后还是挺喜欢这里的生活的,因为比较自由,各方面的自由。
    
    \subsubsection*{您选择了在美国攻读博士学位,它和其它选项相比优劣有哪些?就以博士为最终学位的学术深造而言您在出国深造国家的选择上对学弟学妹有什么建议?}
    美国毕竟还是科技最领先的国家。但就我这个方向(光集成芯片)来说,欧洲也比较强,他们起步更早。要我给建议的话,我觉得不能一概而论,每个方向不一样,特别是读博士,一定要慎重。方向和老板都很重要,做好前期调研工作,多问问同组或者学校的师兄师姐。

    \subsubsection*{对于以博士学位为最终学位的深造而言,先在国内读master作为跳板,国外读master作为跳板和本科直接申请Ph.D.三者上有何优劣?您就此对学弟学妹的建议是什么?}
    至于三种出国的方式我觉得都可以,国外的制度挺灵活,你进来Ph.D.可以转成master,也可以从master转成Ph.D.,也可以换组换导师转专业。关键是自己想清楚要不要读博,读什么方向。选个自己感兴趣的,能坚持的,而且四五年后等你毕业能够给社会带来价值的方向。

    \subsubsection*{目前南大自动化类的同学每年的出国比例都要低于南大本科生出国比例平均值不少,您认为造成这一现象的原因是什么?您对南大自动化(类)在读的学弟学妹们在出国读研方向的选择上有什么建议吗?}
    可能跟风气有关吧,而且国内现在发展也挺好的,出国的越来越少也正常。个人觉得出国看看挺好的,能长长见识,丰富下生活经历。
    
    \subsubsection*{从现在看留学时光,您会给即将开始留学生涯的大四学弟学妹们什么建议?}
    享受生活吧,不要被社会给你的条条框框束缚住,别在乎别人的闲言闲语,趁年轻,该干什么干什么。每个人都有自己的pace。有些人的人生是标准模式,什么时间该干什么,但你的人生应该由你自己做主,因为你是高级玩家,你走自定义模式。
    
    \subsubsection*{就您的了解而言,您目前所读专业的Ph.D.未来发展前景如何,有哪些方向(学术界和工业界),您个人更期待哪一个方向?对于这些方向,在Ph.D.在读期间该做哪些准备?}
    我这个方向应该是比较重要的方向,欧洲在十几年前起步,美国正在加足马力追赶。有个业内人士说,现在的光芯片就是八十年代的电芯片一样正处于奇点。所以比较期待工业界的发展。


\clearpage
\subsection{岳翔宇(EE, MS@Stanford, EECS, Ph.D.@UCB)}
    \textbf{留学教育背景:}Stanford EE MS, 2014,UC Berkeley EECS Ph.D. 2016\\
    \textbf{联系方式:}微信:yuexiangyu618,邮箱: xyyue@eecs.berkeley.edu

    \subsubsection*{在UCB读EECS是怎样的体验?}
        \begin{enumerate}[itemindent=0pt,itemsep=0pt,parsep=0pt]
            \item 课程:\\
            硕士:每个quarter三门课,课业量蛮大的,跟本科感觉不是一个级别的;有的时候还会通宵赶作业什么的;GPA对于找工作或者申请Ph.D.还是有一定帮助的,尤其是找工作。\\
            博士:每学期课可以选的少一些,而且压力也没有那么大了;博士期间的课程要求都挺容易达到的,而且成绩也没有那么重要了。
            \item 科研:\\
                硕士:硕士科研的话还是要主动找教授,而且可能教授不太会一开始就给funding(当然这个不同学校,不同教授都不太一样);如果不给的话那就可能要先免费做一个学期证明一下自己,后面再拿funding。\\
                博士:博士科研的话;感觉还是有一定压力的,一方面来自老师以及合作者对于项目的要求,另一方面来自自己吧, 给自己找一个合适的,比较promising的,能做出成果的方向。
            \item 生活:\\
                硕士:个人感觉如果毕业想找工作的话生活会轻松一些,尽管找工作也会有压力;但我觉得如果打算硕士毕业申请博士的话可能压力稍微更大一点;毕竟换工作容易,换Ph.D.学校,换Ph.D.导师,或者换Ph.D.的方向那就难多了。\\
                博士:生活的丰富程度还是可以的,虽然经常会因为project周末到学校加班。感觉时间支配还算自由。
        \end{enumerate}
    \subsubsection*{您在读研期间经历过哪些实习/科研,它们的体验是怎样的?}
        \begin{enumerate}[itemindent=0pt,itemsep=0pt,parsep=0pt]
            \item 实习:\\
            硕士:SAP实习,公司氛围环境还算不错吧。当时打算申请Ph.D.,没太多实习;这个实习也是当时上课的一个lecturer给推荐的。如果找工作的话实习还是挺关键的,要花不少时间去刷题,面试等等。\\
            博士:腾讯 Seattle。做的项目跟自己在学校做的非常match。公司中国人偏多,加班人感觉比SAP要更多一些,别的方面的话差别倒是没那么明显。
            \item 科研:\\
            硕士:第一年想找老师做computer architecture 方向的research,当时由于本科时期做的项目偏少,好多老师都不太给科研的机会,不拿funding做了两个学期,也没太多成果。第二年找到一个并行计算方面的老师,先免费做了一个学期,之后一个quarter给了RA;后来发现不是很感兴趣,再之后一个quarter又找了另外一个CS的教授做RA。吐槽一下,当时找教授做科研很曲折不顺利的时候还是有点痛苦的。不过不同学校不同专业可能都不一样吧,我也只是把经历给大家坐下参考。别的学长学姐有的可能会顺利不少。\\
            博士:博士科研还是蛮看老师的。老师的方向,老师的期望,老师的network。当然还是主要看自己,尽快找到一个感兴趣的,有动力做下去的方向。
        \end{enumerate}
    \subsubsection*{您现在回顾当初选择留学、选择专业的初衷,在经历了留学生活后有什么新的感受?}
    没有后悔出来留学吧应该,毕竟感受到了不同的生活方式,校园文化。不过国内还是好玩多了,不管是娱乐还是饮食。LOL
    \subsubsection*{您选择了在美国攻读博士学位,它和其它选项相比优劣有哪些?就以博士为最终学位的学术深造而言您在出国深造国家的选择上对学弟学妹有什么建议?}
    不太好回答,美国的优势的话,其实我个人感觉主要是可能优秀的学校,科研人员比其他国家整体上更多一些吧。科研氛围应当也算是领先的,当然我也没去过其他国家的学校,不太好评论。
    我觉得还是学校,导师,方向最重要吧;国家没那么重要。当然在申请到的学校,导师,方向差不多的情况下,我还是推荐来美国的。如果申请时候就打算申请一个国家的话,如果托福,GRE都还可以的话,我建议美国吧。
    \subsubsection*{对于以博士学位为最终学位的深造而言,a先在国内读master作为跳板,b国外读master作为跳板和c本科直接申请Ph.D.三者上有何优劣?您就此对学弟学妹的建议是什么?}
        我在上面标了a,b,c。
        \begin{enumerate}[itemindent=0pt,itemsep=0pt,parsep=0pt]
        \item a:\\
        优势:能够有机会硕士期间发表更多文章,有利于博士申请结果更好;跟国内的老师的关系搞得不错的话对于将来回国也是很好的。\\
        劣势:可能在国内多读了2,3年,因为国内读完硕士出国读博士通常还是需要5年左右的。
        \item b:\\
        优势:能够提前了解国外的学习生活习惯,提前跟国外的老师做research,如果做得好的话,老师可能就直接接受你转Ph.D.了。也有时间explore不同的方向。\\
        劣势:国外硕士一般要自费,当然出去之后再拿TA,RA机会也是不小的,当然不同学校,院系不一样。
        \item c:\\
        优势:省时间,省钱。\\
        劣势:成果少一些,申请结果相对差一些,而且explore,选择research方向的自由度小。
        \end{enumerate}
    \subsubsection*{目前南大自动化类的同学每年的出国比例都要低于南大本科生出国比例平均值不少,您认为造成这一现象的原因是什么?您对南大自动化(类)在读的学弟学妹们在出国读研方向的选择上有什么建议吗?}
    我觉得原因是出国的氛围还是不是很强烈吧,学生也比他们要少一些。我觉得不要管别的院吧,自己做好了还是一切皆有可能的(我们那年除了我,丁家琛学长也申请到了斯坦福)。我还是觉得出国是一个不错的选择的,劝进啊。做好决定之后就努力去拼。中间遇到困难不要灰心,我当时托福就考了6次。。
    \subsubsection*{从现在看留学时光,您会给即将开始留学生涯的大四学弟学妹们什么建议?}
    淡定,放松。大四很多跟同学的时光都是我至今很怀念的。课余时间可以提前补充 一些觉得对自己将来国外生活有帮助的知识。
    \subsubsection*{就您的了解而言,您目前所读专业的Ph.D.未来发展前景如何,有哪些方向(学术界和工业界),您个人更期待哪一个方向?}
    每个Ph.D.其实做的方向真的很窄;我现在做simulation在自动驾驶中的应用相关的项目,学术界和工业界感觉都还可以。


\clearpage
\subsection{羌哲诚(IE, Ph.D.@UCF)}
    \textbf{留学教育背景:}University of Central Florida,工业工程,2017Spring\\
    \textbf{联系方式:}微信名:云君

    \subsubsection*{在University of Central Florida读工业工程是怎样的体验?}
    在ucf课程还比较简单,主要时间用来做科研,生活的话在奥兰多还是很舒服的,气候好,玩的地方比较多,吃的也不错。

    \subsubsection*{您在读研期间经历过哪些实习/科研,它们的体验是怎样的?}
    我以前是信息工程专业,来美国读了商学院的研究生,然后工作了一小段时间,觉得工作不是我所喜欢的,希望能做更有挑战性的工作,就决定读博,掌握更多的知识。博士又选了工业工程优化方向 。科研确实有一点压力,毕竟要学很多新东西去赶上别人,不过还是很有乐趣和成就感的。不断的挑战自己,做自己以前觉得不可能做的事情。

    \subsubsection*{您现在回顾当初选择留学、选择专业的初衷,在经历了留学生活后有什么新的感受?}
    作为留学生的话,选择计算机相关的专业会更好找工作。如果仅是希望读一个硕士在美国留下来工作的读计算机相关专业会比较有胜算。

    \subsubsection*{您选择了在美国攻读博士学位,它和其它选项相比优劣有哪些?就以博士为最终学位的学术深造而言您在出国深造国家的选择上对学弟学妹有什么建议?}
    在美国读博士还是非常好的,你有很大的空间自己去发展,也有比较好的资源和收入。你有更多的时间投入科研而不是被琐碎的事所干扰。当然你也会遇到各个国家的人,互相学习互相竞争。 

    \subsubsection*{对于以博士学位为最终学位的深造而言,先在国内读master作为跳板,国外读master作为跳板和本科直接申请Ph.D.三者上有何优劣?您就此对学弟学妹的建议是什么?}
    我们学校工业工程专业是只收读过硕士的人做博士的。但大多数学校和项目是都收的。

    \subsubsection*{目前南大自动化类的同学每年的出国比例都要低于南大本科生出国比例平均值不少,您认为造成这一现象的原因是什么?您对南大自动化(类)在读的学弟学妹们在出国读研方向的选择上有什么建议吗?}
    原因我猜是在国内读研出来薪水也不错吧。海龟研究生回国找工作没有太多优势(仅是我的猜测)。喜欢出来闯荡喜欢学术的推荐出国读研。

    \subsubsection*{从现在看留学时光,您会给即将开始留学生涯的大四学弟学妹们什么建议?}
    多与他人交流,更好的融入美国。学术能力要强,沟通能力也要强。\\
    学会开车。美国基本都要会开车。


        


\clearpage
\subsection{言浩(CS-ML, Ph.D.@WUSTL)}
    \textbf{留学教育背景:}Washington University in St. Louis, Ph.D. in Machine Learning, 2014入学\\
    \textbf{联系方式:}yanhao8510@gmail.com

    \subsubsection*{在WashU读CS是怎样的体验?}
    都还挺好的。\\
    课程上因为我是机器学习方向,所以我只能就机器学习方向的课程做一下介绍。首先就是我们系相关机器学习的课程不算多,但是覆盖得还比较全面,难度也适中。然后我们系的ML相关课程的老师教的都很好,也都很负责。\\
    科研上的话,每个组的情况都不太一样,所以我也不太好说。但是有一点我想强调的是,如果是申请Ph.D.的话,强烈建议拿到offer之后来美国campus visit一下,跟自己心仪的faculty还有他们组的学生当面聊一下。
    \subsubsection*{您在读研期间经历过哪些实习/科研,它们的体验是怎样的?}
    我做的机器学习,在学校主要做的项目是运用机器学习的方法来量化的解决一些社会科学类的问题。因为是交叉学科,所以主要的体验就是要不停的学习各种除了机器学习以外的其他方面的知识。\\
    除此之外,我还在北京微软亚洲研究院(MSRA)作为research intern实习过。现在正在西雅图的Facebook作为engineering intern实习。实习的最大体验就是,要用于尝试自己之前没有尝试过的方向跟问题,跳出自己的舒适区,这样做,从理论的角度来说,可以增加更多可以research的方向,从实践的角度说,可以丰富自己的简历,在以后的发展上多了其他的一些可能性。

    \subsubsection*{您现在回顾当初选择留学、选择专业的初衷,在经历了留学生活后有什么新的感受?}
    不好说什么新的感受。不过跟其他周围所有Ph.D.交流下来,大家一致都同意的是,Ph.D.最基本的要求就是,需要发自内心的对所研究的问题跟方向有极大的兴趣,不能跟风看什么火就学什么。

    \subsubsection*{您选择了在美国攻读博士学位,它和其它选项相比优劣有哪些?就以博士为最终学位的学术深造而言您在出国深造国家的选择上对学弟学妹有什么建议?}
    我读的计算机的Ph.D.,在美国的话有个最大的优势就是计算机学科的产业很发达,因此产学研结合十分紧密。暂时没发现明显劣势。\\
    我觉得就计算机学科的Ph.D.而言的话,深造的国家无所谓,关键还是导师跟研究方向/项目。

    \subsubsection*{对于以博士学位为最终学位的深造而言,先在国内读master作为跳板,国外读master作为跳板和本科直接申请Ph.D.三者上有何优劣?您就此对学弟学妹的建议是什么?}
    如果有条件的话,可以在国外读个master。因为一方面可以给自己一个缓冲准备的时间,另外一方面在国外master期间可以进入实验室做research,积累研究经历,进一步明确自己的研究方向,这对于Ph.D.申请也是有很大帮助的。

    \subsubsection*{目前南大自动化类的同学每年的出国比例都要低于南大本科生出国比例平均值不少,您认为造成这一现象的原因是什么?您对南大自动化(类)在读的学弟学妹们在出国读研方向的选择上有什么建议吗?}
    我个人建议是,在条件允许的情况下,应该勇于尝试跟探索。包括不限于:\\1. 积极参与南大的本科交换项目;\\2. 积极探索个人兴趣并开展科研活动。

    \subsubsection*{从现在看留学时光,您会给即将开始留学生涯的大四学弟学妹们什么建议?}
    如果是给大四的学弟学妹们的话,那就是:学一下做饭跟驾驶技术。\\
    如果是给其他年级的话,那就是:早做准备。

    \subsubsection*{就您的了解而言,您目前所读专业的Ph.D.未来发展前景如何,有哪些方向(学术界和工业界),您个人更期待哪一个方向?对于这些方向,在Ph.D.在读期间该做哪些准备?}
    我目前从事的机器学习方向发展前景很好,学术界跟工业界都有很多opening。我个人暂时倾向于工业界。由于学科特点,无论是工业界还是学术界,我们这个方向还是要搞好本职工作,做好自己的科研。
                    
    \subsubsection{想说的话:}
    其实在国外留学的经历每个人有每个人的不同的经历跟想法,我在这里只能说一下我自己一个大概的感受。欢迎学弟学妹给我发email,我会根据我自己跟周围人的经历,根据你们的问题给出我的建议。



\clearpage
\subsection{魏天舒(ECE, Ph.D.@UCR)}
    \textbf{留学教育背景:}University of California, Riverside,ECE,2013\\
    \textbf{联系方式:}njuwts@163.com

    \subsubsection*{在UCR读ECE专业是怎样的体验?}
    UCR是加州大学系统中的一所,当初来到这里主要是因为申请到了合适的Ph.D.导师。这里是quarter制,一年三个quarter,包含fall,winter和spring,暑假从6月底到九月底。读Ph.D.的话,主要是第一年上专业课并和导师讨论确定一个Ph.D.的研究方向,第一年结束的时候需要通过一个qualify exam (包含专业课程的exam和一个关于自己Ph.D.研究方向的presentation)。第二年开始跟着导师做research,期间经过导师同意也可以继续选一些课程。正常读Ph.D.的时间基本是5年毕业,身边同学也会有4年或6年毕业的情况(取决于自己的进度和导师的要求)。生活方面读研通常都是在校外附近租房子,美国有些学校附近会有住在房东家里的情况,UCR附近主要都是整栋出租的房子,一般和同学一起合租。美国大部分地区一般需要有辆车才能方便的生活。对于考驾照,在加州通过笔试之后,如果有国内驾照是可以直接拿到美国驾照的;如果没有国内驾照的话,需要再通过一个路考。在美国考驾照比国内简单得多 (不同的州有不同的规定),在加州路考主要是载着考官在城市里按照考官制定的路线开一圈达到各项安全要求即可。

    \subsubsection*{您在读研期间经历过哪些实习/科研,它们的体验是怎样的?}
    读Ph.D.的过程基本是做research,发paper (包含conference和journal),毕业的要求主要是导师定的,每个导师都不同,有些导师会要求发conference paper和journal paper的数量,有些会要求完成project的数量。毕业的要求是可以和导师negotiate的,可以在读Ph.D.期间不断和导师讨论自己的进度和剩余的工作量,把握好自己的进度以免到最后对毕业的时间产生分歧。对于做research的风格和节奏主要取决于导师,我的导师对组里的学生帮助算是比较多,每个project都会和我们讨论,会帮助我们筛选要投的conference或journal,也会帮学生改写好的paper。有些导师对学生管理的相对少一些,可能需要学生自己去找research的方向和project,自己决定投那些会议期刊,导师主要给学生稍作指导。不同的风格各有优点,可以在选定导师之前和导师多聊一些。\\
    暑假一般可以用来实习,实习需要经过导师允许。有些导师对学生的暑期比较宽松,可以自己决定去实习还是留在实验室做research。有些导师可能只希望学生第四年的暑期去实习。除了自己找的实习,有时候导师手里也会有一些资源,可以直接去national lab或者公司里实习。如果自己有实习的计划,尽早和导师讨论来确定合适的时间。我在Ph.D.期间有过两次实习,一次是在United Technologies Research Center,第二次是在Google。第一次的实习是偏research岗,是对方通过我之前的一篇paper直接联系的我,他们觉得我的背景和他们的需求比较相似。第二次Google的实习是我自己找的,期间需要通过2-3次的interview。和在学校实验室相比,在公司里做实习工作时间比较规律,只有工作时间需要work,其他时间不会有人来和你讨论工作的事,在学校实验室里有时候不怎么区分weekday和weekend,导师随时找需要随时回应。还有在公司里实习基本是team合作来共同完成任务,在学校实验室里主要靠自己去解决project里遇到的各种问题。

    \subsubsection*{您现在回顾当初选择留学、选择专业的初衷,在经历了留学生活后有什么新的感受?}
    如果选择读Ph.D.,需要想好自己以后希望做什么。 如果今后想从事科研 (做professor或去研究所做research),Ph.D.学位是很必要的。如果以后想在公司里工作,在美国Ph.D.学位并不是对所有工作岗位都有帮助,要取决于专业。因为不是所有的Ph.D.研究方向都在工业界有应用,大部分Ph.D.的研究内容都超出了普通公司的需求,如果想去公司工作,需要选择在工业界有应用的研究方向才会找工作有帮助。

    在Ph.D.选导师的时候需要自己多了解一些professor的信息,这些很可能影响到今后自己能否顺利毕业以及实验室的工作内容和风格是否适合自己。一般professor都会有自己的主页以及所管理的实验室的主页,在上面可以看到professor研究的方向,发过的文章,可以决定这个professor做的东西自己是否感兴趣,如果读Ph.D.做自己不感兴趣的东西还是比较痛苦的。在网站上可以找到professor以前毕业的Ph.D.学生,可以了解到通常这个组里Ph.D.毕业一般需要几年,以及毕业所需要的工作量。还可以了解一下这个组里投的paper的会议期刊种类,从而可以了解一些professor对research质量的要求。

    \subsubsection*{对于以博士学位为最终学位的深造而言,先在国内读master作为跳板,国外读master作为跳板和本科直接申请Ph.D.三者上有何优劣?您就此对学弟学妹的建议是什么?}
    在申请Ph.D.之前有master学位可能会对申请Ph.D.起到一些帮助,因为申请Ph.D.,professor还是比较看重学生之前是否与相关的研究经历,我的身边也有些professor偏向招有master学位的学生。但是读master再读Ph.D.,也会增加时间成本,如果能在本科期间能参与到一些research project中,或者能知名的conference或journal上发一些paper可以很大程度上帮助Ph.D.的申请。

    \subsubsection*{目前南大自动化类的同学每年的出国比例都要低于南大本科生出国比例平均值不少,您认为造成这一现象的原因是什么?您对南大自动化(类)在读的学弟学妹们在出国读研方向的选择上有什么建议吗?}

    \subsubsection*{就您的了解而言,您目前所读专业的Ph.D.未来发展前景如何,有哪些方向(学术界和工业界),您个人更期待哪一个方向?对于这些方向,在Ph.D.在读期间该做哪些准备?}
    我的专业属于ECE,Ph.D.的方向是energy-efficient cyber physical system,主要做一些优化算法来优化buildings,power grid的能耗。我个人比较偏向去公司里工作,现在在Google工作。在Ph.D.期间可以多关注一些工业界比较流行的technology,我也会尽量在自己的research project里用一些machine learning/deep learning的方法,一方面能帮助自己更好的理解这些方法,另一方面也能增加自己在这些方面的经验。另外如果以后想去公司里工作,可以在自己research相关的范围内,多选一些和工业界联系紧密的课程像machine learning,database,computer architecture之类。在导师允许的情况下,尽量利用暑假多去公司里做一些实习,在找全职的时候可以帮助自己拿到更多的面试机会。如果在一些公司里实习过,在找全职工作的时候,也可以减少几轮面试,增加自己拿到offer的机会。


\clearpage
\subsection{石媛媛(EE, Ph.D.@University of Washington)}
    \textbf{留学教育背景:}University of Washington(华盛顿大学), EE Ph.D., 2015\\
    \textbf{联系方式:}yyshi@uw.edu\\
    \textbf{个人主页:}\url{https://sites.google.com/site/yuanyuanshinju/}
    \subsubsection*{在华盛顿大学读EE是怎样的体验?}
    华盛顿大学(University of Washington)位于西雅图,校园很美,而且西雅图是个特别宜居的城市,有大城市生活的方便,也很有文艺气息。喜欢户外运动的话,西雅图冬天可以滑雪,夏天hiking,sailing,抓螃蟹等等,自然美景很多,所以还是很推荐的!\\
    我在UW的EE专业读博,我们系选课还是蛮自由的,全校其他专业的课程都可以选,比较热门的课程像CS,statistics可能要申请才能选中。EE 硕士项目的话,只要完成了课程的学分要求,也是可以自由选课的。很推荐UW CS项目给大家,UW的王牌专业,各种career fair也超多。EE的话在全美20名左右,也还不错,每年录取博士+硕士一共70人左右。我们同级中有10名左右来自中国的master同学,基本都在美国Tech company找到了心仪的工作。

    \subsubsection*{您在读研期间经历过哪些实习/科研,它们的体验是怎样的?}
    我现在是博士三年级,开学博四。有过两份实习,第一份实习是导师推荐的,在西雅图的一家clean energy公司(Doosan Gridtech,这家公司原来是一家Seattle startup, 2016被韩国Doosan集团收购)。这是我人生的第一份实习,感觉收获很多,遇到了一位非常好的mentor --Tess。她是哈佛大学的博士,毕业后在国家实验室工作一段时间,然后加入Doosan。我们记得我上班的第一天,她就跟我聊到希望我做这份实习,不止能做好自己的project,也希望我多去了解公司的业务逻辑,以及了解他人的工作。她会建议我每周邀请1-2位其他部门的同事(并且尽量是manager级别及以上的同事)一起吃午餐,甚至帮我写好了invitation email范本,超感动!听了很多人的career path以及人生故事,感觉对自己之后的职业规划很有收获。公司的work life balance也很好,还有unlimited vacation$\sim$\\
    第二份实习是今年在湾区的京东Research Center,做优化和machine learning相关的项目。因为现在刚开始实习,所以还没有很全面的感受。但是从公司氛围来说,感觉中国公司要比国外公司工作时间长一些。京东在湾区的R\&D center是近几年才成立的,所以没有大公司那种很强的束缚感,工作灵活度还蛮大的,可以做自己感兴趣的方向研究,也有国内工程团队的支持。工作还是挺有意思的。

    \subsubsection*{您选择了在美国攻读博士学位,它和其它选项相比优劣有哪些?就以博士为最终学位的学术深造而言您在出国深造国家的选择上对学弟学妹有什么建议?}
    我选择了在美国读博。之前本科也去加拿大交换过,相比下更喜欢美国,感觉社会更年轻更有活力,机会也更多一些。

    \subsubsection*{对于以博士学位为最终学位的深造而言,先在国内读master作为跳板,国外读master作为跳板和本科直接申请Ph.D.三者上有何优劣?您就此对学弟学妹的建议是什么?}
    如果想以博士学位为最终学位深造的话,更推荐国外读master作为跳板或者本科直接申请博士。在国内读硕士时间成本比较高(3年),感觉总体来说周期有些长了。如果本科阶段有一些科研经历,GPA和英语考试成绩也不错,可以争取直接申请博士。申请博士的同时也申请一些硕士项目,如果能去比较好的学校读硕士,后面转成博士也是很不错的选择$\sim$

    \subsubsection*{目前南大自动化类的同学每年的出国比例都要低于南大本科生出国比例平均值不少,您认为造成这一现象的原因是什么?您对南大自动化(类)在读的学弟学妹们在出国读研方向的选择上有什么建议吗?}
    第一是感觉我们系出国的氛围可能不是很浓,二是自动化专业国外没有直接对口的专业,在申请的时候一般都要申请EE, CS, robotics这些相关专业。如果大家有出国打算的话,建议尽早开始准备。GPA从入学就要注意,英语考试越早越好,尽量大二下-大三搞定。有出国意愿的同学也可以组一个出国小分队,彼此分享一些有益信息,以及一起备战各种英语考试以及申请$\sim$感觉有peer support在准备出国的过程中很重要!

    \subsubsection*{从现在看留学时光,您会给即将开始留学生涯的大四学弟学妹们什么建议?}
    有时间的话可以在出国前好好练下英语口语,以及了解一些美国人文,历史,政法方面的知识。因为从我自身而言,希望通过这段留学经历,不止学到专业知识,也想更充分地感受美国文化。

    \subsubsection*{就您的了解而言,您目前所读专业的Ph.D.未来发展前景如何,有哪些方向(学术界和工业界),您个人更期待哪一个方向?对于这些方向,在Ph.D.在读期间该做哪些准备?}
    感觉这几年美国确实是CS相关的专业找工作更容易一些,而且software engineer收入也很高,但是接下来几年的情况也不好判断。\\
    我自己是做数学优化方向的,有做过优化应用在能源调度,供应链管理以及系统控制上的项目。因为自己从本科的时候就很喜欢运筹学,所以对现在的研究方向很感兴趣。优化方向的话,如果留在学术界可能方向比较交叉(EE,CS,工业工程学院都有做优化的老师);去业界的话,可能主要还是IT 公司吧。


\clearpage
\subsection{周罗伟(Robotics, Ph.D.@UMich)}
    \textbf{留学教育背景:}UMich, Robotics Ph.D., 2015\\
    \textbf{联系方式:}zhouluoweiwest@gmail.com\\
    \textbf{个人主页:}url{https://luoweizhou.net}

    \subsubsection*{在University of Michigan读Robotics是怎样的体验?}
    我所选的课程基本是和机器视觉(CV)和机器学习(ML)相关的。机器人系(RI)第一年课程压力比较大,尤其是必修的实验课ROB550,科研时间很少。建议前两学期把专业知识学扎实,第一年暑假开始多做科研。CV和ML的课程在EECS系下面,RI的选课优先级比EECS稍低一点,但选上课还是没啥压力。安娜堡是大学城,留学生多,饮食方面各国菜系都有。娱乐方面和国内没法比哈,毕竟中部大农村。

    \subsubsection*{您在读研期间经历过哪些实习/科研,它们的体验是怎样的?}
    我第一年和Satinder做Deep RL在控制方面的应用,但由于课程原因投入时间有限,收效甚微。第一年暑假从了现在的老板Jason,做机器视觉和视频分析方面的研究。整体而言还是比较顺利的。科研的感触有几点。\\
    第一,表达能力太重要了(这一点在企业里更明显),高效的沟通可以促生很多新想法,让其他人对你的工作感兴趣,想和你合作。\\
    第二,选好科研方向很重要,不能太冷门(nobody cares about your work),也不能太热门(too competitive)。找好有潜力的方向,先人一步。我博二和博三分别在Salesforce Research和Facebook AI Research做深度学习相关实习。业界资源多得没话说,而且可以和很多大牛小牛讨论,非常值得推荐。

    \subsubsection*{您现在回顾当初选择留学、选择专业的初衷,在经历了留学生活后有什么新的感受?}
    我出国的考虑并不多,美国在CS处于领先地位,并没有太多纠结。但就目前而言,北美CV和ML的申请压力逐渐加大,建议申请时多咨询专业人士…一亩三分地里资源很多。科研专业排名和导师套磁认准\url{http://csrankings.org}

    \subsubsection*{您选择了在美国攻读博士学位,它和其它选项相比优劣有哪些?就以博士为最终学位的学术深造而言您在出国深造国家的选择上对学弟学妹有什么建议?}
    我觉得做CV和ML方面科研美国是首选。如果想找好的教职工作,美国的Ph.D.学位很有优势,尤其是在北美找教职(美国人也特别看重出身哈哈)。

    \subsubsection*{对于以博士学位为最终学位的深造而言,先在国内读master作为跳板,国外读master作为跳板和本科直接申请Ph.D.三者上有何优劣?您就此对学弟学妹的建议是什么?}
    个人觉得国内master申请比本科生并不会有太大优势。我知道UMich很多老师大多招本科生。本校的undergrad或master进实验室太简单了,转Ph.D.的概率很大。我们实验室今年有三个master和一个research engineer转Ph.D.\dots

    \subsubsection*{目前南大自动化类的同学每年的出国比例都要低于南大本科生出国比例平均值不少,您认为造成这一现象的原因是什么?您对南大自动化(类)在读的学弟学妹们在出国读研方向的选择上有什么建议吗?}
    自动化毕竟工科好就业哈。如果有志长期做科研,有个Ph.D.学位是必须的。最稳妥的方法是先出国读master,进实验室做科研,摸索自身是否适合做科研(find your niche)。进可读博,退可进军硅谷。我最初进UMich是master,第一年暑假转为Ph.D.。UMich的RI对转博非常友好,只要导师同意并且有funding就可以转,不需要额外申请。

    \subsubsection*{从现在看留学时光,您会给即将开始留学生涯的大四学弟学妹们什么建议?}
    多做科研,多交朋友,多运动,哈哈。

    \subsubsection*{就您的了解而言,您目前所读专业的Ph.D.未来发展前景如何,有哪些方向(学术界和工业界),您个人更期待哪一个方向?对于这些方向,在Ph.D.在读期间该做哪些准备?}
    我目前倾向工业界,但毕业应该也会申请教职。没啥技巧,Ph.D.期间多发paper呗,尤其是CV和ML这种热门方向竞争压力太大了\dots 工作遍地都是,找好工作必须要有实力。
    
    \subsubsection{想说的话:}
    Go blue!


\clearpage
\subsection{倪慧佳(Marketing, MS@Columbia University)}
    \textbf{留学教育背景:}Columbia University, MS in Marketing, 2015\\
    \textbf{联系方式:}WeChat:giraffe1122334

    \subsubsection*{在Columbia University读Marketing是怎样的体验?}
    哥大的市场营销硕士是哥大商学院唯二的硕士项目之一,而且program比较小,全球仅招十几人。好处是除了本身专业的必修课之外,商院内MBA和Ph.D.的课可以随便选择,所以读一个硕士,可以同时感受三种生活。毕业不是光修完课程就可以,还需要写一篇高质量的毕业论文,整体压力很大。
    \subsubsection*{您在读研期间经历过哪些实习/科研,它们的体验是怎样的?}
    在做毕设的同时也做了暑期实习,在一家纽约的广告公司工作,agency比较小,没有独立的楼层,租的是WeWork的工位,会碰到带狗来上班的同事,氛围很好。

    毕设也很有趣,研究的是百老汇剧预售票的影响因素,以及各因素的影响能力。导师会帮助联系各个百老汇剧的制作公司,逐一签订保密协议后拿到大量一手销售数据进行研究。
    \subsubsection*{您现在回顾当初选择留学、选择专业的初衷,在经历了留学生活后有什么新的感受?}
    觉得自己的选择还是正确的,商科master的学习,区位优势很重要。CBS的slogan就是in the very center of business.  在宇宙中心纽约,各种机会更多。 
    \subsubsection*{您选择了在美国读master,请问您的选择相比于国内和其他国家地区而言有什么优劣呢?}
    优势:纽约作为国际大都市,机会更多,能看到的读到的体验到的东西多
    劣势:纽约的生活成本会比较高
    \subsubsection*{目前南大自动化类的同学每年的出国比例都要低于南大本科生出国比例平均值不少,您认为造成这一现象的原因是什么?您对南大自动化(类)在读的学弟学妹们在出国读研方向的选择上有什么建议吗?}
    自动化类在国内的就业机会就很多,所以很多同学不一定会选择出国读研。有机会有条件的话当然出去看看更好,master也普遍比国内节约一年时间。
    \subsubsection*{从现在看留学时光,您会给即将开始留学生涯的大四学弟学妹们什么建议?}
    充分利用好课余时间,多做一些项目或者实习。不要局限在中国人的小圈子。
    \subsubsection*{您目前打算毕业后直接就业还是继续读博深造?对于您master之后的去向选择而言,在国内读研和在国外读研有何优劣?就您目前打算的方向而言(读博/工作),在master期间需要做一些怎样的准备?}
    Master毕业之后回国工作了。如果在外企工作的话,国外读研的经历帮助更大,在如何与外籍老板同事打交道方面更得心应手。如果毕业就打算工作的话,一定要从入学第一天就开始注重就业问题。国外院校一般都有专门的老师负责Career 方面的咨询和训练,要多和他们沟通,不断完善简历和cover letter,面试前找老师进行模拟面试等。


\clearpage
\subsection{张缙颔(ECE, MS@UCSD)}
    \textbf{留学教育背景:}UCSD, 专业方向Intelligent Systems, Robotics, and Control,2016年入学\\
    \textbf{联系方式:}Jinhan.zhang94@gmail.com

    \subsubsection*{在UCSD读ECE是怎样的体验?}
    SD的天气和环境真的是好的没话讲啊,La Jolla也是全美治安排名前几的社区,人也大都非常nice,可能是四季如春的气候让大家都很happy吧。课程的话,SD的CSE的课程质量都还算比较高,也有蛮多大牛教授开的课,课程比较实用,对于想学CS的筒子们最好多选一些CSE的课啦。至于ECE的课,大多数并不推荐,很多prof的课非常理论,所以如果不是想走research的话,推荐就只选毕业所需的课。
    
    \subsubsection*{您在读研期间经历过哪些实习/科研,它们的体验是怎样的?}
    因为我个人的打算就是找工作,所以没有什么科研的相关经历。我的实习是在SD当地的一家自动驾驶公司,因为是华人偏多的创业公司,所以并没有美国公司著称的Work-life balance。它们东西都推得挺快的,所以学东西也会相对偏快一些。
    
    \subsubsection*{您现在回顾当初选择留学、选择专业的初衷,在经历了留学生活后有什么新的感受?}
    我个人的想法就是人生在世重在体验嘛 = =,所以留学也算是体验不同生活的一部分啦$\sim$美国的教育和中国的还是有非常大的差异的, 整体节奏会比国内快蛮多。读研的时候大多数人都非常辛苦,强度整体还是比国内本科大蛮多,现在工作之后会慢慢回归正常的生活节奏。

    \subsubsection*{目前南大自动化类的同学每年的出国比例都要低于南大本科生出国比例平均值不少,您认为造成这一现象的原因是什么?您对南大自动化(类)在读的学弟学妹们在出国读研方向的选择上有什么建议吗?}
    个人感觉劝退劝进脱离个体情况不太好说,总之看大家对自己未来的期许是怎样的吧。国外留学生活也是有好有坏,每个人性格和习惯的不同也会导致体验上的巨大差异。

    \subsubsection*{从现在看留学时光,您会给即将开始留学生涯的大四学弟学妹们什么建议?}
    提前准备,找工作的话提前刷题做project丰富简历, 科研的话体验联系老板,过来刷GPA,争取早日进实验室。

    \subsubsection*{您目前打算毕业后直接就业还是继续读博深造?对于您master之后的去向选择而言,在国内读研和在国外读研有何优劣?就您目前打算的方向而言(读博/工作),在master期间需要做一些怎样的准备?}
    上班党已经入职啦。。美国读研最直观的优势就是可以在美国找工作吧,至于美国工作的优势大概就是大多数公司的work-life balance比较好,会有更多自己的时间。要说技术水平的话,其实有多领域国内很多公司是比美国的公司厉害的,但是缺点可能就是工作压力会偏大(当然这也不一定是缺点)。对于找工作来讲,准备的话就是提前刷题,同时注意丰富自己的简历吧,其他真的没有太多捷径可走。


\clearpage
\subsection{杨定东(EE, MS@UMich)}
    \textbf{留学教育背景:}University of Michigan,EE,MS to Ph.D.

    \subsubsection*{在Umich读ECE是怎样的体验?}
    课程很辛苦,又要挤出时间去争取实验室的实习,很紧张。但是Umich真的是非常棒的学校,有很棒的工程学院,学校的硬件和软件都是一流的,工程学院(北校区)的环境也非常好,很安静很适合学习。\\
    科研的环境非常好,老师都有自己的资源,像我在Honglak Lee实验室做实习,他手上有100个Titan X的GPU的服务器集群。
    \subsubsection*{您在读研期间经历过哪些实习/科研,它们的体验是怎样的?}
    科研的环境非常好,我在Honglak Lee的实验室做实习跟的是一个博士后,学到了非常多的东西,那个博士后也很nice,很有耐心。有一个很有经验的“老师”总是能少走很多弯路。Honglak Lee本人也是非常厉害的老师,在业界(Computer Vision)也很有名气,他很耐心细致。当然他要求也很高,你申请他实验室的实习也要填一个详细的申请表。
    \subsubsection*{您现在回顾当初选择留学、选择专业的初衷,在经历了留学生活后有什么新的感受?}
    因为大家都说国外的大学和工科教育是首屈一指的,所以就出去了。总想着能变一个环境,体验一下。国外的教育和国内很不同,虽然可能单门课的课业重很多,但是也能学到更多,而且似乎更enjoy。到好的学校留学你一定不会失望吧

    \subsubsection*{您选择了在美国读master,请问您的选择相比于国内和其他国家地区而言有什么优劣呢?}
    一开始就没有考虑其他的国家。美国是留学的第一选择吧,因为最好的资源都在美国,以后能留在美国的机会也多一些。\\
    劣势的话可能学费贵一点?没有详细比较过。

    \subsubsection*{目前南大自动化类的同学每年的出国比例都要低于南大本科生出国比例平均值不少,您认为造成这一现象的原因是什么?您对南大自动化(类)在读的学弟学妹们在出国读研方向的选择上有什么建议吗?}
    我觉得这个因素很多,自动化学的东西本身就应用较多,本科直接选择工作也不差。
    我觉得出国与否完全是自己的选择,没有好坏之分。出国读博/研后,可以深造可以工作。可能如果要申请美国的博士,美国读研应该是很好的跳板。想在美国工作工作也一样。所以出国读书不是目的,可能再往后想一步才是关键。

    \subsubsection*{从现在看留学时光,您会给即将开始留学生涯的大四学弟学妹们什么建议?}
    想读博的暑假就开始联系即将要读书的学校的老师(读研的话),准备开始积累研究经验。越早越好。一开始即便你会的很少也要硬着头皮上,认真请教学习,看论文。一开始都是voluntary的工作,之后可能还可以争取RA(Research Assistant)免去学费,补充日常开销(不过这都不是主要目的,主要还是简历)。
    想工作的话就开始刷LeetCode吧!(我是也是听说,因为我没有这方面的志愿)

    \subsubsection*{您目前打算毕业后直接就业还是继续读博深造?对于您master之后的去向选择而言,在国内读研和在国外读研有何优劣?就您目前打算的方向而言(读博/工作),在master期间需要做一些怎样的准备?}
    读博。
    国外读研应该好一些,因为美国有它的学术关系网,资源也会好一些,视野也会开阔一些,环境和语言也可以事先学习适应。
    其实也没有办法太详细,每个人的情况都会有所不同。读博的话就像6. 所讲在实验室实习越早越好,认真仔细的像前辈学习,虚心谨慎。多读论文多思考。课业的分数反倒不是那么重要,不过学习是学习知识本身,国外的课程真的会学习到很多。祝大家多发发论文!
    \subsubsection{想说的话:}
    感想、心得以及建议其实上面已经说了很多了,主要说说教训吧。
    我本人的教训主要有进实验室还是有点晚,我是研一结束的暑假才进实验室的,其实我研一下就有机会,我主动拒绝了,因为我想课程考一个好分数,其实完全没必要。那个教授的研究机会还是必须研一上半学期考了那个教授的A+那老师才能给的。比较可惜。虽然后面进了一个更厉害教授的实验室,但是也“浪费”了半学期。
    另外最后的建议就是在做研究的时候虽然之前肯定都是打打“零工”,码码代码搬砖,但是也有会有很多算法上的认识,自己也要多主动读点论文。之后自己要尝试着主动想可能的方向,虽然不一定会转变为一作论文,但是是一个很好的开端。
     

\clearpage
\subsection{杨程(CS, MS@CWRU)}
    \textbf{留学教育背景:}凯斯西储大学,EECS, 2013入学\\
    \textbf{联系方式:} saberemember@yahoo.com

    \subsubsection*{在凯斯西储大学读CS是怎样的体验?}
    我刚入校的时候专业是EE,后来过了一学期就转成了CS,转专业的过程比较顺利。CWRU位于克利夫兰,是一个比较小的私立学校。生活上,活动不是很多,平常主要会和舍友还有同专业的人一起出去玩或者玩游戏。课程上, 这里是比较有名的Nerd学校,课程相对比较难。科研上,我研一下学期跟了后来的导师,她是ACM的fellow是学校的大牛,跟着她,我还有另一个南大软件学院读博士的师兄都比较自主,时间上还有课题上主要是按照自己的步伐,毕业也都是很顺利。其他有些老师可能会比较push,但科研风气都很好。
    
    \subsubsection*{您在读研期间经历过哪些实习/科研,它们的体验是怎样的?}
    实习去了一家附近的投资公司,一边上课一边做20小时的part time。做的是用大数据分析股票趋势,以及写代码模拟某些人的想法。学校网站上会有这些附近的公司,或者一些其他地方大公司的实习,而且学校也允许读书期间选一学期或者一年全职到外地公司实习再回来读书。科研上,跟着我的导师做的东西发过会议的poster以及参加美国本地的会议,没有过多深入的。我的那个师兄比较厉害,发了好几篇高质量的论文,后来还没毕业就被微软挖去了。
    
    \subsubsection*{您现在回顾当初选择留学、选择专业的初衷,在经历了留学生活后有什么新的感受?}
    到美国多见识一下各个地方,接触不同的风土人情,对个人成长有很大帮助。我大四毕业的时候算是一个比较爱玩游戏的宅男吧,来美国几年成熟很多。另外,EECS 专业还是比较好找工作的,尤其是CS,而且工资相对都不错。
    
    \subsubsection*{您选择了在美国攻读博士学位,它和其它选项相比优劣有哪些?就以博士为最终学位的学术深造而言您在出国深造国家的选择上对学弟学妹有什么建议?}
    我觉得我喜欢美国的生活吧,除了纽约,其他的地方人不多,可以自由地开车去各个想去的地方,生活也比较悠闲,人们互相之间的交流都很友好真诚。另外如果到了美国除纽约之外的地方,最好早点买车多出去走走,会对适应这边的生活有很大帮助。另外美国的学术就不用说了,水平很高风气也很好。
    
    \subsubsection*{对于以博士学位为最终学位的深造而言,先在国内读master作为跳板,国外读master作为跳板和本科直接申请Ph.D.三者上有何优劣?您就此对学弟学妹的建议是什么?}
    如果要读美国的博士,不要在国内读master。如果经济上没问题可以去想去的master或者博士,如果有点困难,很多学校的博士给的全奖可以cover基本生活需求,研究生也可以试着在学校里找助教的工作。
    
    \subsubsection*{目前南大自动化类的同学每年的出国比例都要低于南大本科生出国比例平均值不少,您认为造成这一现象的原因是什么?您对南大自动化(类)在读的学弟学妹们在出国读研方向的选择上有什么建议吗?}
    我觉得是专业定位的问题?自动化本身就比较杂,本科时候没有一个专攻的方向。如果想要出国的话,可以早点确立一个专业兴趣方向,如强电/弱电/编程/计算机硬件 等,然后多做一些这个方面的研究工作,在申请的时候可以主要申请这个方向。申请个美国的graduate school然后在美国找个工作,我觉得会是很好的选择。
    
    \subsubsection*{从现在看留学时光,您会给即将开始留学生涯的大四学弟学妹们什么建议?}
    多和别人交流,包括同学和老师。多认识些人。在美国不在纽约的话早点买车。钱不多的话可以几千刀买个二手车几年后还可以卖了。

    \subsubsection*{您目前打算毕业后直接就业还是继续读博深造?对于您master之后的去向选择而言,在国内读研和在国外读研有何优劣?就您目前打算的方向而言(读博/工作),在master期间需要做一些怎样的准备?}
    我现在在美国工作快三年了,找工作的过程比较顺利,面了几个就拿到了offer。现在美国CS专业就业比较好,如果研究生期间GPA高于3.0(越高越好)再有学一些专业课程,做过一些课程项目或者实习的话,找工作基本没问题。                
    
    \subsubsection{想说的话:}
    对自己多一点自信,在学校里有时候会迷茫,但是只要好好学习锻炼自己,EECS在美国的留学生找到的工作都是很好的。

\clearpage
\subsection{欧阳澄宇(Stat, MS@Mizzou)}
    \textbf{留学教育背景:}密苏里大学,统计,2014\\
    \textbf{联系方式:}oycy234@live.cn

    \subsubsection*{在密苏里大学读统计专业是怎样的体验?}
    学校很一般,没有放太多的精力在学习上,课程设置的难度也很轻松,只是为了毕业。毫无科研生活。在美国只有两个目的,一个是体验美国的体制社会,第二,找到个工作干上两年。

    \subsubsection*{您在读研期间经历过哪些实习/科研,它们的体验是怎样的?}
    在纽约,在芝加哥都有实习,这是最好的一部分,在美国工作是最好体会美国社会的契机。在经济独立的时候才能体会到美国体制的设计,中国人在美国的奋斗途径。这是在大学学不到的。

    \subsubsection*{您现在回顾当初选择留学、选择专业的初衷,在经历了留学生活后有什么新的感受?}
    没有,学习在选择出国的决定占比极低,主要去美国的目的就是为了在美国能靠自己没有父母的支援活下来,证明自己。

    \subsubsection*{目前南大自动化类的同学每年的出国比例都要低于南大本科生出国比例平均值不少,您认为造成这一现象的原因是什么?您对南大自动化(类)在读的学弟学妹们在出国读研方向的选择上有什么建议吗?}
    自动化同学整体不如管科的同学家庭环境优越,人数上的差异差得是研究生,不是博士生,不知道我说的对嘛?有钱就出国,没钱就读研。现在的环境,成本越高,收益有可能更高。

    \subsubsection*{从现在看留学时光,您会给即将开始留学生涯的大四学弟学妹们什么建议?}
    Gpa,一定要高,干什么都方便,其他差不多就行。

    \subsubsection*{您目前打算毕业后直接就业还是继续读博深造?对于您master之后的去向选择而言,在国内读研和在国外读研有何优劣?就您目前打算的方向而言(读博/工作),在master期间需要做一些怎样的准备?}
    自己的选择很重要,不要随波逐流,因为放在很长的一段时间内,兴趣才是唯一能都支持你前进的。在master期间要疯狂的投简历,锻炼面试技巧。

    \subsubsection*{学长已经就业了,可以介绍下您的求职经历和现在的工作吗?}
    到美国之后我自知是一个比较差得学校,但是这是我仅有的选择。所以如何在美国找到一份工作是我的终极目标,不管是什么样的工作。对于一个学习成绩不怎么好的学生来说,在美国想要找到工作是一个很困难的事情。唯一能够弥补的方式是在这边充分的实习,利用实习经验和面试技巧拿到工作offer。我一入学就开始了申请美国的SSN,有了这个才能打工。之后就开始在各大招聘网站扔简历,找recruiter。花了半年的时间拿到了一个纽约三个月的实习。因为项目是两年的,回学校后继续准备找最后半年的实习机会,同样花了半年时间准备。最后在芝加哥找到了一个期权交易员的实习,毕业了以后顺利的留在芝加哥工作了一年半。在2018年一月回国找到了中金公司交易员的工作。

    \subsubsection*{学长从自动化转到了统计学是出于什么考虑?自动化本科背景对统计的学习和就业有什么影响?}
    自动化转统计的考虑是,统计在美国比较好找工作,仅此而已。\\
    自动化专业知识我已经全部不记得了。转专业申请并不是那么简单,最重要的就是数学和会编程。自动化专业给我的不是知识,我觉得更是一种思考问题的范式。用系统的思维处理身边的任何事情,比如说反馈机制。自动化的思想处于社会的各个层次和各个阶段。
    
    \subsubsection*{学长提及在纽约和芝加哥都有实习,实习是最好的一部分,能否展开介绍实习经历对您个人的帮助?}
    第一段实习是在纽约,在一家世界知名的再保险经纪公司做战略实习。说句实话,工作中并没有用到任何与自动化或者统计专业相关的知识。主要的工作在于与其他实习生在一起完成一个又一个案例分析。这是我第一次必须用英文在实际的商业环境中进行沟通。这个时候开始从思维上领会了英文运用的方式。每说出一句话都会在脑中预演构思一遍,这个实习后使我更好的与美国人沟通,虽然语法并不能在口头上跟上,但是能把意思表达得更准确了。这对以后的工作学习更有意义。另一个收获就是在纽约广泛的交友,跳出了中国人的圈子,发现另一个角度的美国,思其所思,让我体会到了中国和中国人在国际上的真正地位。能更加客观的看很多事情。\\
    第二个实习是在芝加哥,我在芝加哥商品交易所做场内期权交易员。这个世界不想美国各个种族的人齐聚一堂,各司其职。芝加哥商品交易所内的场内交易员全是白人,老意大利人,老爱尔兰人,种族之间有森严的鄙视链,我是场内唯一一个留学生,亚洲人。招我的人明确告诉我,他就是看到中国有很多有钱人,招你的原因就是要把中国人的钱拉进来,给他们赚钱。刚开始没有人对你有任何感情,发自内心被鄙视孤立的感觉无以言表。第一次深刻感受到,我永远不能真正和美国人成为中国式的好朋友。我虽然能帮他们赚钱,但是更多就是个工具。所以我明确自己的位置努力工作换取我想要的,也坚定了我最后回国的决心。实习后我顺利的转正了,他们也在表面上接受了我。我最后也交了不少朋友。

    \subsubsection{想说的话:}
    珍惜大学的友谊,以后很难再有新的了。



\clearpage
\subsection{胡正鹏(ECE, MS@UT Austin)}
    \textbf{留学教育背景:}University of Texas at Austin, ECE

    \subsubsection*{在UT Austin读ECE是怎样的体验?}
    Austin是一个德州中部的一个大中型城市,同时也是德州首府。治安良好,气候温偏热,消费适中,生活丰富,经济发达,有很多半导体和计算机公司的总部或地区总部坐落于此,被称为“硅丘”。UT Austin EE,CS 全美排名都是前十,选课宽松,学费适中,奖助学金机会较多。Austin的中国学生也很多,本科,研究生都有。

    \subsubsection*{您在读研期间经历过哪些实习/科研,它们的体验是怎样的?}
    我在华为美研所,亚马逊都实习过。我认为研究所实习更多体现的是个人研究能力,和学校研究项目比较类似。他们都追求的不是具体的实现,而是想法的提出和验证。在亚马逊公司相对更要求综合能力一些,沟通能力,快速学习能力,工程能力都会受到考验和提高。\\
    在学校做RA和在研究所比较接近,但是因为是长期项目,不会像实习一样要求阶段性必须有结果,同时老师也理解学期中学生还有学习任务,所以项目的灵活性相对较高。\\
    在学校做TA,尤其是给本科生、低年级课程担任TA, 知识水平上是不会有什么问题的。主要的挑战在语言和表述能力上,是否能够听懂美国学生的问题,你的回答是否能够让学生听懂。除此之外,对于有大作业的课程,实践能力也会受到一定的考验。
    
    \subsubsection*{您现在回顾当初选择留学、选择专业的初衷,在经历了留学生活后有什么新的感受?}
    当初选择留学只因为其他道路上可预见的结果都不能让我完全满意,留学则有一定得不确定性,好在最后录取的结果不错。现在回头看,出国学习生活确实会带来一个全新的环境,全新的挑战,用现在流行的话来说就是脱离舒适区。在这种情况下,个人的潜力会因为生存的本能而得到激发。同时留学会带来很多新的文化观念的冲击,我认为接触了解多元的文化,可以帮助拓宽思维模式。一个简单的例子,我之前觉得好学生本科毕业后就一定要读研,然后在接触了美国学生后我发现,这个完全不是必须的。社会也许会是一所更好的学校,能帮助你更好的认识到自己缺什么。

    \subsubsection*{目前南大自动化类的同学每年的出国比例都要低于南大本科生出国比例平均值不少,您认为造成这一现象的原因是什么?您对南大自动化(类)在读的学弟学妹们在出国读研方向的选择上有什么建议吗?}
        \begin{enumerate}[itemindent=0pt,itemsep=0pt,parsep=0pt]
            \item 学校/专业名气小。实事求是的说,南大是一个以文理为主的学校,在工科上和兄弟院校比还是有相当的差距,这个就导致在录取时没有加分。
            \item 专业不对口。自动化本科在我看来是EE+CS+MATH的混合体,好处在于拓宽了视野,坏处在于基本功不扎实。除非学生有意识的在后期专修一项,不然成绩单上课程会显得有欠缺。
            \item 简历单薄。除课程外,学生们没有什么其他方面的成就。
            \item 缺乏海外交流。海外交流交换既可以是学校组织的,也可以是自己主动联系的。
        \end{enumerate}
    \subsubsection*{从现在看留学时光,您会给即将开始留学生涯的大四学弟学妹们什么建议?}
        \begin{enumerate}[itemindent=0pt,itemsep=0pt,parsep=0pt]
            \item 读硕士准备找工作的同学,早做准备。刚来美国的第一个学期,就要去参加career fair寻找实习机会。
            \item 多开拓眼界,体验不同生活,这很可能是你们在国内最后的一个“长假”了。
        \end{enumerate}

    \subsubsection*{我们了解到您目前已经master毕业工作,您目前的就业方向是什么。就您目前的就业去向而言,在master期间需要做一些怎样的准备?}
    软件工程师。\\
    有一个不差的GPA。好的GPA是敲门砖。\\
    学好CS基础知识:算法和数据结构,基本的组成原理。也许有人会说刷题才是王道,但我要说,刷题也许能帮你找到工作,却不能帮你做好工作。\\
    语言沟通和表达能力。这个在面试和日常工作都很重要。我对于面试的理解就是三个字,“让他爽”。如果面试官面对的是一个磕磕巴巴话都说不利索的人,我很难想象他会高兴。\\
    多练习算法题。这个可以在一亩三分地网站上看到很多,我就不赘述了。



\clearpage
\subsection{饶博(CS, MS@OU)}
    \textbf{留学教育背景:}University of Oklahoma, CS MS, 2017	\\
    \textbf{联系方式:}raobo78@gmail.com

    \subsubsection*{在University of Oklahoma读CS是怎样的体验?}
    我来这个学校真是一言难尽啊,学弟学妹有机会尽量去更好的学校(专业排名,位置)。体验的话,上课不难,即使是英语不好,也可以看板书和课件,没有什么难理解的。课后最好看看书,作业一定要按时按量完成(对我而言作业比上课重要多了)。但是有时候作业和projects真的很多很复杂,多留点时间去做。我暂时没有做科研,生活还比较悠闲。在美国的感觉就是孤独、寂寞、远离熟悉的故乡,感觉和过去的生活相距遥远。适合埋头努力,开始洗心革面的新生活吧。

    \subsubsection*{您在读研期间经历过哪些实习/科研,它们的体验是怎样的?}
    暂时没有。忠告是要提前找实习(一般秋季career fair就可以找第二年暑假的实习或者工作),科研的话跟好老板,上好贼船。

    \subsubsection*{您现在回顾当初选择留学、选择专业的初衷,在经历了留学生活后有什么新的感受?}
    感受就是当初没有好好准备,迷迷糊糊,甚至没有想好为什么要出国,是否要出国,耽搁了很多时间,准备也不够充分。建议学弟学妹早点找清方向,gap一年也是可以的。

    \subsubsection*{您选择了在美国读master,请问您的选择相比于国内和其他国家地区而言有什么优劣呢?}
    计算机专业还是推荐来美国的,尽管比较难,但是就业率和学习质量相对较高。但是建议去就业环境好的城市(主要大城市,加州德州纽约)。

    \subsubsection*{对于以博士学位为最终学位的深造而言,先在国内读master作为跳板,国外读master作为跳板和本科直接申请Ph.D.三者上有何优劣?您就此对学弟学妹的建议是什么?}
    国内读master: 省钱,提前适应研究生节奏,好找女朋友。缺点是费时间(三年),适合长远研究型方向,基本上这辈子就投在科研上了。如果半途决定不读博了比较可惜。\\
    国外读master: 最大的好处是有利于毕业找工作,特别是好找工作的专业。要读博可选择性也比较高。只是花费大,在国外待的时间会比较长。\\
    直接读Ph.D.: 想通了一条路走到黑是最好的(不过也怕半路学不下去)。而且博士有工资,博士不怎么花钱。但是一定要选好专业方向和导师。相对来说比较难,需要本科有一定科研背景。

    \subsubsection*{目前南大自动化类的同学每年的出国比例都要低于南大本科生出国比例平均值不少,您认为造成这一现象的原因是什么?您对南大自动化(类)在读的学弟学妹们在出国读研方向的选择上有什么建议吗?}
    基本上是因为相比于南大其他专业,本专业竞争力不足、没有进取氛围、专业实力较弱(锅就不知道该让谁背了)。对于学弟学妹,建议是想好出路(工作、保研、考研、出国),如果要出国,想好申请方向,及早准备,也能去不错的学校。

    \subsubsection*{从现在看留学时光,您会给即将开始留学生涯的大四学弟学妹们什么建议?}
    刚来美国的留学生要面临几个问题:签证、机票、公寓(室友)、开学前准备、开学后日常生活。前两者及早准备、查询好。找公寓是个技术活,建议在微信群、论坛、邮件找寻中国室友(和外国室友一起住总得来说弊大于利)。然后从机场到学校、入住公寓(有些公寓非要某个日期以后才能入住,请看好)、买生活用品等等可能需要开车的事,建议在群里早点勾搭个有车的学长/学姐,不然挺愁人的。如果以后需要的多,还是尽早买车。基本上多和学长学姐交流,能学到很多经验和忠告。

    \subsubsection*{您目前打算毕业后直接就业还是继续读博深造?对于您master之后的去向选择而言,在国内读研和在国外读研有何优劣?就您目前打算的方向而言(读博/工作),在master期间需要做一些怎样的准备?}
    先找工作,能工作就先工作吧,博士还没想好。我的准备就是多学点实际知识,早点准备找工作,最好要有实习经历。以及多去其他州gay某些gay学$\male$长。

    \subsubsection{想说的话:}
    在人生每一个阶段,都尽量不要做会让你后悔的事。

\clearpage
\subsection{张艺川(CS, MS@NEU)}
    \textbf{留学教育背景:}Northeastern University(US),Computer Science,2016 Spring\\
    \textbf{联系方式:}Wechat : zyc\_nju  邮箱:yczneu@gmail.com

    \subsubsection*{在NEU读CS是怎样的体验?}
    课程内容多,作业和project多。不仅不能缺席任何一堂课,而且还需要课后自己阅读和练习。有的课程需要和其他同学一起做项目,要求有基本的英语沟通能力。硕士一般一周2到3门课,作业用时20到30小时每周。如果有兴趣可以去旁听其他老师的课,旁听不需要缴费,课程资源一般在Piazza这种公开的网站上,老师会每次更新每周的内容。不懂的问题需要及时向助教(TA)提问,或者自己stackoverflow上解决。一般一门课开始前几周的runtime environment setup费时费力,需要耐心。代码提交的时候TA可能会code review,一般非常严格。期末一般会有project或者考试,不过基本不难。成绩一般分为A,A-,B+,B等等,A一般前5\%还是稍微有难度。不管考试准备如何,不要有任何作弊的行为。\\
    我的硕士项目不是面向科研的,主要是面向美国就业的。老师也有从业界请来的经验丰富的高工。基本上除了上课,做作业还要刷题。刷题主要是针对算法,像一些online OJ网站提交代码,用的比较多的是Leetcode,Lintcode 和POJ。我的情况是业余时间做题比较多,截至到毕业的时候大概做了600+题,submission1400。
    生活方面美国生活没有国内丰富,KTV聚会烧烤party我去得比较少,平时和周末会打篮球和健身,有时候也会外出旅游。

    \subsubsection*{您在读研期间经历过哪些实习/科研,它们的体验是怎样的?}
    我在第二年的时候去实习,主要也是做software engineer intern, 实习时间为6个月。实习会有一个mentor带,每周有一次department大会和组内小会,周五有manager 1 on 1汇报。公司对实习生要求不高,每周完成manager给你的任务就好,有什么不懂的不耻下问,其他工程师都会帮助你。

    \subsubsection*{您现在回顾当初选择留学、选择专业的初衷,在经历了留学生活后有什么新的感受?}
    我当时是在国内工作,一方面是对公司发展前景不看好,二是想继续学习,因此选择了出国。专业选择CS是目前美国就业形势最好的专业,而且自动化也学习了一些CS的基础课程,所以并不算跨太远的专业。经历留学生活来看一方面是增长了眼界,也培养了自主的精神,租房选课等等都是得自己操心。美国社会比较包容,各色人种混杂,相对国内体验不同。

    \subsubsection*{您选择了在美国读master,请问您的选择相比于国内和其他国家地区而言有什么优劣呢?}
    相比较于国内三年硕士而言,美国硕士压缩到了2年,因此绝不是一件轻松的事情。有谣言说国外硕士水,但是就我目前来看,至少我学习得并不轻松。美国硕士优势在于环境可能会更好一点,技术的交流发展更快更通畅,没有墙。任何人都可以在Github上学习别人的代码,我所在的波士顿有很多tech talk,大多数对学生免费,很多人也乐于和你探讨他们用的技术。美国的劣势在于生活环境没有国内的安全,而且生活费用高,不是每个人都能承受。美国大部分城市基础设施差,公共交通远远劣于国内。其他国家我不太了解,不过就学校和学术来说,留学还是建议首选美国。

    \subsubsection*{目前南大自动化类的同学每年的出国比例都要低于南大本科生出国比例平均值不少,您认为造成这一现象的原因是什么?您对南大自动化(类)在读的学弟学妹们在出国读研方向的选择上有什么建议吗?}
    我从来都认为留学并不普遍适用于每个人,每个专业,要认真考虑自己的需求再做出选择,所以和其他专业比较意义不大。在我看来自动化本科课程设置比较general,有点万金油的意思,这样硕士或者博士的选择面比较广。美国并没有这样general的专业,专业都划分得比较细,所以美国并没有自动化的本科或者硕士。这样其实自动化能选择的美国硕士或者博士就有EE,CE和CS,根据和计算机/芯片硬件的偏好程度来选择,EE基本上和硬件打交道,CS完全是软件,算法和程序。

    \subsubsection*{从现在看留学时光,您会给即将开始留学生涯的大四学弟学妹们什么建议?}
    最重要的建议是雄关漫道真如铁,而今迈步从头越。既然迈出了留学第一步,要再接再厉,抓紧时间学习。在美国要学会自己管理自己,不枉费付出的青春和金钱。

    \subsubsection*{您目前打算毕业后直接就业还是继续读博深造?对于您master之后的去向选择而言,在国内读研和在国外读研有何优劣?就您目前打算的方向而言(读博/工作),在master期间需要做一些怎样的准备?}
    我目前已经在美国开始工作。我认为在国内读研的优势在于南大的硕士在国内可谓是天之骄子。很多国内的企业相信南大的牌子,这是我们南大人的优势,基本上简历关不会被为难。在美国而言,我并不是native speaker,语言上和美国人或者印度人处于劣势,美国的企业大多时候根据职位的match程度和面试表现来判定。另外美国企业有耐心招new graduate student的并不多,而国内很多企业的校招通道每年招收大量的应届生培养。因此在美国找第一份工作难度较大,而目前随着签证收紧和新总统的政策,难度还在增加。\\
    Master期间一定要在美国本土的企业实习,不管是能拿return offer还是当作跳板都非常重要。没有美国公司的实习,全职面试可能会被质疑你有没有适应美国的工作方式和工作环境。如果实在没找到美国的实习,回国实习也是可以的。效果是:美国企业实习>>国内企业实习>没有实习。我周围而言还没听到过硕士期间没有实习过的同学找到了美国的全职工作。找实习/全职可以找人内推,增大获得面试的机会。可以是南大的学长学姐,也可以是LInkedin上该公司的任何人。一亩三分地论坛留学生用得比较多,里面信息可能会有用。

    \subsubsection{想说的话:}
    我从来都认为留学并不普遍适用于每个人,每个专业,要认真考虑自己的需求再做出选择。想想自己要远渡重洋除了拿到学位还想要什么。留学的代价很大,不仅仅是金钱和时间,还有不能和家人经常团聚,枯燥的生活,高强度的学习等等。但是走出这一步,恭喜你走出了自己的comfort zone,你大有可为。





\clearpage
\section{香港篇}

\subsection{彭凤超(CSE, Ph.D.@HKUST)}
    \textbf{留学教育背景:}Hong Kong University of Science and Technology,CSE,2014入学

    \subsubsection*{在HKUST读CSE是怎样的体验?}
    港科大CSE的课程设置涵盖了所有CS领域热门的研究领域,如AI,Database,Network,Software Engineering,算法,密码学等,并且会随着CS的发展潮流,适当的调整各门课程的开设频率,甚至是增开新课程,比如近两年新开的并行计算方面的课程。本科生的课程则是全面覆盖。授课形式以lecture为主,考核方式基本采用作业/project/考试的方式,一门课几种方式可能兼而有之。师资力量当属国际一流水平,所有的教授都会开课,也都会参与本科生的毕业论文工作,每门课都有本系优秀的研究生、博士生【夸自己一下咩哈哈】担任助教。可以说课程质量绝对杠杠的。
    \subsubsection*{您在读研期间经历过哪些实习/科研,它们的体验是怎样的?}
    我的科研经历基本就是做了几篇文章,文章质量么,很惭愧,除了一篇顶会文章以外,难说优质。感触就是做论文主要还是靠自己,第二作者能贡献10\%的力量就很好了,不只是我,我的很多同学都是这样。多读文章、多讨论、多尝试【不管想法看起来多扯淡】,再加上一定的运气,以及导师关键的一点点指导,文章才能成,读博还是挺艰难的。我有两段实习经历,一个在阿里,一个在地平线。阿里的体验就是,毕竟大厂,所有事情有章可循,分工明确,办事情效率很高,员工日常985本硕,海归Ph.D.一搂一大把。地平线规模不大,但技术实力不遑多让,AI芯片的研发已经走在国内前列。两次实习都是参与到了一个项目中去,任务很具体,数据清洗,设计模型,实现模型,调参,反复尝试bla。对于练手、寻找研究idea、积累简历、近距离接触业内大佬,都有帮助。
    \subsubsection*{您现在回顾当初选择留学、选择专业的初衷,在经历了留学生活后有什么新的感受?}
    还是应该申请之前就仔细了解一下,自己喜欢做哪方面的研究,不要为了读博而读博,我进门以后,万幸遇到一个我有兴趣的方向,回想一下是有些后怕的。
    \subsubsection*{您选择了在香港攻读博士学位,它和其它选项相比优劣有哪些?就以博士为最终学位的学术深造而言您在出国深造国家的选择上对学弟学妹有什么建议?对于以博士学位为最终学位的深造而言,先在国内读master作为跳板,国外读master作为跳板和本科直接申请Ph.D.三者上有何优劣?您就此对学弟学妹的建议是什么?}
    我没有去欧美学校的经历,所以难说香港的Ph.D.和国外的有什么优劣之分。至于先读master还是直接Ph.D.,我觉得如果你很确定就研究某一个方向,且确定要读博,那就直接申博士,没有必要先读个master浪费时间。当然了,如果你要读名校Ph.D.,本科直接申请可能申不上,拿个master做跳板,蹭几封大佬推荐信,是个很好的选择。

    \subsubsection*{目前南大自动化类的同学每年的出国比例都要低于南大本科生出国比例平均值不少,您认为造成这一现象的原因是什么?您对南大自动化(类)在读的学弟学妹们在出国读研方向的选择上有什么建议吗?}
    我觉得原因是自动化这个专业,在海外基本没有了,我们出去要么申EE,要么CS,可能这会让一部分同学觉得出去也学不到啥,专业也不对口。但事实,完全不是这样,海外招生对于专业的要求十分不严格,EE、CS方向,你只要能和编程、算法、电路沾上边就行,有的cs老板还特爱招物理系、数学系毕业的学生。所以大胆的申啊!康忙北鼻动特比晒
    \subsubsection*{从现在看留学时光,您会给即将开始留学生涯的大四学弟学妹们什么建议?}
    找个女朋友
    \subsubsection*{就您的了解而言,您目前所读专业的Ph.D.未来发展前景如何,有哪些方向(学术界和工业界),您个人更期待哪一个方向?对于这些方向,在Ph.D.在读期间该做哪些准备?}
    cs,具体到ai领域,出路主要是继续深造走学术道路,或者进企业,进企业也分去业务部门和去研究部门,前者是应用导向,后者和学术道路没差。像我这种文章少的只能选择去企业了。
    \subsubsection{想说的话:}
    找个女朋友
        
\clearpage
\subsection{李tǐng光(EE, Ph.D.@CUHK)}
    \textbf{留学教育背景:}香港中文大学,电子工程系EE,2016入学\footnote{“王”字旁,右边“廷”的tǐng无法正确显示,故征得学长同意的使用拼音,特此说明并向学长表示歉意。}\\
    \textbf{联系方式:} tgli0809@gmail.com
    \textbf{个人主页:} \url{http://www.ee.cuhk.edu.hk/~tgli/}

    \subsubsection*{在香港中文大学读EE是怎样的体验?}
    先说位置,香港中文大学(CUHK)在新界,距离口岸坐地铁只有半个小时,是香港所有大学里面去深圳最方便的一个。另外因为离市区较远,因此CUHK占地面积据说是香港其他六所大学之和,相对香港其他学校,CUHK的Ph.D.除了第一年可能申请不到宿舍之外这几年都可以住在学校,这是其他学校所没有的。
    总体上,香港的课程难度,工作量,有用程度应该是远胜于内地的,我们这届的Ph.D.要求至少修五门(我的下一届改成七门了),所以第一年和第二年上半年基本都在挣扎于课程。但作为Ph.D.十分矛盾的一点是,科研成果有时候和课程关系不大,而香港这边学制又比较短,所以需要权衡两者之间的关系。
    科研上比较看方向看导师,不同导师风格不同毕业要求也不同。我们导师散养风格,一切靠自己。
    \subsubsection*{您在读研期间经历过哪些实习/科研,它们的体验是怎样的?}
    我们实验室研究方向是机器人,我个人是强化学习用于地面移动机器人(详见我的个人主页)。总体而言科研是一个挺痛苦的过程,它和本科阶段上课考试完全不一样,因为你做的东西往往没有正确答案,需要不停的摸索,同时紧跟时代的方向,就是努力了也不一定会出成绩。但是当文章发表,去参加会议的时候又特别幸福,总体上是喜乐参半吧。
    \subsubsection*{您现在回顾当初选择留学、选择专业的初衷,在经历了留学生活后有什么新的感受?}
    我来香港的动机很简单,这边的Ph.D.时间短,性价比高,而且申请流程相对简单,不需要考GRE。现在来看这个选择也没什么问题,适合懒人。
    \subsubsection*{您选择了在香港攻读博士学位,它和其它选项相比优劣有哪些?就以博士为最终学位的学术深造而言您在出国深造国家的选择上对学弟学妹有什么建议?对于以博士学位为最终学位的深造而言,先在国内读master作为跳板,国外读master作为跳板和本科直接申请Ph.D.三者上有何优劣?您就此对学弟学妹的建议是什么?}
    这个我着重讲讲,欧洲情况我不太清楚,澳洲留学据说比较水,我着重比较比较美国香港和新加坡。美国的Ph.D.(人工智能方向)最近很难申请,用我朋友的话说就是神仙打架,很多申请人本科就坐拥好几篇顶会paper,但是美国Ph.D.质量非常高,前景也非常好,可以先申请美国master作为跳板,至于国内读个master作跳板我个人感觉没必要。美国适合很早就开始准备出国(包括语言成绩,GPA,比赛,交换,尽早进实验室进行科研)的同学。剩下新加坡和香港则各有利弊。新加坡的两所学校排名更靠前,新加坡更国际化一些,但香港优势在于紧邻大陆,暑期回去实习非常方便,同时对国内政策,产业情况都比较清楚,可以提早打下基础。
    我认为香港非常适合那些GPA很靠前,但是出国准备的晚了的或者之前只准备保研的同学,因为每年这边都会有summer workshop招收这样的学生,我觉得算是一条捷径。

    \subsubsection*{目前南大自动化类的同学每年的出国比例都要低于南大本科生出国比例平均值不少,您认为造成这一现象的原因是什么?您对南大自动化(类)在读的学弟学妹们在出国读研方向的选择上有什么建议吗?}
    还是没有形成出国的风气,大家对于出国了解的比较少。我觉得在家庭经济条件允许的情况下相比国内读研,我还是倾向国外读,美国Ph.D.>香港新加坡Ph.D. = 清华Ph.D.(清华势头很好,可以认真考虑)>美国master>国内master。个人看法仅供参考。
    
    \subsubsection*{从现在看留学时光,您会给即将开始留学生涯的大四学弟学妹们什么建议?}
    好好的玩吧,不要在这个时间假装学习了。
    
    \subsubsection*{就您的了解而言,您目前所读专业的Ph.D.未来发展前景如何,有哪些方向(学术界和工业界),您个人更期待哪一个方向?对于这些方向,在Ph.D.在读期间该做哪些准备?}
    目前人工智能产业正处于风口浪尖,工业界的待遇很可观。Ph.D.期间适当多关注新闻,了解产业情况,条件允许的情况下可以进入公司实习。

    \subsubsection{想说的话:}
    相比保研,考研,其实出国的路更难走,需要尽早进行准备。祝学弟学妹们好运!

        
        

   


\clearpage
\subsection{汤明辉(BME, Ph.D.@CUHK)}
    \textbf{留学教育背景:} 香港中文大学,生物医学工程系(微流控),硕博,2014-2018(已毕业)\\
    \textbf{联系方式:} 289460137@qq.com

    \subsubsection*{在香港中文大学读生物医学工程是怎样的体验?}
    四年拿到博士学位,而且奖学金cover在港所有花费后还有剩余,性价比高。\\
    港校(港大,港科大,港中大等)因教师薪水全球范围内都很有竞争力,所以师资力量雄厚;学校管理水平高;研究经费当前相比于同等国际排名的内地学校而言并没有优势。\\
    生活成本高,好在奖学金也不少;饮食清淡,西餐较多,好在港中大离深圳很近,可以周末去深圳打打牙祭;来港也可以体验不同的文化,虽然差异没那么大。

    \subsubsection*{您在读研期间经历过哪些实习/科研,它们的体验是怎样的?}
    科研:管理比较规范,大部分博士生和导师之间很多时候更像朋友,地位比较对等。
    
    \subsubsection*{您现在回顾当初选择留学、选择专业的初衷,在经历了留学生活后有什么新的感受?}
    主要考虑香港离家不是很远,与之相邻的深圳也是一个很好的就业的地方。\\
    专业更多的是要依托于本科知识背景来做选择。\\
    当前觉得对当初的选择还是很满意的。\\
    不想离家太远,又想快速拿到博士学位的可以考虑港校;真心想走科研路线的,还是建议美国。

    \subsubsection*{您选择了在香港攻读博士学位,它和其它选项相比优劣有哪些?就以博士为最终学位的学术深造而言您在出国深造国家的选择上对学弟学妹有什么建议?对于以博士学位为最终学位的深造而言,先在国内读master作为跳板,国外读master作为跳板和本科直接申请Ph.D.三者上有何优劣?您就此对学弟学妹的建议是什么?}
    优:四年全奖博士学位,性价比高,回家方便;缺:并没有能真正体验欧美的文化,虽然是英文教学,但是周围博士生大多为大陆人,平常还是普通话交流更多;整体科研实力不如欧美。\\
    建议:拿博士学位,第一要考虑毕业去业界还是学术界;其次,具体选择哪个研究方向,此方向适不适合学术界 or 业界(这个很关键);然后是导师,抛开专业技能,导师的人品也很重要;最后才是专业和学校。\\
    能硕博或者直博就一步到位。有的master学位真的很水,读了并不能为申请Ph.D.加分,比如港校的MSC;当然,美国名校的master还是很值得一念的。

    \subsubsection*{目前南大自动化类的同学每年的出国比例都要低于南大本科生出国比例平均值不少,您认为造成这一现象的原因是什么?您对南大自动化(类)在读的学弟学妹们在出国读研方向的选择上有什么建议吗?}
    主要原因:整体院系资源有限。\\
    建议:如果还想继续深造,一定要坚定出国出境的念头,南京大学已经是很好的起点了;能出去就不要想着继续留下来。

    \subsubsection*{从现在看留学时光,您会给即将开始留学生涯的大四学弟学妹们什么建议?}
    英语是基础,搞清楚目标学校的英语最低线,过了就OK了,没必要刷多高;申港校GPA很关键,申美校科研经历很关键;搞好GPA,进进实验室,最好发发论文,你的申请之路就顺利多了。

    \subsubsection*{就您的了解而言,您目前所读专业的Ph.D.未来发展前景如何,有哪些方向(学术界和工业界),您个人更期待哪一个方向?对于这些方向,在Ph.D.在读期间该做哪些准备?}
    我做的微流控这一块,算是属于医疗器械这个领域的,跟信息工程专业的背景还是很契合的。诊断类医疗器械的检测端还是更多的依赖于光机电这一块。这个也算是当前诊断类医疗器械的一个革新技术了。当前不管国内还是国外,业界都看好。对于生物医学工程这个大方向,水平有限,也说不太清楚。
                    
    \subsubsection{想说的话:}
    能出去就不要继续留下来。


\clearpage
\subsection{焦文祥(CSE, Ph.D.@CUHK)}
    \textbf{留学教育背景:}香港中文大学(CUHK),计算机科学与工程系(CSE),2017年8月\\
    \textbf{联系方式:}wxjiao@cse.cuhk.edu.hk

    \subsubsection*{在香港中文大学读CSE是怎样的体验?}
    CSE的Ph.D.需要修7门课,其中至少4门CSE的课程。第一年选修过模式识别、概率图模型、在线学习、图论、优化算法、矩阵分析,这几门课都是比较高水平的课程,授课老师也是在学界和业界知名老师。课程任务比较合理,有几次作业,有考试,也会有写小论文,平时需要花一定的时间学习,否则会有些吃力。
    科研方面,我们组方向比较分散,导师也不限制大家的方向。第一年主要是上课,看看paper,老板也不催我们的科研进度。一般来说,过了大半年,每个人能确定自己的基本研究方向,然后做一些小的尝试。老板也鼓励我们不管创新性有多少,先尝试把一篇完整的paper写出来,让我们对写paper这个过程有个比较清楚的认识。老板人还是很nice的,组里面大家也会互相请教,氛围比较轻松。

    
    \subsubsection*{您现在回顾当初选择留学、选择专业的初衷,在经历了留学生活后有什么新的感受?}
    留学的初衷是不想在原来的方向(光学工程)继续科研工作,因为个人感觉光工在业界的前景比较窄,自己又不想去高校。正好正值互联网和机器学习兴起,所以才决定转到CSE,之后再进入业界。\\
    目前来看,我很高兴自己做了这样的选择。CSE特别是机器学习、深度学习方向有很多有趣的方向,而且CSE方向的Ph.D.在业界也很吃香。留学让自己的眼界拓展很多,在更轻松开放的环境里,做自己喜欢的科研,感觉就很棒。

    \subsubsection*{您选择了在香港攻读博士学位,它和其它选项相比优劣有哪些?就以博士为最终学位的学术深造而言您在出国深造国家的选择上对学弟学妹有什么建议?}
    选择在香港读博是因为,从光学工程转CSE相对美国更容易一些,香港博士的毕业时间相对更短一点,而且个人比较想来感受一下香港的生活。
    有条件和机会还是去美国吧,大牛导师比较多,毕业后的机会也更多一些。

    \subsubsection*{对于以博士学位为最终学位的深造而言,先在国内读master作为跳板,国外读master作为跳板和本科直接申请Ph.D.三者上有何优劣?您就此对学弟学妹的建议是什么?}
    一般来说,想好了要去国外留学读博,就尽量本科毕业就争取一下吧,毕竟读博时间比较长,早点开始对之后的人生计划都好一点。次一点的就是在国外读个master,然后在找机会留下继续读博。至于在国内读master作为跳板,我个人不太推荐,除非是国内的领先国际的科研方向,否则待遇不好压力还大。

    \subsubsection*{目前南大自动化类的同学每年的出国比例都要低于南大本科生出国比例平均值不少,您认为造成这一现象的原因是什么?您对南大自动化(类)在读的学弟学妹们在出国读研方向的选择上有什么建议吗?}
    南大自动化类估计相当一部分人都留在南大读研究生了吧。
    有机会还是出去开开眼界吧。个人觉得南大的研究生教育确实不算好的,自动化类的研究生很多还处于散养的状态,很难说能培养多好的科研人才。

    \subsubsection*{从现在看留学时光,您会给即将开始留学生涯的大四学弟学妹们什么建议?}
    既然都努力出去留学了,就好好珍惜吧;选择一个自己喜欢的方向,为之而奋斗。注意劳逸结合,科研之余加强身体锻炼,有空多出去转转。

    \subsubsection*{就您的了解而言,您目前所读专业的Ph.D.未来发展前景如何,有哪些方向(学术界和工业界),您个人更期待哪一个方向?对于这些方向,在Ph.D.在读期间该做哪些准备?}
    CSE无疑是目前最热门的发展方向之一,主要是这些年互联网企业兴起以及人工智能的火爆带来了对计算机人才的需求,计算机方向的Ph.D.未来的发展前景自然也是比较广阔。在这样一个全民AI的时代,计算机方向以机器学习、深度学习最为热门, 其中包括对计算机视觉,语音处理,自然语言处理等子方向的研究。CV计算发展经历了这几年的大爆发,目前来看水很深了;个人比较看好 NLP,技术相对不够成熟,应该会有一个爆发期。
    准备的话,主要是对主流技术有个了解,熟悉一下各种深度学习平台,例如Pytorch、TensorFlow等;从复现已有模型,再到加入自己的想法,需要一定的知识和时间的积累。
    
    \subsubsection{想说的话:}
    说一些我对CUHK老师 的看法。不夸张的说,CUHK的老师基本上都是欧美顶尖高校毕业的优秀科研工作者,他们在自己的领域卓有成就,而且眼界宽广为人礼貌。以我接触的几位老师为例,他们待人很绅士,将学生视为自己的朋友,没有给人以高人一等的姿态。他们乐于与学生交流有趣的话题,鼓励学生劳逸结合,鼓励学生去公司实习、去外面开拓眼界。我们老板更是提醒我们要把年假都用掉,不然过期了就没有了。这些,以我在南大6年的经历,基本上很少能看到。



\clearpage
\subsection{王志(SEEM, Ph.D.@CityU)}
    \textbf{留学教育背景:}City University of Hong Kong, Department of Systems Engineering and Engineering Management, Ph.D., 2015.09-2019.08\\
    \textbf{联系方式:}896743110@qq.com

    \subsubsection*{在CityU读SEEM是怎样的体验?}
    读博还是一件比较艰辛而持久的事情,需要自己去发现问题、探索研究的方法以及持之以恒的努力吧。如果能找到自己喜欢的课题,对以后的职业有一个规划并且喜欢,那读博还是一件很值得的事情。\\
    读博的时间还是比较自由的,一般是自己安排,看看论文,跑实验或者投投稿。生活上,在香港的话大陆生博士生的圈子还是比较小,不像在国内一个组几十号人,需要耐得住寂寞,哈哈。
    
    \subsubsection*{您在读研期间经历过哪些实习/科研,它们的体验是怎样的?}
    一直在科研吧,选题方向很重要。要选那种有世界级大牛引领的方向,选那种有很多课题组也在做的大方向;千万不要一个人死抓一个点,结果发现全世界都没其他人在做。

    \subsubsection*{您现在回顾当初选择留学、选择专业的初衷,在经历了留学生活后有什么新的感受?}
    最终还是会选择回大陆工作,出来留学是为了更清楚自己想要的生活吧。
    
    \subsubsection*{您选择了在香港攻读博士学位,它和其它选项相比优劣有哪些?就以博士为最终学位的学术深造而言您在出国深造国家的选择上对学弟学妹有什么建议?}
    这个对比我不清楚,当时只申请了香港。课题组的话建议选择年轻老师做导师吧。

    \subsubsection*{对于以博士学位为最终学位的深造而言,先在国内读master作为跳板,国外读master作为跳板和本科直接申请Ph.D.三者上有何优劣?您就此对学弟学妹的建议是什么?}
    我是直接本科申请的,跳板的话比较建议香港或者美国的master吧,一般只需要一年就可以毕业了,时间上投入不会太多。

    \subsubsection*{目前南大自动化类的同学每年的出国比例都要低于南大本科生出国比例平均值不少,您认为造成这一现象的原因是什么?您对南大自动化(类)在读的学弟学妹们在出国读研方向的选择上有什么建议吗?}
    这无可厚非吧,相反也挺好的啊。南大是传统文理科型学校,理科专业有实力也有出国深造的需要,而且氛围很浓。自动话是工科,好就业,但是出国本来难度就比理科大得多。在自动话专业的选择面比较宽,建议往人工智能方向靠一靠,也是很不错的。
    
    \subsubsection*{从现在看留学时光,您会给即将开始留学生涯的大四学弟学妹们什么建议?}
    选好导师,选好课题,耐住寂寞,你会追寻到想要的生活的。

    \subsubsection*{就您的了解而言,您目前所读专业的Ph.D.未来发展前景如何,有哪些方向(学术界和工业界),您个人更期待哪一个方向?对于这些方向,在Ph.D.在读期间该做哪些准备?}
    读博的话,比起专业,更重要的是你做的课题是什么方向的。我目前在SEEM系,但做的内容是强化学习人工智能相关的。如果想进工业界的话,最好选择人工智能、机器人这种前沿的研究方向;如果立志进学术界,那什么研究方向都ok,自己喜欢而且能出成果就好。
                    
    \subsubsection{想说的话:}
    南京大学出去的学生都还是很不错的,要对自己有信心。出国之后,你身边的很多都是C9和985里面top的学生,要时刻保持谦卑的心态学习他们的优点。留学的生活应该会很忙碌,祝愿学弟学妹们能在忙碌的充实中学到知识,更能看清楚未来的路和自己最想要的真实生活。
        

    
\clearpage
\subsection{王煜烨(BME, Ph.D.@CUHK)}
    \textbf{留学教育背景:}香港中文大学,生物工程方向,2017年\\
    \textbf{联系方式:}微信:PermanentYY

    \subsubsection*{在香港中文大学读生物工程是怎样的体验?}
    整体的氛围非常轻松和自由。课程完全都是由自己选择,也可以跨专业选课,都非常方便。科研上主要看老板的性格和要求等等,像我们组的话就非常轻松了,老板不push,所以大多数时间都是自己看着办吧hhh. 我个人生活上也很ok, 很幸运地申到了单人间,唯一要吐槽的就是吃的不太好。

    \subsubsection*{您现在回顾当初选择留学、选择专业的初衷,在经历了留学生活后有什么新的感受?}
    香港还算是比较符合我的期待的,虽然经历了被迫转专业的尴尬(EE到BME),但这边整体的学术氛围都比较符合我的心理预期,实验室的师兄师姐也都很照顾我,总之不管生活还是学习上体验都还算不错。

    \subsubsection*{您选择了在香港攻读博士学位,它和其它选项相比优劣有哪些?就以博士为最终学位的学术深造而言您在出国深造国家的选择上对学弟学妹有什么建议?对于以博士学位为最终学位的深造而言,先在国内读master作为跳板,国外读master作为跳板和本科直接申请Ph.D.三者上有何优劣?您就此对学弟学妹的建议是什么?}
    香港的优势主要是离家近,读Ph.D.的时间成本低,相对也比欧美国家好申请一些。\\
    我的建议是,如果真的已经很确定自己想在科研上有所建树,那还是尽量去美国,就算一开始申不到好学校的话也可以先申美国的master作为跳板。因为学术界来说对美国的认可度还是要比香港高很多的。如果还不是很确定自己对科研的热情,又不愿意花太多钱出国镀金,香港的Ph.D. 还是性价比非常高的。\\
    第三个问题的回答仅仅针对香港,美国不清楚。国内读master作为跳板没有必要,除非是因为本科学校特别不好然后先去比较好的学校读研,否则以南大本科为例,本科就来香港读博其实是非常容易的,性价比也极高,在内地读完研究生再来香港读博和本科直接读博所需时间是相同的,所以我个人非常不建议前者。

    \subsubsection*{从现在看留学时光,您会给即将开始留学生涯的大四学弟学妹们什么建议?}
    还是要多多考虑自己期待的到底是什么,明确自己心之所向,才能一往无前。所有的纠结和困惑其实都还是不太明确自己想要什么,但这真的是需要花费大量时间精力弄清楚的一件事情。

    \subsubsection*{就您的了解而言,您目前所读专业的Ph.D.未来发展前景如何,有哪些方向(学术界和工业界),您个人更期待哪一个方向?对于这些方向,在Ph.D.在读期间该做哪些准备?}
    我目前读的小方向是微流控,学术界和产业界其实都有发展的可能性。目前还是更想去产业界吧,但其实不管想去哪都还是要加油科研发paper啦,博士期间最重要的事情必然还是paper!

    \subsubsection{想说的话:}
    我有写过一篇香港夏令营申请的攻略。感兴趣的学弟学妹可以戳以下链接:\\
    \url{https://www.zhihu.com/question/41537308/answer/247419737}
    


\clearpage
\subsection{赵晨项(ECE, Ph.D.@HKUST)}
    \textbf{留学教育背景:}Hong Kong University of Science and Technology (HKUST), ECE, 2013年入学

    \subsubsection*{在香港科技大学读ECE是怎样的体验?}
    港科大ECE专业的课程压力不算大,只要修够一定的学分满足系里的要求跟学校的要求就OK了。Ph.D. 的话在前两年需要通过PQE(Ph.D. Qualification Exam),才能从Ph.D. student转为Ph.D. candidate,主要就是从14门本科课中选择6门进行考试,只要认真准备,一般都没什么问题。科大的ECE的科研方向包括了无线通信、IC、半导体器件、光波导、生物光学、显示技术(LCD、OLED等)、图像处理、Robotics(大疆无人机)、solar cell、神经信号探测等,具体可以查阅科大ECE的官网。

    \subsubsection*{您在读研期间经历过哪些实习/科研,它们的体验是怎样的?}
    读研期间的体验其实很大情况取决于导师的风格,每个导师都有自己的考虑与出发点,有的导师鼓励学生实习跟去国外的学校exchange,但是有的导师会认为这些会打乱学生的研究计划,所以关于这方面,需要根据所在科研组的具体环境来决定。因此,也想在这里提醒大家,可以先联系你们心仪的科研组的学生了解下,当然也不要仅仅根据这些来挑选科研组,还是要看自己的兴趣以及科研方向在未来的发展。

    \subsubsection*{您现在回顾当初选择留学、选择专业的初衷,在经历了留学生活后有什么新的感受?}
    首先关于是否选择留学,大家毕竟都已成年,也会有自己的考虑与判断,考虑清楚,大胆决策就好。关于专业,我建议大家可以提早进行思考,最幸运的情况是你非常了解自己的兴趣点,那就很方便选择了,如果你只是有个大概的方向,也是OK的,读博期间实在不喜欢也还是有其他的出路,比如转成master毕业之后申请其他的学校或者转组。

    \subsubsection*{您选择了在香港攻读博士学位,它和其它选项相比优劣有哪些?就以博士为最终学位的学术深造而言您在出国深造国家的选择上对学弟学妹有什么建议?}
    在哪里读,真得就没有很多经验了,完全取决于自己的考虑。香港感觉就像是大陆跟国外的中转站,毕竟离大陆很近,很多事情都很方便,而且也够开放,跟国外的科研组还是有很多合作交流,所以毕业之后再去国外相对来说也还方便。不过香港毕竟是以金融为主的,工业界相对比较薄弱。

    \subsubsection*{对于以博士学位为最终学位的深造而言,先在国内读master作为跳板,国外读master作为跳板和本科直接申请Ph.D.三者上有何优劣?您就此对学弟学妹的建议是什么?}
    其实各有优劣吧,个人觉得考虑时间成本的话还是本科直博比较划算,因为很多组对于读博之前有没有master degree看的不是很重,本科申请的话,也尽量可以在本科期间发表一些文章,journal或者conference paper都会有帮助。国内跟国外读master还是推荐在条件允许的情况下,首选国外master,而且是你想读博的学校的专业的master,这样方便跟professor套磁,成功的机会也会大很多。在国内读master的话尽量多发些文章,对于申请比较重要。

    \subsubsection*{目前南大自动化类的同学每年的出国比例都要低于南大本科生出国比例平均值不少,您认为造成这一现象的原因是什么?您对南大自动化(类)在读的学弟学妹们在出国读研方向的选择上有什么建议吗?}
    可能是因为南大的强项是理科跟文科吧,工科发展的比较晚,没有形成气候。不过,也不必感伤,大胆申请就可以了,EE跟CS方向都可以的,而且各自的细分方向也有很多。

    \subsubsection*{从现在看留学时光,您会给即将开始留学生涯的大四学弟学妹们什么建议?}
    放松心态就好,做好适合自己的规划。

    \subsubsection*{就您的了解而言,您目前所读专业的Ph.D.未来发展前景如何,有哪些方向(学术界和工业界),您个人更期待哪一个方向?对于这些方向,在Ph.D.在读期间该做哪些准备?}
    我的专业方向是关于液晶显示与光学器件的。现在主要的小屏,像是手机,有向OLED (organic LED)发展的趋势 (三星的大部分手机跟iPhone X的显示屏都是OLED),大屏的话还是以LCD (Liquid crystal display)为主,当然还有QLED (quantum dot LED)、Micro LED等。目前看得话,未来OLED会是一个趋势,大家可以考虑下。

\clearpage
\subsection{侯跃南(IE, Ph.D.@CUHK)}
    \textbf{留学教育背景:}香港中文大学,信息工程(IE),2017\\
    \textbf{联系方式:}wechat (Hou\_yuenan\_nju)\\
    \textbf{个人主页:}linkedin (HOU Yuenan)

    \subsubsection*{在香港中文大学读信息工程是怎样的体验?}
    当初在考虑是否读博的时候,还是有些小纠结的,毕竟想像中的博士生活还是非常辛苦的。不过,当真正来到中文大学来读书时,我发现这里的学习生活还是比较轻松的。虽然说有时候赶会议的截稿日期的时候会有些累,但是平时的课程学习生活还是很友好的。中文大学对于课程的设置非常多样,而且硬性的考试要求也比较低,这样就可以选择自己喜欢的课。相比于本科时代紧张的考试周,博士阶段的学习生活还是相当nice的。科研的话,很庆幸自己能选择到最近很热门的信息工程专业,也希望自己博士这几年能多为这个领域做点实质的贡献。至于平时的生活,除了饮食以外,其他方面和大陆没有太大的差异。总之,我还是很庆幸自己能来中文大学读书的。

    \subsubsection*{您在读研期间经历过哪些实习/科研,它们的体验是怎样的?}
    之前在香港商汤科技公司实习,做无人驾驶的部分技术研发,感觉还是学到不少东西的。不过很惭愧,和周围的同事相比,自己的能力水平还是非常有限的。个人觉得以后要去科技公司做技术方面的研发,博士阶段还是得多学点编程方面的知识,尽可能接触不同的研发语言(tensorflow、torch、pytorch、caffe等),这样以后去公司才能更加得心应手。

    \subsubsection*{您现在回顾当初选择留学、选择专业的初衷,在经历了留学生活后有什么新的感受?}
    当初还是挺纠结是否要来香港读博士的,不过现在还是挺开心自己选择来中文大学读博士的。博士阶段要做的项目还是有一些难度的,正好可以锻炼自己的实际能力。平时的meeting就是和导师的“斗智斗勇” (开个小玩笑啦\^{}\_\^{} ),毕竟有时候需要说服导师做某个方向的。总之,我觉得对于有条件出国读书的同学,来香港读博士还是非常nice的。年轻的时候多出去折腾折腾,人生才有滋味呀!

    \subsubsection*{您选择了在香港攻读博士学位,它和其它选项相比优劣有哪些?就以博士为最终学位的学术深造而言您在出国深造国家的选择上对学弟学妹有什么建议?}
    个人觉得来香港或者新加坡读博士还是非常有意义的,而且生活方面也不会有太大变化。其实读博士也不一定以后就从事科研这条道路,毕竟很多公司也有研发岗位。出国读博士还是很有用的,毕竟国外的研究氛围更浓厚一些,科研经费也更充足(我们实验室就是人均8个GPU\^{}\_\^{})。

    \subsubsection*{就您的了解而言,您目前所读专业的Ph.D.未来发展前景如何,有哪些方向(学术界和工业界),您个人更期待哪一个方向?对于这些方向,在Ph.D.在读期间该做哪些准备?}
    信息工程专业还是很热门的,所以说不会太担心毕业以后的就业问题。至于毕业以后的选择,我个人会更加倾向于学术界,不过这取决于我博士阶段的研究成果了。如果高质量paper不够的话,可能就会选择工业界了。至于怎么准备的话,那当然是多做项目多发paper啦!
    
    \subsubsection{想说的话:}
    趁着年轻,多出去折腾折腾 \^{}\_\^{} !

\clearpage
\section{新加坡篇}

\subsection{陈孟机(ECE, Ph.D.@NUS)}
    \textbf{留学教育背景:}新加坡国立大学,电子与计算机工程系博士(under prof. Yang Hyunsoo),2015年8月入学\\
    \textbf{联系方式:}邮箱: chen.mengji93@gmail.com

    \subsubsection*{在新加坡国立大学读电子与计算机工程系是怎样的体验?}
    我就读于新加坡国立大学电气与计算机工程系。博士生通常只有前面三个学期需要上课。除了专业课之外,还有修读英语和研究伦理与道德这样的通识教育课。国大的课程压力对博士生来讲还是比较适中的,课程既包括了本专业的核心课(比如微电子方向大家都会选择量子力学),也包括了更高级的能和研究相关的课程(比如我的研究方向是磁学,就会有专门一门讲授磁学的课程)。博士生的课程基本都是小班教学,每堂课的课容量适中。几乎每一门课都包括课堂讨论和小组展示,展示内容基本都是和个人研究领域相关,有助于快速掌握一个人对某一研究方向的了解。期末考试的题目比较困难,通常需要提前两个礼拜到一个月开始准备,即使是这样,考试的时候答不完题目也是经常出现的。总而言之,课程和考试反而是博士期间最轻松的事情了。\\
    任何一个学校的科研过程都是时间长且枯燥的,国大也是一样。从开始接受一个项目,拿到一整套可以发表的数据,论证分析数据的合理性,写成论文投稿,回复严苛的审稿人到最后看到自己的成果发表。这个过程无疑是漫长痛苦枯燥的。但是,当自己能够拿到自己想要的数据,分析不了的数据突然间豁然开朗的时候,当看到自己的成果发表在一个相对不错的杂志,得到同行评攒的那一刻,自己的内心还是很满足的。读博士本来就是一个修炼的过程,这个过程不只是对研究领域的不断深入与了解,你还会学到如何与各种仪器以及材料供应商打交道,如何自信的站在台前把自己的研究成果传达给其他人,如何优雅的回复邮件,和来自各个国家的人沟通交流。我一直觉得,博士是对一个人基于特定的研究方向全方位的培养。博士期间所培养的严谨,快速学习已经沟通交流的能力,是一个人在本科和硕士期间接受不到的,也让我坚信是以后走上工作岗位的利器。诚然,对于一个博士生而言,研究占据了一大部分时间,如果遇上比较严格的导师,心理压力也会很大。所以适当的娱乐运动和社交,以及每年适当的假期出游,可以给本来就单调的博士生活添点儿彩。引用我学姐在博士论文中写的一句话与大家共勉:\\“They say that if life hands you lemons, make lemonade. They also say that anything worth doing is worth doing well, and when push came to shove, it turns out that I went on and made the best darn lemonade that I could. At the end of the day, though, the journey mattered more than the destination. Anyway, cheers – I hope you enjoy the lemonade!” 

    \subsubsection*{您选择了在新加坡攻读博士学位,它和其它选项相比优劣有哪些?就以博士为最终学位的学术深造而言您在出国深造国家的选择上对学弟学妹有什么建议?}
    在我看来,新加坡读博士相比较于欧美,在某些特定的方面是有优势的。\\
    一, 新加坡的博士理论上来讲是4年毕业,除非你的成果严重不足或者老板刻意延毕(延毕的情况比较少,新加坡国立大学这边的话院系本身不希望你延毕,他们也会帮忙和导师沟通让你正常毕业),相比较美国的至少5年的限定,新加坡Ph.D.项目是十分高效的,而在欧洲,像法国,更是需要研究生学历才能读博士的。\\
    二, 新加坡奖学金是囊括Ph.D.期间四年的,是由新加坡政府出钱,而且奖学金项目相对来说比较多,足够优秀的人可以尝试申请更高的奖学金,而且一般奖学金足够你的生活所需。(不过这两年开始新加坡减少了国际学生奖学金的名额。中国CSC留学基金委项目是另一个途径,申请来新加坡的通过率似乎比较高。)\\
    三, 说实话,读Ph.D.之前并没有去了解过科研设备实力这一项,但是Ph.D.期间你会发现,设备和经费,是高效科研的重要保障。新加坡国立大学是一个资金充足的学校,相比较而言,美国有些高校的实验室条件根本比不过新加坡(当然,MIT这种除外)。\\
    四, 和欧美相比,新加坡的地理位置离中国较近;和香港相比,其他海外国家的学生更多,更有利于英文交流学习的提升(选中国导师的话可能更多是中文交流,但是非中国导师的话,不仅身边同学同事很多是外国人,在组里的交流都是英文。比如我的导师是个韩国人,我们组里除了中国学生和博后外,还有来自印度,美国,法国,伊朗,韩国等国家的,和他们相处你会感觉到世界各国的不同文化)\\
    而劣势也是非常明显的,世界名校的排名放在那里,意味着科研水平也是有高有低的。而世界排名靠前的学校更多的集中于欧美(虽然新加坡国立大学排名相对靠前,亚洲第一)。这就意味着各个领域的大牛们也相对于集中于这些地方,离大牛们更近意味着你有机会和他们接触合作,这对于你的学术水平的提高是显而易见的优势。以新加坡国立大学为例,他们自身聘请的教授老师,更多的是来自欧美学校毕业的博士,这也侧面印证了这些学校毕业的博士生的科研水平和成就。

    \subsubsection*{对于以博士学位为最终学位的深造而言,先在国内读master作为跳板,国外读master作为跳板和本科直接申请Ph.D.三者上有何优劣?您就此对学弟学妹的建议是什么?}
    就我个人而言,在国内读一个master再申请Ph.D.是一个更加有利于自己规划人生的途径。  \\
    当回首已经完成的Ph.D.时光,我自己会不禁感叹是如何走到今天的。因为身为本科毕业就直接来新加坡读Ph.D.的学生,我们很可能缺少了对科研本身的认识。科研不是本科时候上上课,考考试就能毕业的,科研也不是临时抱佛脚就能出成果的,科研更不是你努力了就一定会有好回报的。因此,在踏入Ph.D.这条路的时候,思想准备必须做足。\\
    在国内读master的话,是可以累积一定的科研经历的,这个科研经历会让你认识到你所选择的方向是否适合你以及科研是否适合你,而这两个问题正是很多海外Ph.D.第一年经常会思考的问题,而且很多Ph.D.因为对这个问题太多纠结和烦恼,最后选择放弃Ph.D.生涯。因此,当你在国内读master的时候,经过一年两年,你依旧有走学术路线的欲望的时候,一段Ph.D.历程是你未来的方向。国外的Ph.D.更多的是导师提供一个平台,你需要自己去探索和研究,因此国内master经历会让你在这方面更加游刃有余。而出国读master的话,授课型master是很少会接触科研这一块的,所以对于你是否选择Ph.D. 这条路的指引相对较少。\\
    相比较而言,直接读Ph.D.是有风险,但是时间短,跳过了国内master的三年。如果你选择了直接读Ph.D.可能就要多花点时间去适应这个从本科到Ph.D.转变的过程。而一些学校支持你Ph.D.读一半在过资格考试的时候可以选择降级为master从而提前毕业。\\
    总而言之,读Ph.D.最好是要经过认真思考,读什么专业,研究什么方向更是要慎重。这是对你自己负责。而我最大的失误就在于一开始对Ph.D.的认识不深,以为和master差不多只是要花两年时间。因此我花费了一到两年的时候去过心理这关,也导致前两年的效率不高。尽管和自己预期的不一样,但Ph.D.生活也切切实实的教会我,这是一个学习的过程,每个人在这个期间都在学会各种技能。比如做事情的有条理性,一个系统性的实验研究才能出来一份卓越的科研成果,而这也是各大企业看中博士毕业文凭的一个重要因素。当你毕业之时,你已经是一个技能点加满的状态,在这个领域,你是属于最顶尖的那群人。

    \subsubsection*{目前南大自动化类的同学每年的出国比例都要低于南大本科生出国比例平均值不少,您认为造成这一现象的原因是什么?您对南大自动化(类)在读的学弟学妹们在出国读研方向的选择上有什么建议吗?}
    我个人认为可能是从一开始就缺少这种想法吧,至少我当年接触身边同学的时候,管科同学明显有更强烈的出国欲望,而自动化这边相对较少。因此,条件允许的话,越早定下这么一个出国的目标为之奋斗越好。 我希望有机会大家需要出去走走看看,不是外面有多好,而是外面总会有不一样,值得我们去学习。\\
    对于出国读Ph.D.的话,在专业选择上,我觉得首先要认清自己毕业了之后想要做什么。进公司还是继续做科研?如果你选择做科研,我希望你找一个你有兴趣的方向,因为科研是一个漫长的过程, 一个有兴趣的科研,只要你有动力,脚踏实地,无论如何都会有结果,只是说这项工作的影响大小有区别(发nature等顶刊还是说一些小杂志。)而进公司工作不一样,申请工作时你需要考虑你的专业的就业面(搞通信和搞半导体的,我个人认为区别就很大,后者更局限),Ph.D.期间你相当于是提前进入了工作准备,在这里你学习各项技能,提前为你进入社会工作做准备。
                    
    \subsubsection{想说的话:}
    Ph.D.生涯是一个至少4年的生活,在这个更大的平台上,我认识了来自不同地方的同龄段的Ph.D.朋友们,也认识了很多比我年长比我更有能力的博后们,还有各种学术上颇有建树的教授们。我看到了他们对科研的热情以及态度。我希望有更多对科研有兴趣的学弟学妹们能勇敢的选择这条路,能和我一样在这个过程中学会接受各种挑战,并越走越远。这是一条不平凡的路,等待着不平凡的你们。


\clearpage
\subsection{董善东(ECE, Ph.D.@NUS)}
    \textbf{留学教育背景:}新加坡国立大学, ECE, Ph.D., 2015 \\
    \textbf{联系方式:}1003544927@qq.com

    \subsubsection*{在National University of Singapore读Electrical and Computer Engineering是怎样的体验?}
    新加坡被称为花园城市,城市非常干净,而且景色也是很不错的。\\
    气候方面,新加坡是典型的热带国家,全年夏天。不过有分雨季和旱季,雨季的话基本每天下雨,温度正好合适。旱季的时候温度稍微有点高,要是没有空调的话,晚上睡觉会感觉有点热。\\
    文化上面,由于华裔和中国人居多,大概70\%左右,因此很容易融入进去。生活上新加坡和国内有点相似,总体饭菜还是很合中国人的口味。交通也是很方便,基本公交+地铁可以到达新加坡的任何地方。\\    
    科研方面,由于新加坡大力引进欧美的教授,加上一流的实验设备,因此科研环境还是不错的。而博士一般学制为4年,因此相比美国很多学校来讲,压力会小一些。相比于其他院系,ECE的老师还是相对比较有人情味,大部份同学都会4年毕业。\\
    课程方面,新加坡的博士需要修完6门课程,不过由于光纤通信/传感的课程比较少,我最后选修了挺多数学或计算机类的课程来完成要求。\\
    缺点的话,就是新加坡太小了,加上没有季节的变化,生活时间长了,会觉得比较枯燥无聊。

    \subsubsection*{您在读研期间经历过哪些实习/科研,它们的体验是怎样的?}
    科研:
    我博士的方向是光纤传感器的设计。我导师人很nice,给予了我很大的支持帮助和自由。在光纤传感领域,基本还是以实验为主(理论已经很成熟,继续做难度比较大),实验进度和结果相对可控。不过由于我个人不是特别喜爱做科研,因此在博士第二年就开始考虑博士之后的出路。15年底-16年初,适逢机器学习的兴起,正好我对算法比较感兴趣,因此我就和导师积极交流,把研究课题往机器学习在光纤传感的应用做了一些倾斜。因此,读博期间,如果有任何想法或者问题,我建议学弟学妹们积极的和导师去沟通交流。\\
    实习:
    在今年也是有幸去腾讯实习了2个月,实习内容为深度学习在异常检测中的应用研究。腾讯给我最大的感觉就是文化很自由,基本没有特定的打卡制度,有时候有急事需要请假,和领导说下基本也就可以了。福利方面也是很不错的,晚上加班的话会有一些相对应的福利(夜宵券,打车报销),也有覆盖全市的免费班车等。工作上面,在腾讯最大的感受就是以业务为核心,人人都要有产品经理的思维,而技术只是为业务做服务的,这也对我以前技术为上的想法有一些改变。\\
    回顾整个过程,我的感觉是如果对科研很感兴趣+方向也很喜欢,毕业了希望继续从事Faculty的路线,还是要沉下心来,多读paper,多做实验,多思考。如果对方向不太喜爱,则要早点去和导师进行沟通与交流,看看能不能做一些你比较喜爱的课题。\\
    如果说真的对科研没啥兴趣,那就应该早点去思考自己喜欢的事情,做好未来的规划。由于博士处于一个半工作,半学生的状态,因此有大量的时间去进行未来的思考,去找到自己感兴趣的点,去思索清楚自己到底想要成为什么样的人,这或许意义更大。这也是我博士期间最大的收获。

    \subsubsection*{您现在回顾当初选择留学、选择专业的初衷,在经历了留学生活后有什么新的感受?}
    其实我大一大二的时候,和大多数的自动化类同学一样,从来没有过出国的想法。后来也是机缘巧合,我们上一届的师兄们有很多去新加坡香港等地读博的,我们也就跟着这个路线去了。读了三年博士之后,再去回想这些选择,觉得当时还是没有考虑清楚,纯粹是觉得出国这个选择不错就去了。不过也不会觉得后悔,毕竟也是人生很好的经历,而且现在慢慢清晰了自己的想要做的事情。

    \subsubsection*{您选择了在新加坡攻读博士学位,它和其它选项相比优劣有哪些?就以博士为最终学位的学术深造而言您在出国深造国家的选择上对学弟学妹有什么建议?}
    新加坡在很多人眼里,是中国和欧美之间的一个缓和地,也有挺多学生拿新加坡作为一个跳板,以后继续去欧美的。所以如果纯粹的想做科研,个人建议还是直接去美国,毕竟出来的认同感会更强烈一些。如果只是想读个博,以后进入工业界,那么新加坡和香港还是不错的选择。

    \subsubsection*{对于以博士学位为最终学位的深造而言,先在国内读master作为跳板,国外读master作为跳板和本科直接申请Ph.D.三者上有何优劣?您就此对学弟学妹的建议是什么?}
    这几个各有利弊把。\\
    如果以博士为最终目标,一定要读完博士,我觉得本科可以好好准备下,早点选择一个比较感兴趣的方向,跟着导师做一些科研尝试,然后直接申请国外的博士,这样时间成本会相对比较低,性价比更高。\\
    当然如果读完master,再去读博,一方面更了解自己是不是想读博,另外一方面对研究可能理解的更全面,做的更好。\\
    对于读博,我个人的建议就是一定要想清楚自己为什么要读博,到底想不想读博。因为读博的过程一般来说是比较枯燥的,很多时候心理压力也会比较大。所以如果不是特别热爱科研或者为了以后走Faculty 路线,我觉得直接读个master然后工作也是挺好的。

    \subsubsection*{目前南大自动化类的同学每年的出国比例都要低于南大本科生出国比例平均值不少,您认为造成这一现象的原因是什么?您对南大自动化(类)在读的学弟学妹们在出国读研方向的选择上有什么建议吗?}
    可能是意识+周围环境把。一方面,对比金工,自动化类的同学一开始有出国意识的并不多,会觉得出国比较难,比较烧钱。另外一方面,周围自动化的同学出国的也不多,老师们也很少会去宣传出国的事情。因此自己没有一个明确的,可供选择的路线。拿我自己来讲,当时也是看到师兄们出国读博,我也才知道,原来出国读博是可行的方案。\\
    对于想出国读研或者读博的,我个人的建议还是早做准备,一方面从大一开始就得注意GPA的提升,早点准备托福/雅思等英语考试,另外一方面则是除了自己申请以外,也需要多关注一些学校级别的合作机会。

    \subsubsection*{从现在看留学时光,您会给即将开始留学生涯的大四学弟学妹们什么建议?}
    没有什么特别的建议,好好玩耍把。



\clearpage
\section{欧洲篇}

\subsection{李研亭(CS, MS@Télécom ParisTech)}
    \textbf{留学教育背景:}巴黎高科项目 Télécom ParisTech, 
    专业:Data Science, Distributed Software Systems, 
    2016入学\\
    \textbf{联系方式:}微信:15861817168

    \subsubsection*{在Télécom ParisTech就读是怎样的体验?}
    ParisTech 9+9 项目和南大合作其实已经有十多年,只是往年申请的学生大多来自匡院,物理,数学,生科等基础学科院系,该项目在我们院知名度并不是很高。我最初了解到巴黎高科项目是在南大交换生网站上,随后也去翻阅了其他院系前几届的飞跃手册。学长学姐们的经验确实在申请过程中给了我很大帮助。我本人选择ParisTech是因为相比于科研,我更倾向于做应用性强一些的工作。而刚好法国的工程师教育(diplôme d’ingénieur)提供了优质的课程,教育非常注重实践,课上讲授的内容基本上都是最新的和企业用得到的。另外也是因为刚好可以借此机会学习法语。ParisTech的12所学校中,Télécom,ENSTA和IOGS都适合自动化或信工的同学申请,有兴趣可以去交换生网站或者ParisTech 9+9网站了解更多信息。\\
    我所在的Télécom主要为cs方向,针对9+9项目的学生学制为一年半课程+半年实习(法国工程师教育一般为三年,École Polytechique四年,9+9项目学生从二年级读起)。二年级需要在13个专业中选择两个专业,每个专业8门课。一学年的课程分为4个阶段。除专业课以外,还需要修7到8门的通识课(计算机方向的通识课,例如统计,优化,编程范式等),两种语言课,以及一年两次每次一周的人文教育课,一年两次的为期一周的交换课程。三年级选择一个方向继续,可以选择去其他学校读双学位,上半年上课下半年实习。由于Grande école的职业化导向,学校经常有各大企业的conference,午餐,以及求职论坛。在学校时就有较多的接触企业的机会,找实习也相对容易。\\
    学制及课程以外,我最大的感受是Télécom为学生考虑非常周到。入学前办各种各样的活动帮助外国学生融入(包括人文课老师带着逛附近的街区等等),上课时老师也对外国学生比较照顾(刚开始会经常问有没有听懂没懂可以随时打断),课程内容很实用,课程形式注重实践(一般都是一半时间上课一半时间写代码或者习题课),上面提到的和企业接触的机会,以及会有专门负责国际学生签证居留等程序的老师。\\
    学校以外,学生时期法国政府会提供住房补助,博物馆大多免费,交通卡优惠很多,以及其他大大小小针对学生的福利。住在巴黎平时出门就能看到历史遗迹,博物馆美术馆非常方便,地理位置以及法国的假期非常有利于旅行。
    
    \subsubsection*{您在读研期间经历过哪些实习/科研,它们的体验是怎样的?}
    写这篇飞跃笔记的时候我已经开始毕业实习三个月,由于本科期间没有过实习经历,这是我第一份实习。法国的工作环境和职场环境相对轻松,每周工作35小时,晚上和周末的时间完全属于自己。带我的导师和同事都很热心,在工作中大大小小的问题上都帮过我很多。我的老板会愿意让我做一些自己感兴趣的内容,也会愿意花不少时间给我讲他认为会对我有用的东西。\\
    实习对于语言提高来说是一个特别棒的机会。对我来说在学校期间其实大部分时间还是和高科项目一起过来的中国同学一起,大家一起会感觉很温暖很踏实,但同时其实并没有给自己一个纯法语环境。实习中因为经常要和导师沟通,另外不少法国人都会有工作间歇喝咖啡聊天的习惯,一段时间下来会能明显感觉到语言上的进步。
    
    \subsubsection*{您现在回顾当初选择留学、选择专业的初衷,在经历了留学生活后有什么新的感受?}
    当初选择留学其实并没有考虑特别多,一是因为本科课程期间慢慢发现自己比较喜欢计算机,考虑课程质量,二是想要体验一下不一样的学习生活,不想留下遗憾,于是决定出国。更早一些,大二大三时在交换生网站上注意到巴黎高科项目,后来也翻过计科物院匡院的飞跃手册,看到当时的学长学姐对学校对项目以及留学生活的描述就觉得很期待,觉得那就是我想要的。另外也是希望再学一门外语,比较喜欢法语,刚好巴黎高科项目也为我提供了这样一个机会。\\
    到现在快两年的时间,不管是法国还是专业,对于自己当初的选择从来没有后悔过。只是回头来看会觉得很长一段时间自己没有有意识地给自己提供更多的法语环境,课程上也并没有尽力,会觉得有些遗憾。不过不后悔的是经历过一段时间后自己心理上比起之前独立很多,也经常出去走走看看,给自己一些新的体验(潜水攀冰等等)。

    \subsubsection*{目前南大自动化类的同学每年的出国比例都要低于南大本科生出国比例平均值不少,您认为造成这一现象的原因是什么?您对南大自动化(类)在读的学弟学妹们在出国读研方向的选择上有什么建议吗?}
    和匡院物院相比,我个人认为我们院出国比例低的原因一是一直以来选择出国的比例都比较小,并没有太多的经验;二是可能院系内于国外学校接触的机会比较少。总体而言应该是大家对于留学这个选项的了解并不是太多。\\
    其实我有认识一些特别棒的学弟学妹,他们在本科学习以及申请方面比我当初做得好很多,相信他们也会有很棒的建议。只有一点,我觉得读研方向的选择其实也是多一个选择专业的机会,要选择自己真正感兴趣的。留学生活可能对有些人来说和国内差不多,但也有不少人会经历一个比较难的阶段。选择一个喜欢的专业首先会有一个让自己努力的动力,其次至少专业是自己喜欢的,可能其他方面即便遇到一些问题也不会显得那么难。
    
    \subsubsection*{从现在看留学时光,您会给即将开始留学生涯的大四学弟学妹们什么建议?}
    还是看过一些学弟学妹申请的学校,觉得大家都特别棒。我想说的可能更偏向于自己经历过两年留学生活之后的一些心得。我觉得要清楚自己留学想要的是什么,融入这个词也许有些太大了,但是如果想要真正体验留学生活的环境,还是要勇敢一点给自己创造环境。其实生活在国外都不太容易,尤其是一个人,因此特别容易,或者说自然而然地就会呆在一个中国人的圈子里(可能对我来说大家在出国前一起学了半年法语,更容易出现这种情况)。
    另外就是要抓紧机会让自己多一些新的体验,多交些朋友。除了专业以外,有机会做一些自己没做过的事,认识一些很棒的人,也是留学生活中很有价值的一部分。

    \subsubsection*{您目前打算毕业后直接就业还是继续读博深造?对于您master之后的去向选择而言,在国内读研和在国外读研有何优劣?就您目前打算的方向而言(读博/工作),在master期间需要做一些怎样的准备?}
    选择工程师学校的初衷就是想要毕业后直接就业。我认为在国外读研,课程方面学到的内容会更实在,接触企业的机会也会比在国内读研要多。如果已经有了就业方向,master期间可以参加一些比赛多一些经验。如果对将来工作的方向还不是很明确的话,比方说有两三个感兴趣的方向,其实可以考虑在读研期间gap一年去做两段实习。另外可以早一些参加学校举办的各种和找实习找工作相关的活动,比如说招聘会以及修改简历之类的,完全可以在无压力的轻松的状态下早一点了解企业招人的要求,程序等等,其实为了也是早一点有个比较明确的目标。
        
    \subsubsection{想说的话:}
    两年不短,其实在回答上面每个问题的时候都没写得太细,是怕会说得太杂或者万一并不是大家关注的重点。但是如果有学弟学妹想要了解关于法国关于巴黎高科或者上面任何回答中有感兴趣的地方,欢迎私戳。


\clearpage
\subsection{王思菡(EM, MS@KCL)}
    \textbf{留学教育背景:}伦敦大学国王学院(KCL),Engineering with Management,2013年入学\\
    \textbf{联系方式:}wangxiaobian@live.cn

    \subsubsection*{在伦敦大学国王学院读Engineering with Management是怎样的体验?}
        \begin{enumerate}[itemindent=0pt,itemsep=0pt,parsep=0pt]
            \item 课程内容对于南大的同学来说应该是比较好把握的,有很多Lab,基本上每天都要去机房做各种编程,大部分情况包括最后的Project都是用Matlab,偶尔也会在Linux环境下编程。少量课程会有很难的课题(全班没一个人完全编出来的那种),但是绝大部分情况下都比较简单,结合Matlab的帮助和查阅资料就能独立完成。
            \item 会遇到一些学习能力稍欠缺的同学,他们经常来问一些简单问题(或者借作业抄),身为南大学渣本渣产生了一种自己是学霸的幻觉。
            \item KCL的工科课程更侧重于人工智能和机器学习,几乎所有的工科课程和project最后都指向这两点。2013年国内的AI行业还不算热门,但英国高校的AI研究已经非常普遍了,课程设置从最基础的运筹学(和本科课程完全重合)到机器人、计算机视觉、模式识别等等非常全面。虽然工科并非KCL的强项,但师资不错,Professor Maria Fox和Professor Derek Long据说是人工智能研究的Top5之二。
            \item 生活上,由于学校地处伦敦市中心的中心,四周环绕着各种地标、商圈、剧院等等,喜欢人文历史的同学会非常喜欢这里,生活也非常便利。(一年两次的伦敦时装周就在学校隔壁举行,上下学穿梭在各种模特明星摄影师中,这种经历也是很特别了。)
        \end{enumerate}

    \subsubsection*{您在读研期间经历过哪些实习/科研,它们的体验是怎样的?}
    因为一直听说在英国找到工作的几率不高,所以在校期间投了两个实习没有回应也就没有继续找了。\\
    能算得上科研的就是最后一个学期的Project了,研究模糊控制的算法,刚开始感觉很困难,课程基础是远远不够的。好在professor是个工作狂人,几乎一天24小时都在做研究,咨询问题的邮件都回复得非常快。感受是对于前沿算法的研究,一个学期的project可能刚刚够入门,很难有突破,更适合Ph.D.阶段要继续该方向研究的学术大神,打酱油的话还是选择应用类的课题,能收获更多。另外professor的选择很重要,有同学不幸选到了非常任性的老师,第二个学期考试结束就去度假了,学生发过去的邮件石沉大海,最后只能自由发挥。所以喜欢自学的同学最好还是去上课了解一下老师。

    \subsubsection*{您现在回顾当初选择留学、选择专业的初衷,在经历了留学生活后有什么新的感受?}
    英国master的一年半对我而言更像Gap Year,主要是想体会不一样的生活,了解世界各地的人(学习是第二位的,所以没有很努力去做研究或者为毕业找工作)。对比国外人们的行为习惯,更容易反省在国内不容易发现的自身的不足,改变一些长期在国内生活所形成的刻板印象,最大的收获是获得了发自内心的自信和空前开阔的心境。

    \subsubsection*{您选择了在英国读master,请问您的选择相比于国内和其他国家地区而言有什么优劣呢?}
    其他国家不太了解,但最明显的缺点是留英工作和移民相对困难。优点是后续深造的申请应该比国内简单,尤其是申请本校的Ph.D.几乎没有难度(我在申请Ph.D.的时候,professor给我秒发了unconditional offer)。

    \subsubsection*{目前南大自动化类的同学每年的出国比例都要低于南大本科生出国比例平均值不少,您认为造成这一现象的原因是什么?您对南大自动化(类)在读的学弟学妹们在出国读研方向的选择上有什么建议吗?}
    自动化类的同学毕业后选择很多,工作机会也很多。所以不必像文理学科,追求很高的学历。如果没有明确的出国目的(移民、深造等),大可不必随大流地出国,本科毕业参加工作,也可以有很好的发展。反之,可能会在留学期间产生“浪费时间”的消极情绪。同理,在专业选择上,在满足学校要求的前提下,尽可能地选择自己想学习的专业,个人认为不必追求与本科专业对口,相反,如果在这个阶段有自己感兴趣的其他的专业方向,这是一个很好的扩宽知识面的机会。
    \subsubsection*{从现在看留学时光,您会给即将开始留学生涯的大四学弟学妹们什么建议?}
        \begin{enumerate}[itemindent=0pt,itemsep=0pt,parsep=0pt]
            \item 中国留学生在国外是非常容易抱团的(这也是我一开始选择KCL的原因,因为听说KCL的中国留学生比较少,结果去到之后依然是一个班有一半的中国学生),抱团的后果就是大家在一起说中文,不会有外国同学参与到你们的互动中来了(练口语的机会就没了)。所以如果想结识一些其他国家的朋友,积极主动地与他们沟通吧。
            \item 提前从学长学姐那边了解老师,哪些老师口音太重你可能完全要靠自学,哪些老师考试非常变态,哪些老师的课值得一听,哪些老师可能对亚洲学生有偏见等等。
            \item 不光是留学生活在改变我们,我们也同样在影响着外国人眼中的中国人形象,要比在国内的时候更加注意公共场合的言行得体。
        \end{enumerate}
        
    \subsubsection*{您目前打算毕业后直接就业还是继续读博深造?对于您master之后的去向选择而言,在国内读研和在国外读研有何优劣?就您目前打算的方向而言(读博/工作),在master期间需要做一些怎样的准备?}
    个人认为,无论是在国内读研还是国外读研,对于毕业后参加工作而非继续进行科研的人来说,它最大的意义不在于学术本身,而在于面对难题时,解决问题的能力和方法论的形成,这是我认为在master期间要重点训练的——找到适合自己的快速学习一门新知识的方法。工作是一个全新的开始,无论是读完了本科硕士还是博士,参加工作之后面对的都是实际工作中具体问题,而这个时候学习的方法论,比知识储备更为重要。   

\clearpage
\section{澳洲篇}

\subsection{陈东阳(ME, MS@ANU)}
    \textbf{留学教育背景:}澳洲国立大学,机电一体化,2016年入学\\
    \textbf{联系方式:} 微信:lelouchcc911

    \subsubsection*{在Australia National University读机电一体化是怎样的体验?}
    我读的是ANU授课型master,虽说是授课型但与南大的授课完全不是一回事。一学期一共四门课,但包含lecture,tutorial,lab三种需要参加的授课形式。Lecture与本科没有太大区别,主要是老师讲课学生听。Tutorial是为了巩固学生对lecture内容而由讲师或tutor组织的小班答疑课程。Lab则是帮助学生将理论转化为实践,并为学期末的project做准备。ANU Engineering专业基本每门课最终成绩都是由lab + 期中考试 + 期末考试 + final project + assignments 计算出来的,有时候甚至期末还没考这门课就已经过了,但也要注意平时分获取不够多的情况下期末压力会非常大。
    生活上澳洲是个非常不错的选择(适合养老),尤其是堪培拉。作为首都一点都没有首都盛气凌人的气势(其实就是村儿的不行)。晚上6点以后商场所有店铺基本都关了,9点所有餐厅都关了,所以喜欢丰富夜生活的孩子们还是别来了,两年下来会闷出事儿的。但是对于喜欢清静喜欢自然的同学澳国立是个不错的选择。一到了晚上车辆特别少,压马路甚至可以滚着压,所以有种说法,澳国立的同学都喜欢学习,因为除了学习没事可做。 
    \vspace{3ex}

    % \begin{figure}[!h]
    % 	\centering
    % 	\includegraphics[width=0.5\textwidth, height=0.32\textwidth]{kanpeila.png}
    % 	\caption{傍晚5点的堪培拉街道}
    % 	\label{kanpeila}
    % \end{figure}
    \centerline{\includegraphics[width=11cm]{kanpeila.png}}
    \centerline{傍晚5点的堪培拉街道}
    \centerline{\includegraphics[width=11cm]{grifenlake.png}}
    \centerline{格里芬湖}
    \subsubsection*{您在读研期间经历过哪些实习/科研,它们的体验是怎样的?}
    可以说每一门课程都是一个挑战。但最重要的一点,Engineering的master必须学会自学。我刚来ANU的第一个学期,带着本科的一点C语言和一点MATLAB底子,上来就碰上了python编程为基础的Robotics。这边默认每一位master同学都有良好的python基础,所以每一个assignment都是大量的google,同时还要自学python,linux,安装虚拟机,其中出现了各种各样的问题。每一个周六日都在学校和同学探讨研究。如果没做好吃苦的准备,恐怕现实会非常沉重。但熬过来就发现自己的成长要远比课堂上听课迅速。

    \subsubsection*{您现在回顾当初选择留学、选择专业的初衷,在经历了留学生活后有什么新的感受?}
    当初打算留学其实就是为了开拓下视野,同时两年时间获取国外研究生学位会在将来就业上带来更大的竞争力。但现在发现竞争力还是没多大(掩面哭泣)。不过有着这两年的国外生活经验,我自认为适应能力,团队协作能力,英语口语都有些进步,这也许才是最大的收获吧。

    \subsubsection*{目前南大自动化类的同学每年的出国比例都要低于南大本科生出国比例平均值不少,您认为造成这一现象的原因是什么?您对南大自动化(类)在读的学弟学妹们在出国读研方向的选择上有什么建议吗?}
    南大整体来讲出国比例都不是特别高,我觉得这种原因还是学校的氛围问题。熟悉的学长学姐都保研或考研,那自己走到这一步时可能就只会想着我要考研/保研。但如果相熟的学长学姐大都去了国外,朋友圈里都是国外的生活,那或许选择会完全不同。同时学长学姐的留学经验也可以帮助学弟学妹少走弯路,这可以说是大学文化里的另一种传承。所以我很看好这个飞跃手册,希望可以帮到学弟学妹!\\
    在选择上,主要还是根据实际情况来,家庭条件允许或有条件公费出国的话,出国是个不错的选择。条件相对不是太优越而硬要砸锅卖铁出国其实也没有太大意义,国内毕竟可以享受众多实习资源。都要回国工作的话,实习经历的重要性不言而喻。

    \subsubsection*{从现在看留学时光,您会给即将开始留学生涯的大四学弟学妹们什么建议?}
    那就给点实实在在的建议,如果你们来读master的话:
    \begin{enumerate}
        \item 时间紧的话申请完全可以靠自己,中介效率太低。自己申请也可以锻炼阅读能力。
        \item 来到这边一定要拓展人脉,尽快认识并了解自己同专业的同学,在开学各种课程组队的时候可以避免很多坑队友(都是泪)。
        \item 多自学多google,不懂不要慌,都是这么走过来的。
        \item 觉得刚到这里英语可能跟不上的话上课前把老师的ppt下载下来,提前预习查生词。
        \item 学会独立生活,身体是革命的本钱。我一个队友第一学期来了之后完全不会独立生活,第三个星期就犯胃病,家长过来陪读,第一学期结束就回国了。
        \item 遇到问题及时和老师沟通,这边老师都很敬业,发邮件很快就会回复。
        \item 租房多留心,有条件住宿舍绝对是最省心的。
        \item 多给家里打打电话,发发照片。
    \end{enumerate}

    \subsubsection*{您目前打算毕业后直接就业还是继续读博深造?对于您master之后的去向选择而言,在国内读研和在国外读研有何优劣?就您目前打算的方向而言(读博/工作),在master期间需要做一些怎样的准备?}
    毕业后打算直接回国就业。国内读研有实习机会但平时要完成导师的任务。国外读研自由度高,锻炼能力,但实习机会难得。所以出国读master的话一定要好好利用寒暑假,澳洲的暑假有3个月,用这段时间实习是一个很好的选择。

    \subsubsection{想说的话:}
    每一个选择都是你当时做出最适合自己的决定,所以永远不要后悔。愿学弟学妹不忘初心,各自辉煌。






\clearpage
\subsection{XXX(, Ph.D., MS@XXX)}
    \textbf{留学教育背景:}\\
    \textbf{联系方式:}XXX

    \subsubsection*{在XXX University读XXX专业是怎样的体验?}
        \begin{enumerate}[itemindent=0pt,itemsep=0pt,parsep=0pt]
            \item 难道是
            \item 因为
        \end{enumerate}
    \subsubsection*{您在读研期间经历过哪些实习/科研,它们的体验是怎样的?}

    \subsubsection*{您现在回顾当初选择留学、选择专业的初衷,在经历了留学生活后有什么新的感受?}

    \subsubsection*{[Ph.D.]您选择了在(香港/新加坡/美国/欧洲/…)攻读博士学位,它和其它选项相比优劣有哪些?就以博士为最终学位的学术深造而言您在出国深造国家的选择上对学弟学妹有什么建议?}

    \subsubsection*{[Master]您选择了在(香港/新加坡/美国/欧洲/…)读master,请问您的选择相比于国内和其他国家地区而言有什么优劣呢?}

    \subsubsection*{[Ph.D.]对于以博士学位为最终学位的深造而言,先在国内读master作为跳板,国外读master作为跳板和本科直接申请Ph.D.三者上有何优劣?您就此对学弟学妹的建议是什么?}

    \subsubsection*{目前南大自动化类的同学每年的出国比例都要低于南大本科生出国比例平均值不少,您认为造成这一现象的原因是什么?您对南大自动化(类)在读的学弟学妹们在出国读研方向的选择上有什么建议吗?}

    \subsubsection*{从现在看留学时光,您会给即将开始留学生涯的大四学弟学妹们什么建议?}

    \subsubsection*{[Ph.D.]就您的了解而言,您目前所读专业的Ph.D.未来发展前景如何,有哪些方向(学术界和工业界),您个人更期待哪一个方向?对于这些方向,在Ph.D.在读期间该做哪些准备?}

    \subsubsection*{[Master]您目前打算毕业后直接就业还是继续读博深造?对于您master之后的去向选择而言,在国内读研和在国外读研有何优劣?就您目前打算的方向而言(读博/工作),在master期间需要做一些怎样的准备?}
    
    \subsubsection{想说的话:}









\chapter{2018届申请总结}

\clearpage
\section{徐旭东(IE-CV, Ph.D.@CUHK)}
    \subsection*{个人背景}
        \textbf{GPA:} overall 92/100, major 92/100 \\
        \textbf{Ranking:} 1/37\\
        \textbf{TOEFL(R/L/S/W):} 105(29/26/24/27)\\
        \textbf{GRE(V+Q+AW):} 无\\
        \textbf{推荐信:} 自动化系内导师陈春林、赵佳宝强推。\\
        \textbf{科研/实习/交流:} 项目一,用深度强化学习的方法做机械臂抓取;\\ $\qquad$ 项目二,研究量子强化学习,利用量子机制加速学习过程。\\ 
        \textbf{联系方式:} Email: forrest.tusi@gmail.com WeChat: Sheldon9710   \\
        \textbf{homepage: } \url{https://sheldontsui.github.io}, 

    \subsection*{申请结果}
        \textbf{Accept:} CUHK

    \subsection*{申请的前期准备}
        \subsubsection*{选择}
        刚进入南大的时候,我只想安安静静地做一个普通的学生,能够顺利毕业,努把力考个研即可。但是,学长学姐们努力拼搏的故事感召了我(在此,感谢在前引路的学长学姐们)。另外,参与创新项目之后,个人觉得做科研比较有意思。自此,我走上了一条不归路。目标从考研到保研,从keep No. 1到出国读Ph.D.。到大三末的时候,我发现最后一个目标实现不了。原因有很多:出国准备较晚、没有交换/暑研经历、美帝CV的Ph.D.比较难申请。失败就要吸取教训,下面重点总结一下。

        \subsubsection*{GPA}
        4.6/5.0
        \subsubsection*{科研/实习/比赛/交流}
        科研项目两个/无实习/三个比赛,成绩还行/无交流经历(非常遗憾)。\\
        我和工管后面几届的学弟学妹们接触过,发现好多同学都想转CS(跳入CS的大坑)。个人觉得,美国CS的Ph.D.的含金量非常高,强烈建议赴美读博。但是,如今美国CS的Ph.D.的申请现状越来越严峻。由于Ph.D.时间长,培养难度大,导师在押宝招收Ph.D.的时候,会越来越看重推荐信,特别是著名实验室的强推信,有时候一封大佬的强推信比一篇顶会论文还管用。因此,强烈建议早进组,多交换,进优秀实验室,混大佬的强推信。具体地,只要早进组接触科研,自然会了解很多业内的著名实验室,到时候,暑研/实习都可以进入这些实验室,推荐信就顺利多了。
        \subsubsection*{GRE/TOEFL}
        无/105

    \subsection*{申请过程}
        \subsubsection*{选校}
        只选择了CUHK(因为离家近,实验室还不错,Ph.D.时间短,录取时比较看重GPA和科研经历,好吧,说实话,美帝的CV方向Ph.D.太难了)。\\
        关于选择导师,我大三上的时候仔细咨询了侯跃南学长,具体了解了香港中文大学多媒体实验室各位导师(汤晓鸥、王晓刚、林达华等,可以自行Google),以及其在计算机视觉届的地位。我只能说,he is not the best, but the most suitable. 我们实验室,在业界挺有名,进去也不简单。但是,香港的Ph.D.只有4年时间,我还是强烈推荐美国的Ph.D.。

        \subsubsection*{文书}
        CUHK不太看重文书,照着模板自己写的。

        \subsubsection*{推荐信}
        自动化系内导师强推。

        \subsubsection*{网申}
        只要有导师要你了,网申就是走形式。

        \subsubsection*{项目介绍}
        4年Ph.D.,比较水。


\clearpage
\section{李志成(EIE, MS@CityU)}
    \subsection*{个人背景}
        \textbf{GPA:} overall 80.8/100, major 79/100\\
        \textbf{Ranking:} 21/37\\
        \textbf{推荐信:} 无\\
        \textbf{科研/实习/交流:} 无\\ 
        \textbf{联系方式:} lizhicheng26@outlook.com/ WeChat :chilk0

    \subsection*{申请结果}
        \textbf{AD:} 香港城市大学\\
        \textbf{Rej:} Nil\\
        \textbf{Accept:} 香港城市大学, Electronic Information Engineering

    \subsection*{申请的前期准备}
        \subsubsection*{选择}
        MS, 出国。
        \subsubsection*{GPA}
        没有重新刷学分绩。
        \subsubsection*{科研/实习/比赛/交流}
        参加微电老鼠比赛。
        \subsubsection*{GRE/TOEFL}

    \subsection*{申请过程}
        \subsubsection*{选校}
        选取不需要雅思托福成绩的学校。
        \subsubsection*{文书}
        DIY
        \subsubsection*{网申}
        选择需要申请的学校后在官网了解各个项目的申请截止时间,按照步骤完成。一般的学校都会提供申请步骤说明,只要认真阅读,DIY很简单。
        \subsubsection*{项目介绍}
        摘自官网:\\
        The MSc programme aims to provide an enhancement of specialist technical knowledge in strategically selected areas, such as RF \& microwaves circuits \& applied electromagnetics, wireless communications, networking \& data communications, and optical communications that are relevant to the current and anticipated future needs from industry. Due to the rapid advances of existing technologies, the programme is designed for practicing engineers who want to upgrade and update their knowledge in this field.\\
        The programme offers a good selection of business management electives, preparing engineers with knowledge on top of their technical fronts to take up managerial positions in their engineering career.\\
        The programme also offers Industrial or Research Internship experience via the Applied Research Internship Scheme preparing students for full employment immediately after their research internships, or for further studies as Ph.D. candidates in their related technology areas either locally or in overseas countries.

    \subsection*{总结}
    我的经历比较适合找工作不合心意,还想继续深造但没有提前准备语言考试和刷GPA的同学,一般不要借鉴。

\clearpage
\section{练仕琪(CS, MS@WUSTL)}
    \subsection*{个人背景}
        \textbf{GPA:} overall 4.2+/5, major 如果按照cs课程算的话大概在4.3,4.4之间\\
        \textbf{Ranking:} N/A\\
        \textbf{TOEFL(R/L/S/W):} 102(28/26/23/25)\\
        \textbf{GRE(V+Q+AW):} 152+167+3\\
        \textbf{推荐信:} 李华雄老师,张益昕老师(现工院),陈林老师(计算机系)\\
        \textbf{科研/实习/交流:} 无实习科研交流,建模美赛H奖,全国教育机器人特等奖\\ 
        \textbf{联系方式:} sqilian@gmail.com

    \subsection*{申请结果}
        \textbf{AD:} wustl/uf/brandeis/utd\\
        \textbf{Rej:} ucsd/rice/purdue/neu/uva/nyu/osu/psu/jhu/uci\\
        \textbf{Withdraw:} tamu(太晚没有结果)\\
        \textbf{Accept:} wustl\\
        以上学校的专业除了UVA同时申请了CE(免费加申。。。)和CS之外,全部都是CS。

    \subsection*{申请的前期准备}
        \subsubsection*{选择}
        我选择出国的决定是在大三开学的时候,那时候我的GT都完全没有任何准备,GPA也比较糟糕,我为申请的准备全部是在大三和大四第一学期完成的。

        \subsubsection*{GPA}
        我在大二暑假决定出国转cs之前,overall GPA大约在4.1左右。大三重修了不少科目外加跨专业选修大量计算机系课程才把overall拉到85左右。作为南大出国申请者的一员,我的GPA也算是中等偏下了。从17年秋季开始,重修的课程不能覆盖原成绩,学弟学妹们更需要认真对待第一次考试。我在cs申请上相对于其他转专业同学最大的优势是较为全面的cs课程背景。除了我系开的cs相关课程之外,我还选修了算法,OS,编译,离散数学,计算机网络,Java等等,可以说是cover了所有cs本科课程背景了。不过这相比于申请上的帮助,我认为一个全面扎实的基础知识体系对于研究生阶段学习乃至就业或深造的帮助更为重要。

        \subsubsection*{科研/实习/比赛/交流}
        我申请的时候软背景非常差,科研实习交流三无。只有两个用处不大的比赛奖项。由于种种原因,我们自动化学生在科研实习上普遍空缺,系里相关资源也不足。对于大学期间定好目标出国的同学,我的建议是可以试试跨院系去计科电子自己联系老师做科研,实习也要自己去关心起来。不能因为周围人不做或是院系没有提供相关资源就偷懒不做。如果你想要本科申请Ph.D.(尤其是美国)的话,没有好的科研经历是难以弥补的致命伤。就算是你有极高的GPA,挽救的作用也非常有限。而好的实习经历对于master申请也是有极大帮助的。

        \subsubsection*{GRE/TOEFL}
        我GT开始准备已经是大三寒假,而大三第二学期忙着选修计算机系课程和重修实际上并没有在英语上花时间。所以我GT实际上是在大四第一学期申请期间才花时间考了好几次考出了勉强够用的成绩,希望诸位有充足时间的话还是尽早准备。一般理工科申请的各项硬指标重要性来说,GPA $\gg$ T > G。如果临近申请实在时间不够的话可以适当在GRE上少花时间。

    \subsection*{申请过程}
        \subsubsection*{选校}
        我的选校是在十月份开始申请时就基本完成了。主要是参考了往年的计算机系和软件学院的飞跃手册。事实证明,我的这个方法有很大的问题。因为前两年计算机系和软院录取友好的学校在今年一定会被计算机系和软院的同学申爆,美国是一个特别讲diversity的地方,一般一所学校出身的人都会实际上内部竞争来确保名额配比平衡。我作为背景普通跨专业转CS的申请者,面对科班出身的竞争者无疑是处于劣势的,这也是我今年申请结果不佳的重要原因之一。

        \subsubsection*{文书}
        由于缺少足够软背景,文书自然也是乏善可陈。我在PS/SOP中主要介绍了我对CS专业和本专业的理解和选择,两次比赛的经历收获以及一个计科课程的大作业。我的文书请了芝士圈留学的文书修改服务和认识的native speaker修改了两次。\\
        关于文书的重要性,我个人理解文书在很多项目里是非常重要。在实际申请的过程中,有不少项目都会被冠以玄学录取的名号,其实所谓的玄学录取就是参考了不少你的文书里呈现的软背景。一旦一个项目的录取没有和申请者三维呈现强相关性就会认为是玄学录取(典型如MSCS@UMN,MSCS@UVA等),但是在后来的观察中我发现,这些项目正是更加关注了申请者难以被量化的软背景。所以大家一定要给予文书充分的重视。但正所谓巧妇难为无米之炊,好的文书也是建立在丰富的软背景上的,这就又回到了大学期间科研实习积累的根本问题上。

        \subsubsection*{推荐信}
        我的推荐信老师都是任课老师或者竞赛指导老师,只能算中规中矩吧。一般普通的国内老师推荐信在美国硕士申请中的作用可以忽略。但是如果旨在申请顶尖的美国Ph.D.项目(如cs四大),推荐信就成了最重要的一项因素。一份好的推荐信的作用甚至要比出色的GPA和科研更为重要,想要获得好的推荐信,就需要本科期间自己去寻找美国的研究机会,这在我系的背景环境下也是非常不容易的,加油吧诸位。

        \subsubsection*{网申}
        这个没啥好说的,赶在ddl前按部就班提交材料,仔细填写核对即可。

        \subsubsection*{项目介绍}
        从我的结果可以看出,我今年的申请结果并不好。我是属于全申cs的头铁型选手,而美国CS申请难度大大高于其他任何专业且在每年水涨船高。经过种种考虑,我最后选择了MSCS@WUSTL。这是一个非常标准的两年MSCS项目,进可科研读博,退可就业找工。学校的优势在于较小的项目人数规模,这意味着在找工投简历时会面临较小的diversity筛选(对比USC等cs人数超多的学校),如果想要科研转博,每个人也更容易分配到研究机会。华大的CS专排在普通的top50档,专排不高的原因是CS系的规模不大(情况相类似的还有jhu,Dartmouth等),但是华大的综排较高(今年usnews ranking 18名,去年是14名),所以学校在美国的声誉还是很好的。华大的医学极强(仅次于哈佛和jhu),所以工科和医学相关的方向也会非常强。主要缺点当然是地理位置,糟糕的地理位置造成的负面影响在寻找小公司就业的时候会凸现出来,但是对想要读博的同学影响不大。此外,华大MSCS的学费也是被人诟病的一点。不过据我的了解,华大的费用水平并没有网传的那么高,如果能第一年多修课程一年半毕业的话实际总费用和rice的一年半cs项目差不多,两年的话总费用会高一些,但也不会高于nyu,哥大的cs项目。值得一提的是,华大的cs对我校计算机系非常不友好(详情参考16,17年的计科飞跃手册统计),但对我系总体还行,南大自动化系的08级系友Yan, Hao学长就在华大读CS博士。

    \subsection*{总结}
    14级同学们一起发起南大自动化类单独的飞跃手册的初衷,不是劝进或者劝退各位学弟学妹出国,而是希望能让大家多一点对未来选择的思考。咱们南大自动化的学生,平均上来说家庭条件不如同校的商科学生,加上系里缺少出国相关的信息和指导,往往对于出国的选项了解甚少。如果是以博士为目标的留学申请,家庭经济条件因素是完全不用担心的,因为绝大多数国外Ph.D.项目都是全额奖学金,我们需要做的只有提升自身实力以满足申请条件。如果是以硕士为目标的申请,英美国家的花费较高,但是欧洲一些国家的硕士项目和加拿大学术性硕士项目也是提供奖学金的。在英语能力方面,其实GT考试得分还是以应试为主,相信以诸位高考考上南大的学习能力来说,只要投入充分学习时间,达到理工科出国申请的英语基本要求绝非难事。\\
    在自动化同学申请方向选择上,一般根据申请难度要求来说,美国Ph.D. > 加拿大/欧洲/港校 Ph.D. > 美国 master > 其他国家地区 master。而在专业选择上,当下工业界出路最好的专业无疑是CS(尤其是在北美),北美的申请难度也是CS $\gg$ 其他专业。一般来说申请北美cs方向学校都会比其他方向降一到两个档次,所以如果你对cs兴趣一般而且没有在北美就业的需求的话就没有必要和千军万马一起去挤CS的独木桥了。具体内容大家可以参考国外在读或是毕业学长们的飞跃经验,我就不在这里继续班门弄斧了。\\
    最后希望各位一定要在大学期间多多思考自己未来的路该怎么走,切忌躲在自己的舒适圈里用战术上的勤奋掩盖自己战略上的懒惰。
    Good Luck,Have Fun!
    

\clearpage
\section{王心佐(CS, MS@UVa)}
    \subsection*{个人背景}
        \textbf{GPA:} overall 84.6/100, major 86/100\\
        \textbf{Ranking:} Unknown\\
        \textbf{TOEFL(R/L/S/W):} 102(28/26/23/25)\\
        \textbf{GRE(V+Q+AW):} 152+170+3\\
        \textbf{推荐信:} 副教授,副教授,实习上司\\
        \textbf{科研/实习/交流:} 跟导师做的大创项目,出成果有四篇在投论文,两篇专利,实习在一家网络安全公司做SDE\\ 
        \textbf{联系方式:} QQ:124279963

    \subsection*{申请结果}
        \textbf{AD:} CS@UVa, ECE@Cornell, JHU, UCSD, NEU, CE@NU, USC, TAMU, EE@Columbia, USC\\
        \textbf{Rej:} CS@USC, NEU, ECE@UMich, CMU, UT Austin, Brown, Duke\\
        \textbf{Accept:} CS@University of Virginia

    \subsection*{申请的前期准备}
        \subsubsection*{选择}
        MS出国。
        \subsubsection*{GPA}
        大四疯狂重修拉高。
        \subsubsection*{科研/实习/比赛/交流}
        科研做了三年,稍微有点东西,实习是大四找的工作,比赛水(可以视为无),交流无。
        \subsubsection*{GRE/TOEFL}
        半年时间随便刷了刷,够用就行。

    \subsection*{申请过程}
        \subsubsection*{选校}
        8月。
        \subsubsection*{文书}
        9月-11月。
        \subsubsection*{推荐信}
        11月-1月。
        \subsubsection*{网申}
        11月-4月。
        \subsubsection*{项目介绍}
        本来打算从了康奈尔但是uva给了个小奖cs,想想cs前景更好就去了这个项目。

    \subsection*{总结}
    本科多学点cs很重要,英语越早考越好,没必要请中介,科研稍微来一点无伤大雅,保留Ph.D.的可能性。最后一句最重要-GPA一定要保持住!最最少86+才稳一些。



\clearpage
\section{张恒源(ECE-ISRC, MS@UCSD)}
    \subsection*{个人背景}
        \textbf{GPA:} overall 86.4/100, major 85.3/100\\
        \textbf{Ranking:} 11/36(不计算重修)\\
        \textbf{TOEFL(R/L/S/W):} 109(30/29/23/27)\\
        \textbf{GRE(V+Q+AW):} 155+170+3.0\\
        \textbf{推荐信:} 陈春林,赵佳宝和李华雄三位老师\\
        \textbf{科研/实习/交流:} 光电大赛飞行器调试、实验室飞行器项目经历,无实习交流。\\ 
        \textbf{联系方式:} 邮箱:zhy19951213@126.com/ 微信:HenryZhangWechat

    \subsection*{申请结果}
        \textbf{AD:} ECE@UCSD、CS@TAMU、ECE@OSU、CE@UCSC、EE@MSU、CE@ASU \\
        \textbf{Rej:} Robotics@CMU、Robotics@Umich、Robotics@Upenn、CS@Gatech、CS@USC、CS@UMass、CS@Oregon State、CE@VT\\
        \textbf{Accept:} ECE@University of California San Diego

    \subsection*{申请的前期准备}
        \subsubsection*{选择}
        大一的时候并没有出国的打算,想要直接就业。但在大二下鬼使神差看了一次Google I/O的keynote,被工程师们流露出的工程改变世界的情怀所点燃,突然非常想要去另一个国度去学习,与有这样精神的团队一起工作,那时候便有了出国的想法。有了想法之后慢慢地也就和身边准备出国的同学开始了交流分享信息,听过一些国外的公开课,感觉确实很喜欢一些老师的讲授方式,所以慢慢下定了决心,到大四得知有保研机会的时候也就没有犹豫了。
        \subsubsection*{GPA}
        GPA本身是一道线,但这里面最重要的是数学、物理、电路、计算机这些硬课。然而不巧的是,我大一大二的软肋恰是这些(不务正业$\sim$)。当大三开始着手准备出国时,重修的几门数学电路给我带来了很大的负担。但是重学的时候也确实会有新的收获,比如可以换一个班选一个好的老师认真地听下去这样的经验只有对比过后才能知道有多大不同。下定决心出国后专心于课程学习和出国准备,大三的专业成绩有了很大的进步,听闻学校有时也会更注重最后两年的成绩,所以这一点可能会对之前的GPA劣势有一定的补救。\\
        参与活动与学业的平衡一直是我没有解决的问题。大一大二时学校会有丰富的活动机会,这无疑会减少你在图书馆自习的时间,分散你的精力。但对于我来说,最要好的外专业朋友都是在那个时候通过各种活动结识,大学生活也因为这些同伴和与他们一起经历的事而无比丰富。虽然确实花费了时间,但这对于我比GPA更重要(当然,查看你成绩单的国外老师可能不这么认为$\sim$)。

        \subsubsection*{科研/实习/比赛/交流}
        大二的暑假留校和同学们一起做光电大赛是第一次的比赛经历。虽然我的小组没能被选拔代表学校参赛,但是我选择继续配合其他组调试飞行器。在那个假期静下心来对着巨大的飞控工程文件试图理解它的结构和运作方式,并且将自己的理解与其他同伴分享,感受到了莫大的快乐。之后慢慢地能够看懂项目的架构并且做出一定的修改,为以后面对多文件的大项目打下了基础也增加了信心。\\
        之后在实验室担任小组长负责一个飞行器项目的组织。我们做了相当一部分的培训工作,但是在科研方面没有深入下去,所以最后也没有什么成果。

        \subsubsection*{GRE/TOEFL}
        首先应该强调的是英语在学校考察时没那么重要,我直到在网上申请的时候才发现大多数学校对理工科生的英语没有太硬的要求,所以前期的重点还是应当在GPA和科研经历。但在之前已经为英语付出了很大的精力,写下我的经历供参考。\\
        我个人的英语背景是二层次、旁听多一层次的英语课并且前五个学期都有选修一些其他院系英语授课的课程来保持感觉、如商院的组织管理、新传的英语新闻报道、通识课当代西方人文精神概论等。托福一战的时间很晚,预约的是大三下的4月1日。这个时候很多同学都已经考了几次了。我准备大约用一个月,没有去上培训班,准备方式就是在图书馆五楼练打字提高作文字数,早上上课前在教室练口语和晚上在自习室做TPO的阅读听力。第一次考试是100/120分,不高但是够用,加上大三下那个学期必修、重修和选修课程有七门硬课,所以就先放下了托福。学期中间和好友一起参加了一个两周的TFT口语小组,因为每天都有背诵任务,强度很高,所以有一定的帮助,但线上课程主要还是靠自己自觉,结伴上课会更有状态。\\
        大三暑假开始准备GRE,在八月前因为暑期课程等,只零零散散背了一千个单词,是小伙伴推荐的APP墨墨背单词,八月份基本按照每天100个新增词的速度背要你命三千,结合词根词缀记忆词典记忆,但是由于速度过慢,到8月25日考试前都没怎么做过阅读和verbal的实战训练。所以一战考出来155的verbal其实是非常侥幸的。数学部分做了网上的一些难题集,把比较坑的点过一下170没什么问题。作文实在没有时间准备,所以只有3.0。GRE的325对于我的条件和期望的学校算是够用的分数,所以也就没有再进行准备。TOEFL在9月17日考的第二次中得到了109分,主要是在GRE备战之后因为单词量的提升,阅读变得轻松了一些。作文随着学会了盲打(真的是准备TOEFL练出来的!推荐:\url{https://www.keybr.com})字数上去了成绩也有所提高。\\
        英语准备总结:根据自己的起点进行准备,报班不是必须的。尽早开始、尽早报名(如果不是那么壕的话交了1600+的报名费之后你会认真的)、先GRE后托福对阅读有帮助、但是口语基础不好的话要尽早开始训练,可以组团训练(缅怀一下口语自救小队$\sim$)。

    \subsection*{申请过程}
        \subsubsection*{选校}
        选校工作主要是对着自己感兴趣的方向找学校。我的兴趣方向是无人驾驶,这个可以从Robotics、CS、EE的角度做。所以我申请的方向也就是这些。一方面我在一亩三分地里面找以前申请Robotics相关学校的同学给出的选校信息。另一方面是在Google上直接搜索相关的学校,还有就是直接在US News上找对应专业的排名筛选。\\
        结合这三者确定一个大致的学校范围之后,开始逐个查看这些学校官网给出的相关实验室的介绍和研究生项目介绍进行筛选。并且将自己选择的学校按照档次进行划分来平衡选择(冲刺、主申、保底)。\\
        对于自己的评估可以参考一亩三分地往年的申请汇报、可以对比申请人和自己的条件、看得多了也就能猜得差不多。由于很多人这个时候基本都在担心自己会没学上,论坛上也常有焦虑传播,所以容易低估自己。个人感觉是保底校不要太多,有的你申请到了也不会看一眼的。冲刺可以多一些。不留遗憾。\\
        时间安排上很多学校在12月15日截止,我的建议是如果要细致考察,那么选校要在大四的暑假准备。但是即使晚了也不用太担心,截止前两周也是足够的。\\
        对于想要学习Robotics这个方向的同学,由于这个领域的交叉比较强,所以可能在不同学校中院系设置不同。Robotics学院开设学校比较少,很多是在ECE或者是CS的一个分支下,所以对于目标学校的CS、ECE都要关注一下,在官网看看他的实验室设置。我的兴趣方向是无人驾驶,自动化的课程设置给我们打下了软硬件的基础,加上大三大四我也选修了一些计算机专业的课程,所以CS、EE的课程基本都可以接受。所以我选择的范围是有车辆机器人相关实验室的Robotics、CS、EE。
        
        \subsubsection*{文书}
        写Statement of Purpose是对我帮助最大的环节,因为需要回答自己必须要面对的问题:我究竟想要什么?这个学校的项目将如何帮我达成?为什么这个项目应当选择培养我?当你能够对自己回答清楚这些问题的时候,其实文书也就完成了。大约在图书馆用了两周的时间写文书,修改到不想再修改就打印出来自己读一读或者请同学看一看然后继续改。最后提交文书前针对每个学校的具体要求修改结构,并且在最后一段加上了感兴趣实验室和研究方向。

        \subsubsection*{推荐信}
        一般而言实验室的工作经历是最有帮助的,但我大三的精力分配主要在课程上提高GPA,在实验室没有什么成果,所以并不能够用成果支撑推荐信。但是因为一直在当团队负责人的工作,所以和导师赵佳宝老师沟通较多,得到了赵老师的支持。陈春林老师和李华雄老师在我发邮件询问后给了热心的回复。十几所学校的申请对于老师来说工作量是极大的,他们在繁忙的教学科研之余抽出时间实属不易。我们院系虽然老师不多,但老师们都希望帮助自己栽培的学生一把,推他们到更高的地方,非常感激。

        \subsubsection*{网申}
        确定学校准备好材料后就是网申。这一部分主要是细心地在ddl之前完成网上填表就好。我把各个学校的网申放到一起来做,所以真正在填写的时间不长,主要还是对着要求修改文书。预计需要一到两周。有一些常用信息可以先用excel表格整理好然后再填写。如果使用浏览器自动补全表单一定要记得核查!网申阶段一直用excel表格管理每个学校的进度和账号密码以及一些特殊要求,对自己梳理很有帮助。

        \subsubsection*{等待结果}
        由于从一亩三分地分析得出的定位确实比较准确,收到了冲刺校们的rej和保底校们的AD,有了第一个AD后,确实是没怎么关注过这个保底校,直到等到Ohio State University的ECE才开始考虑。这个时候收到学校要求3.月15答复,但很多结果还没出,所以在同伴的建议下写了封很诚恳的信表明自己很多申请结果还没有出,难以抉择的处境,得到了延期答复的准许。所以鼓励大家面对问题要勇敢地争取$\sim$在三月下旬收到TAMU的CS录取,其实已经很满意了,进一步了解情况,正准备去的时候,三月底得到了UCSD的ECE喜欢的track的录取,也就果断决定了。我等待录取的过程其实很幸运,恰好是一个结果比一个满意。身边也有同伴奋斗到五月份才收到梦校的录取,非常焦灼,真的很不容易,大家一定要保持良好的心态面对任何可能的结果。

        \subsubsection*{项目介绍}
        首先申明,由于还没有去学校,以下内容只是我所收集到的信息,不保证准确性。\\
        UCSD工科很好,ECE和CS在US News排名都在 10-20 之间。ECE分为很多track,我选择的是ISRC(Intelligent system, Robotics, Control)是和自动化比较对口的方向。根据官网的信息,我们的项目是比较偏理论的,适合于有研究志向的人为了Ph.D.做准备。UCSD的学制是quarter制,一年一般上三个quarter,每个十一周左右,所以会节奏很快,好处是暑假很长。根据学长的介绍,一般大家会需要5个quarter完成所有学分的修读,也就是一年半。\\
        ISRC方向属于ECE中比较偏软的方向,会有很多想要转码的小伙伴。由于Robotics理论性较强,想要深入研究还是建议转Ph.D.。现在(mark:2018)人工智能比较火,所以这个方向会持续有更多人加入。我们这级中国学生,这个方向已经有三十余人,整个ECE的中国学生两百人左右,所以项目的规模还是挺大的。\\
        UCSD气候宜人、生活环境好、学校周边比较安全,大概是个适合长居读Ph.D.的好地方 $\sim$

    \subsection*{总结}
        申请的过程就是功到自然成,文书推荐信网申等等工作只是对自己前几年大学学习生活的呈现。如果没有前期良好的铺垫,面对无法改变的GPA和贫乏的经历,纵有万丈豪情,也难以在竞争中超过其他申请者。至于提交之后获得的结果,AD还是Rej,接受就好。

\clearpage
\section{李睿(ECE, MS@Gatech)}
    \subsection*{个人背景}
        \textbf{GPA:} overall 4.3/5.0, major的没算过\\
        \textbf{Ranking:} 2/11\\
        \textbf{TOEFL(R/L/S/W): } 105 (29/29/24/23)\\
        \textbf{GRE(V+Q+AW):} 160+170+4.0\\
        \textbf{推荐信:} 宋跃江老师一封,李密老师一封,Umass AP一封,电子学院施院长一封,清华大学教授一封。\\
        \textbf{科研/实习/交流:} 科研方面,跟宋老师那边做微腔,然后有一篇文章投了没录(跪,不过反正也是挂名;跟李密老师做的激光通信那一块,投了一篇IEEE的会议,自己的通讯作者;什么创新创业计划啦etc;实习方面,大一暑假在一个行业内的国企水水地实习过,不过也写上去了;交流方面,大二暑假去了APRU ULP,是在墨西哥的蒙特雷科技大学举办的一个summer camp,大三下在UC Berkeley交换。还有一些会议啊吧啦吧啦奖什么的也都水水地写上去了。\\ 
        \textbf{联系方式:} wechat: ripray

    \subsection*{申请结果}
        \textbf{AD:} Gatech, UW-Seattle, Columbia, NYU, UCI, GWU\\
        \textbf{Rej:} UCLA, Stanford, Caltech\\
        \textbf{Accept:} Gatech\\
        以上,全部EE MS

    \subsection*{申请的前期准备}
        \subsubsection*{选择}
        一开始就想申MS的,当时还没有读Ph.D.的想法。怎么说呢,出国这个事情感觉是个很自然的选择,因为身边的同龄人十个有八个都是出国的,所以从大一开始就决定出国了吧,也没有考虑别的选择。

        \subsubsection*{GPA}
        GPA我其实是很一般的= =感觉自己不太会学习?哈哈哈哈哈。

        \subsubsection*{科研/实习/比赛/交流}
        尽可能多参加一下,其实我的这部分感觉还是挺欠缺的,因为太宅了;不过对于交流自己很上心,对这方面非常有兴趣,当初南大吸引我也是说交流项目很多。要说的是!南大的机会真的很多,但是大家可能都没有注意到;每学期初的时候会放下一学期的交换交流项目,多上国际处的网页看看(\url{http://stuex.nju.edu.cn});非常推荐出去交换一下,从各种意义上都会带来很正面的影响。

        \subsubsection*{GRE/TOEFL}
        英语这个事情就是要花时间呀,然后就是要根据自己的情况去准备复习;尽量早准备吧。GRE是自己准备的,当时也是时间很晚了,孤注一掷,结果还行,分享一个自己的准备心得吧:\url{http://www.1point3acres.com/bbs/thread-293954-1-1.html}

    \subsection*{申请过程}
        \subsubsection*{选校}
        选校这个事情差不多是当时最担心的了,总担心选校不对没学上。总之就是随喜好选了几所,然后和比较有经验的老师讨论了一下,定下来的选校名单。觉得大家一定(有点)想申的都申啊,最后我还是觉得有几所学校可以多申申。

        \subsubsection*{文书}
        文书确实是在申请过程中比较重要的一环了。因为大四上也不需要再考语言成绩什么的了,当时文书就比较上心。大概花了一个月去写PS和SoP什么的吧,一般来说,是按照比较想去的一所学校写文书,然后其他学校在这基础上进行修改。大概前前后后改了有十几二十稿;但是说实话,文书这个部分到底能起到多大的作用,其实我也不太相信的,但是至少在申请阶段似乎没有别的什么好发力,就在这上面稍微弥补一下吧2333。

        \subsubsection*{推荐信}
        推荐信和文书是配套的;关键点都是要突出自己的核心竞争力,RL和SoP其实相当于是从不同的面来彰显自己的厉害(Nooooo文书和推荐信这些事情简直一说起来没个完,这两个部分我还是花了不少功夫的,具体真的很感兴趣的话可以私聊。\\
        对了推荐一下芝士圈,虽然我一般是在十几稿的最后一两稿再给芝士圈进行最后的润色,不过使用感觉还是挺好的(不是托!)。

        \subsubsection*{网申}
        说起来很麻烦,其实很简单。网申的根本意义其实是简化申请流程,我觉得大多数学校的网申系统都做的很人性化的。

        \subsubsection*{项目介绍}
        也许去了GT之后会有更深刻的体会吧2333不过还是想推一下UW-Seattle!(你个二五仔)\\
        从各个方面来说UW-Seattle都是很好的选择,但是似乎申请热度不高?建议大家试试看嘿嘿嘿=) 

    \subsection*{总结}
    申请真的是一件很麻烦的事情,但是大家还是要相信自己;说实话身边的大家一开始都是有着完了完了没学上的恐慌,但凡是认真准备了的同学都有不错的结果。前期的积累是非常重要的,无论是科研,语言,还是交换,etc.,而到了申请季,无非是将这一部分的自己展现给对方学校而已。祝大家都能有理想的申请结果,有什么问题也可以大力地微信上烦我!欢迎大家来亚特兰大玩耍=)。

\clearpage
\section{XXX(, Ph.D., MS@XXX)}
    \subsection*{个人背景}
        \textbf{GPA:} overall xx/xx, major xx/xx\\
        \textbf{Ranking:} xx/xx\\
        \textbf{TOEFL(R/L/S/W): }xxx (xx/xx/xx/xx)\\
        \textbf{GRE(V+Q+AW):} xx+xx+\\
        \textbf{推荐信:} \\
        \textbf{科研/实习/交流:} \\ 
        \textbf{联系方式:} 

    \subsection*{申请结果}
        \textbf{AD:} \\
        \textbf{Rej:} \\
        \textbf{Accept:} 

    \subsection*{申请的前期准备}
        \subsubsection*{选择}
        \subsubsection*{GPA}
        \subsubsection*{科研/实习/比赛/交流}
        \subsubsection*{GRE/TOEFL}

    \subsection*{申请过程}
        \subsubsection*{选校}
        \subsubsection*{文书}
        \subsubsection*{推荐信}
        \subsubsection*{网申}
        \subsubsection*{等待结果}
        \subsubsection*{项目介绍}

    \subsection*{总结}
    

\chapter{致谢}
手册结构设计参考了计算机系2015年的飞跃手册,在此表达感谢。\\
手册由练仕琪发起、张恒源组织制作,李睿参与设计问卷。徐旭东参与初稿审阅。从开始分发到制作完成历时约一个月。夏娟辅导员提供了很多支持,特别鸣谢。
\end{document}